\subsubsection{La heur\'istica Tab\'u}\label{tabu:explicacion}
\par Ya habiendo pasado por las dos heur\'isticas anteriores y un algoritmo exacto,
    ahora deseamos ver si utilizando una heur\'istica m\'as parametrizada
    se pueden encontrar buenas soluciones al problema en buenos tiempos. Es decir,
    queremos ver si podemos mejorar los resultados sin afectar notablemente la
    eficiencia del algoritmo.

\par La b\'usqueda tab\'u tiene como objetivo tratar de superar los m\'aximos
    (o m\'inimos, dependiendo de si se busca maximizar o minimizar) locales
    que puede tener una vecindad de una b\'usqueda local.

\par Sobre el esquema de la b\'usqueda tab\'u no pretendemos decir mucho, la misma
    ya est\'a hecha y el model aplica a nuestro problema\footnote{\url{%
    http://www.dc.uba.ar/materias/aed3/2013/2c/laboratorio/heuristicas.pdf}}.

\par A modo de clarificaci\'on de terminolog\'ia, y para evitar confusiones,
    se aclara que en el marco de este trabajo nos referimos como ``soluci\'on
    actual'' o ``soluci\'on parcial'' a la soluci\'on que se tiene al entrar
    en cada iteraci\'on del ciclo de b\'usqueda tab\'u. Y llamamos ``soluci\'on
    \'optima parcial'', o por abuso de notaci\'on ``soluci\'on \'optima'' a la
    mejor soluci\'on hallada hasta el momento desde que comenz\'o la b\'usqueda
    tab\'u.

\par En nuestro caso definimos las siguientes caracter\'isticas de nuestra
    b\'usqueda tab\'u:

\begin{itemize}
    \item \textbf{Soluci\'on Inicial: } Proponemos las mismas 2 variantes que
        se utilizaron en la b\'usqueda local\footnote{Secci\'on~\ref{busqueda},
        \emph{\nameref{busqueda}}.};
        comenzar con \textbf{alg\'un nodo de al menos grado promedio} o con
        la \textbf{clique de la heur\'istica golosa}.

    \item \textbf{Vecindad: } Decidimos utilizar la vecindad ``reducida'' de
        la b\'usqueda local. La misma consiste en todas las cliques que tengan
        un nodo m\'as o uno menos respecto de la clique/soluci\'on actual
        que se tiene en la iteraci\'on.

    \item \textbf{Lista Tab\'u: } M\'as all\'a de las estructuras de datos
        que utilizamos para esta \emph{lista} (que se detallan m\'as adelante),
        decidimos guardar los ``movimientos'' que se realizan durante el movimiento
        entre vecinos. Estos movimientos ser\'ian agregar o quitar un nodo%
        \footnote{No se decidi\'o considerar el movimiento/vecindad de intercambio
        de nodos, ya que como ahora se permiten movimientos que no mejoren la
        funci\'on objetivo, estos estar\'ian, de alguna manera, inclu\'idos en
        estos 2 movimientos. Aunque claramente el criterio de selecci\'on no
        es el mismo, ni los saltos de vecindad ser\'ian los mismos y por lo tanto
        los resultados no ser\'ian necesariamente los mismos. Pero el coste
        de recorrer esta vecindad es muy costoso, como ya se vi\'o en
        \emph{\nameref{busqueda}}. A\'un as\'i, queda a criterio
        del lector implementar dicha vecindad y experimentar para poder compara
        los resultados obtenidos.}. Por lo tanto, una vez realizado un movimiento
        (por ejemplo, agregar un nodo $v$ o quitar un nodo $w$) son marcados
        en la lista como movimientos \emph{tabu} y se tratar\'a de evitar
        de repetirlos mientras est\'en en dicha lista y haya otras opciones
        no tab\'u v\'alidas. Queda a\'un pendiente especificar por cu\'anto
        tiempo/iteraciones se considera a un movimiento tab\'u. Esto
        se analizar\'a m\'as adelante.

    \item \textbf{Funci\'on Objetivo: } Al igual que en la b\'usqueda local,
        el criterio/par\'ametro que se desea maximizar es la frontera de
        las cliques de un grafo $G$. Sin necesidad entonces de justificar
        nuevamente, decidimos utilizar la misma funci\'on objetivo.

    \item \textbf{Criterios de Terminaci\'on: } Existen varios criterios
        de terminaci\'on implementables. En particular, decidimos utilizar
        2 criterios:
        \begin{enumerate}
            \item \emph{Limite global de iteraciones de la b\'usqueda: } Simplemente
                deseamos establecer un l\'imite global para el tiempo que puede
                tardar esta heur\'istica. Esto se hace, ya que dependiendo de los
                otros par\'ametros, se podr\'ia llegar a entrar en un ciclo
                muy largo sin que termine la heur\'istica. O, podr\'iamos tener
                una instancia del problema que se pueda mejorar mucho durante
                un per\'iodo muy largo de iteraciones, y si la funci\'on objetivo
                crece (virtualmente) constantemente, llega un punto donde tiene
                m\'as sentido quedarnos con la soluci\'on obtenida hasta el momento
                que iterar un per\'iodo de tiempo muy extendido para no mejorar
                considerablemente la soluci\'on.

            \item \emph{Cantidad de iteraciones sin mejorar la soluci\'on actual: } El
                \'item en si mismo es descriptivo. A\'un asi, decidimos plantear
                dos variantes de este criterio. Podemos considerar que la
                soluci\'on no mejora si la soluci\'on actual no mejora mediante
                decisiones no tab\'u (es decir, si se toma una decisi\'on tab\'u
                ya que no hab\'ia otra opci\'on, y esta mejora la soluci\'on actual,
                no se considera que efectivamente haya mejorado la soluci\'on) o
                podemos considerar cualquier decisi\'on que incremente la
                funci\'on objetivo de la soluci\'on parcial, sea tab\'u o no.%
                \footnote{En particular, existe una alternativa que por falta de
                tiempo no fue implementada, que es considerar a la soluci\'on optima
                en lugar de la soluci\'on actual. O considerar ambas.}
        \end{enumerate}
\end{itemize}

\bigskip

\par En nuestra heur\'istica particular, decidimos inicialmente (antes de implementar
    y experimentar con las variantes) utilizar una funci\'on de aspiraci\'on que nos
    permita saltar a un vecino que mejore nuestra soluci\'on \'optima parcial aunque
    sea este tab\'u. Tambi\'en, decidimos que el contador de iteraciones sin mejoras
    de la soluci\'on actual ser\'ia reestablecido (es decir, volver\'ia cero, o
    equivalemente, consideramos que mejor\'o la soluci\'on actual) solamente cuando
    la funci\'on objetivo se viera incrementada mediante la funci\'on de aspiraci\'on
    o mediante un movimiento \textbf{no} tab\'u. Es decir, si se tiene que realizar
    un movimiento tab\'u que no sea parte de la funci\'on de aspiraci\'on, este contador
    ser\'a incrementado aunque dicho movimiento haya incrementado la funci\'on objetivo.

\subsubsection{Pseudoc\'odigo descriptivo}
\par Se presenta a continuaci\'on un pseudoc\'odigo que describe los pasos que sigue
    nuestra heur\'istica propuesta. El mismo no contempla estructuras de datos sobre
    los cuales se implement\'o (cosa que se realiza en el c\'alculo de complejidad),
    simplemente tiene como objetivo describir la heur\'istica.

\bigskip

\begin{pseudocodigo}[Heur\'istica de Busqueda Tab\'u para \emph{CMF} - Descriptivo]
    \Require\Statex
        \begin{itemize}
            \item Un grafo $G$ de $n$ v\'ertices y $m$ aristas.

            \item Una funci\'on $candidatos(K)$, que dado un conjunto de v\'ertices
                $K$, devuelve el conjunto de v\'ertices de $G$ que son adjacentes
                a todos los elementos de $K$, cuyo grado es mayor $2|K|$.

            \item Una funci\'on $vecinos(K)$, que dada una clique $K$, representa
                el conjunto de todas las cliques $K'$ que sean vecinas de $K$ para
                nuestra vecindad.

            \item Una funci\'on $tabu(K,K')$, que dada una clique $K$ y un vecino $K'$,
                nos dice si el movimiento para pasar de $K$ a $K'$ es tabu.

            \item Una funci\'on $movimiento(K,K')$ que nos devuelve el movimiento
                para pasar de una clique $K$ a su vecina $K'$.
        \end{itemize}
    \Statex
    \Ensure Una \emph{clique} $K$ de $G$ con una frontera $\delta(K)$ que se
        espera que sea de cardinalidad cercana o igual a la de $\delta(K_{max})$,
        siendo $K_{max}$ la \emph{clique} de m\'axima frontera de $G$.

    \Statex

    \If{$G$ es un grafo completo $K_n$}
        \State $K^* \gets \lfloor\sfrac{n}{2}\rfloor$ nodos distintos de $V(G)$ cualesquiera.
        \Statex

    \Else
        \State $K \gets v \in V(G)$ tal que $v$ es un nodo de, al menos, grado promedio.
        \State $\delta(K) \gets d(v)$
        \State $K^* \gets K$
        \State $\delta(K^*) \gets \delta(K)$
        \State $contador\_global \gets 0$, $sin\_mejorar \gets 0$
        \Statex
        \While{%
        $\begin{pmatrix}
            \text{\Huge{$\land$}} &
            \begin{matrix}
                contador\_global < MAX\_GLOBAL\\
                sin\_mejorar < MAX\_SIN\_MEJORAR
            \end{matrix}
        \end{pmatrix}$%
        }
            \Statex
            \State $contador\_global \gets contador\_global+1$
            \Statex
            \State $mejor\_tabu \gets s \in vecinos(K), tabu(K,s)$ tal que
            \Statex\hspace{3cm}$(\forall s' \in vecinos(K))$ $tabu(K,s') \implies \delta(s) \geq \delta(s')$
            \Statex
            \State $mejor\_no\_tabu \gets s \in vecinos(K), \neg tabu(K,s)$ tal que
            \Statex\hspace{3cm}$(\forall s' \in vecinos(K))$ $\neg tabu(K,s') \implies \delta(s) \geq \delta(s')$
            \Statex
            \State $menos\_tabu \gets s \in vecinos(K), tabu(K,s)$ tal que $s$ es la soluci\'on menos tabu%
                \footnote{Donde entendemos por ``menos tabu'' aquella soluci\'on que fue marcada como
                tabu hace m\'as tiempo que todas las dem\'as tab\'u.}
            \Statex
            \If{$%
            \begin{pmatrix}
                \delta(mejor\_tabu) > \delta(mejor\_no\_tabu)\\
                \land\\
                \delta(mejor\_tabu) > \delta(K^*)
            \end{pmatrix}%
            $} \Comment{Funcion de Aspiraci\'on}
                \Statex
                \State Actualizamos $movimiento(K,mejor\_tabu)$ en la lista tab\'u
                \State $K^* \gets mejor\_tabu$
                \State $K \gets mejor\_tabu$
                \State $sin\_mejorar \gets 0$
                \Statex
            \ElsIf{$mejor\_no\_tabu$ es un movimiento v\'alido}
                \If{$\delta(mejor\_no\_tabu) > \delta(K)$}
                    \State $sin\_mejorar \gets 0$
                \Else
                    \State $sin\_mejorar \gets sin\_mejorar+1$
                \EndIf
                \State Actualizamos $movimiento(K,mejor\_no\_tabu)$ en la lista tab\'u
                \State $K \gets mejor\_no\_tabu$
                \Statex
            \Else \Comment{Selecciono una opci\'on tab\'u}
                \State Actualizamos $movimiento(K,menos\_tabu)$ en la lista tab\'u
                \State $K \gets menos\_tabu$
                \State $sin\_mejorar \gets sin\_mejorar+1$
            \EndIf
        \EndWhile

    \EndIf

    \State \Return{$K^*$}
\end{pseudocodigo}

\bigskip

\subsubsection{Variantes\label{tabu:variantes}}
\par Como se puede observar, esta heur\'istica tiene muchos par\'ametros
    con los cuales se puede experimentar con la finalidad de afinar
    la heur\'istica (por lo general, teniendo informaci\'on sobre las
    instancias de entrada del problema).

\par Entre estos par\'ametros es f\'acil distinguir los criterios
    de terminaci\'on. Experimentando con el limite $MAX\_GLOBAL$ y
    $MAX\_SIN\_MEJORAR$ se pueden conseguir distintos resultados (%
    Secci\'on~\ref{tabu:experimentacion}). Tambi\'en, en sinton\'ia
    con estos atributos se puede incluir al tiempo que se considera
    que un movimiento es tab\'u una vez efectuado. Si bien el mismo
    no aparece escenificado en el pseudoc\'odigo anterior, este
    par\'ametro es de gran importancia.

\par Dejando de lado las variantes seg\'un los valores de los
    par\'ametros ya mencionados, planteamos algunas variantes (y
    sus combinaciones) que modifican un poco el esquema de la
    b\'usqueda tab\'u:

\begin{itemize}
    \item \textbf{Sin funci\'on de aspiracion: } Cuando se encuentra
        un movimiento tab\'u que mejora la soluci\'on \'optima
        parcial, esta se registra y se guarda, pero no se utiliza
        como posible vecino\footnote{Excepto que no haya vecinos no tab\'u
        y esta sea la opci\'on menos tab\'u.} esta opci\'on.

    \item \textbf{Entrada Golosa: } En lugar de utilizar un nodo de grado
        al menos promedio, se alimenta a la b\'usqueda tab\'u con el resultado
        de la heur\'istica golosa\footnote{Secci\'on~\ref{golosa},
        \emph{\nameref{golosa}}.}.

    \item \textbf{Los movimientos tab\'u que mejoren la soluci\'on parcial
        reestablecen el contandor de iteraciones sin mejorar: } Esta variante
        es bastante descriptiva por si misma. La idea en si mismo trata
        de considerar a los movimientos tab\'u como posibles buenos movimientos
        previo de descubrir si sirven para mejorar a la soluci\'on \'optima.
\end{itemize}
