\documentclass[11pt, a4paper, spanish]{article}

%Margenes de la pagina.  otra opcion, usar \usepackage{a4wide}
\usepackage[paper=a4paper, left=1.5cm, right=1.5cm, bottom=1.5cm, top=3.5cm]{geometry}

%Cambios para mejorar posicionamiento de floats
\renewcommand{\textfraction}{0.05}
\renewcommand{\topfraction}{0.8}
\renewcommand{\bottomfraction}{0.8}
\renewcommand{\floatpagefraction}{0.75}

%este paquete permite incluir acentos.  Notar que espera un formato ANSI-blah de archivo.  Si en lugar de eso se tiene un utf8 (usual en los linux), entonces usar \usepackage[utf8]{inputenc}
\usepackage[utf8]{inputenc}

%Este paquete es para que algunos titulos (como Tabla de Contenidos) esten en castellano
\usepackage[spanish,es-noshorthands,es-ucroman]{babel}

%El siguiente paquete permite escribir la caratula facilmente
\usepackage{caratula}

%este paquete es innecesario para el TP, aca lo uso para recuadrar
\usepackage{framed}

%Paquetes para opciones comunes
\usepackage[pdftex]{graphicx}
\graphicspath{%
    {img/}%
    {img/caratula/}%
}
\usepackage[usenames,dvipsnames,table]{xcolor}
\usepackage{caption}
\usepackage{subcaption}
\usepackage{enumerate}
\usepackage{float}
\usepackage{amsmath}
\usepackage{comment}

% url – Ver­ba­tim with URL-sen­si­tive line breaks - para el formato de referencias
\usepackage{url}

%Para la Toc
\usepackage{hyperref} % hyperref - Para que la TOC sea interactiva

%configuro hyperref para que haya hiperlinks en la TOC
\hypersetup{%
	colorlinks=true,  	% false deja el texto en negro con los links metidos en cajas en rojo como la mayor parte de la documentacion de los paquetes de latex
	linkcolor=black,	% El default es rojo lo que implica la TOC en roo
	linktoc=all,		% Toda la TOC con hyperlinks
%	hidelinks=true 		% no funciona con mi version del paquete
	urlcolor=blue		% el default es rosa, si se sienten gay friendly, comentamos esta linea
}
\usepackage{nameref}

%Paquetes para hacer tablas especificas
\usepackage{multirow}

%Paquetes para dibujar grafos
\usepackage{tikz}
\usetikzlibrary{arrows,chains,shapes,matrix,trees,positioning,petri,topaths}
\usepackage{tkz-graph} %Especifico de grafos, muy facil, ver doc
\usepackage{tkz-berge} %especifico para grafos conocidos, arboles, caminos, etc
%\tikzstyle{vertex}=[circle,fill=black!25,minimum size=20pt,inner sep=0pt]
%\tikzstyle{selectedVertex} = [vertex, fill=red!50]
%\tikzstyle{edge} = [draw,thick,-]
%\tikzstyle{weight} = [font=\small]
%\tikzstyle{selectedEdge} = [draw,line width=5pt,-,red!50]
%\tikzstyle{ignoredEdge} = [draw,line width=5pt,-,black!20]
\usepackage{epstopdf}

%paquete para pseudocodigo con algorithmicx
\usepackage{algorithm}
\usepackage{algorithmicx} %Es cargado tambien por algpseudocode
\usepackage{algpseudocode}
\usepackage{fixltx2e}
\MakeRobust{\Call} %Para poder hacer llamadas a funciones como par\'ametro de funciones

%configuro el paquete listings para C++
\usepackage{listings}
\lstdefinestyle{c++}{%
    belowcaptionskip=1\baselineskip,
    breaklines=true,
    frame=L,
    numbers=left,
    %numbersep=5pt,
    %xleftmargin=\parindent,
    xleftmargin=3em,
    language=C++,
    showstringspaces=false,
    basicstyle=\footnotesize\ttfamily,
    keywordstyle=\bfseries\color{green!40!black},
    commentstyle=\itshape\color{purple!40!black},
    identifierstyle=\color{blue},
    stringstyle=\color{orange},
}
\lstset{style=c++}

%Paquetes para encabezado y pie de página
\usepackage{fancyhdr}

\pagestyle{plain}
%\newcommand{\real}{\ensuremath{\mathbb{R}}}

% Encabezado y Pie de pagina
\pagestyle{fancy}
\fancyhf{}
\lfoot{\rightmark}
\lhead{Algoritmos y Estucturas de Datos III}
\rhead{Trabajo Pr\'actico 3}
\usepackage{lastpage}
\rfoot{\thepage\ de \pageref{LastPage}}

\begin{document}
    %Declaro layers para usar con tkiz
    %\pgfdeclarelayer{background}
    %\pgfsetlayers{background,main}

    %Para tener cosas en castellano en los pseudos
    \makeatletter\renewcommand{\ALG@name}{Algoritmo}
    \algrenewcommand\algorithmicwhile{\textbf{mientras}}
    \algrenewcommand\algorithmicdo{\textbf{hacer}}
    \algnewcommand\algorithmicto{\textbf{hasta}}
    \algrenewcommand\algorithmicend{\textbf{fin}}
    \algrenewcommand\algorithmicfor{\textbf{para}}
    \algrenewcommand\algorithmicforall{\textbf{para todo}}
    \algrenewcommand\algorithmicif{\textbf{si}}
    \algrenewcommand\algorithmicthen{\textbf{entonces}}
    \algrenewcommand\algorithmicelse{\textbf{en otro caso}}
    \algrenewcommand\algorithmicrequire{\textbf{Entrada: }}
    \algrenewcommand\algorithmicensure{\textbf{Salida: }}
    \algrenewcommand\algorithmicreturn{\textbf{devuelvo}}
    \algrenewcommand\algorithmicfunction{\textbf{funci\'on}}
    \algrenewtext{For}[3]{\algorithmicfor\ #1 \gets #2 \algorithmicto\ #3 \algorithmicdo}
    \algrenewtext{EndFor}{\algorithmicend}
    \algrenewtext{EndIf}{\algorithmicend}
    \algrenewtext{EndWhile}{\algorithmicend}
    \algblockdefx{ForEach}{EndForEach}[1]{\algorithmicfor\ \textbf{cada}\ #1 \algorithmicdo}{\algorithmicend}
    \algloopdefx{NoEndIf}[1]{\algorithmicif\ #1 \algorithmicthen}
    \algloopdefx{NoEndForEach}[1]{\algorithmicfor\ \textbf{cada}\ #1 \algorithmicdo}

    %esto construye la caratula y el \'indice
    \materia{Algoritmos y Estructuras de Datos III}
\submateria{2do cuatrimestre 2013}
\titulo{Trabajo Pr\'actico 3}
%\subtitulo{}
%\fecha{4 de Octubre}
%\integrante{Aisemberg, Dan Ezequiel}{242/12}{dea4493@hotmail.com}
%\integrante{\'Alvarez, Mat\'ias Exequiel}{090/12}{matyy.alvarez@gmail.com}
\integrante{Kantor, Sacha Sebasti\'an}{009/11}{sachakantor@gmail.com}
%\integrante{Litwak, Brian Mois\'es}{241/12}{brian.litwak@gmail.com}
\integrante{Quintana, Jorge Luis}{344/11}{jorge.quintana.81@gmail.com}

%\maketitle

    \maketitle

    \thispagestyle{empty}
    \tableofcontents
    \pagebreak

    \phantomsection
    \addcontentsline{toc}{section}{Introducci\'on y aclaraciones}
    \section*{Introducci\'on y Aclaraciones}
        \phantomsection
\addcontentsline{toc}{subsection}{Sobre las referencias de este documento}
\subsection*{Sobre las referencias de este documento}
\par A lo largo de este informe existen referencias tanto a \emph{papers}
    encontrados durante la elaboraci\'on de las resoluciones como as\'i
    referencias a cap\'itulos espec\'ificos de libros que son parte de la
    bibliograf\'ia de la materia.

\par Los mismos han sido puestos a disposici\'on del corrector en un repositorio
    de archivos \emph{online}, cuya direcci\'on \emph{web} est\'a inclu\'ida
    en los archivos provistos con este documento.

\phantomsection
\addcontentsline{toc}{subsection}{Sobre la compilaci\'on de los archivos fuente}
\subsection*{Sobre la compilaci\'on de los archivos fuente}
\par En los archivos provistos junto con este trabajo se pueden distinguir dos
    directorios donde hay archivos fuentes compilables: \emph{src} y
    \emph{experimentaci\'on}. En ambos hay ya un archivo \emph{Makefile} con el
    cual se pueden compilar todos los binarios desarrollados y utilizados en
    el marco de este trabajo.

\par Cabe destacar, que en la implementaci\'on provista, se permite compilar
    dichos binarios con un dise\~no \emph{multi-thread}\footnote{\url{%
    http://www.openmp.org}}, donde cada \emph{hilo} de ejecuci\'on tomar\'a
    una instancia distinta del problema provista por el \emph{standard input}
    y la resolver\'a con el algoritmo correspondiente en paralelo. Esto nos
    permite ejecutar muchas instancias en simult\'aneo, pero sin afectar
    el tiempo medido (ya que cada \emph{thread} es resuelta en un procesador
    diferente). En caso de querer compilar de esta manera los binarios, se
    debe ejecutar el siguiente comando desde el directorio \emph{src} o
    \emph{experimentaci\'on}:

\bigskip
\par make EXTRACXXFLAGS=-fopenmp
\bigskip

\par Para poder compilar el binario de esta manera, se requiere tener la 4.4
    o superior del compilador \emph{GNU Compiler Collection}\footnote{\url{%
    http://gcc.gnu.org/}}.

\phantomsection
\addcontentsline{toc}{subsection}{Sobre los generadores de instancias aleatorias}
\subsection*{Sobre los generadores de instancias aleatorias}
\par Una vez que se implementaron los algoritmos que resuelven los problemas
    presentados por el enunciado de este trabajo, se llega a la etapa de
    an\'alisis/experimentaci\'on donde se quieren verificar los resultados
    te\'oricos obtenidos con informaci\'on emp\'irica. Para ello, en cada caso/%
    algoritmo, se hizo un an\'alisis sobre alg\'un subconjunto/familia de grafos
    de inter\'es (explicado en cada caso particular durante este informe). Luego,
    se implement\'o un generador de instancias aleatorias para estas familias
    (el cual es compilado seg\'un como se indica en el punto anterior) el cual,
    mediante una interfaz de men\'u por l\'inea de comando, nos permite seleccionar
    la familia, el tama\~no de las instancias a generar, cuantas instancias por
    tama\~no generar y el archivo donde se desea guardar toda esta informaci\'on%
    \footnote{Puede ocurrir que para ciertas familias se requiera alg\'un otro dato,
    el cual ser\'a demandado al usuario por la pantalla.}.

\par El concepto de aleatoriedad en estos generadores ocurre a la hora de elegir
    los extremos de las aristas (siempre preservando la estructura de la familia
    elegida). Para poder hacer esta selecci\'on aleatoria, se utilizaron funciones
    del a \emph{STL} de \emph{C} portadas a \emph{C++}\footnote{%
    \url{http://www.cplusplus.com/reference/cstdlib/rand/}} y algunas de la
    \emph{STL} nativa de \emph{C++}\footnote{%
    \url{http://www.cplusplus.com/reference/algorithm/random_shuffle/}}.

\par Hay una familia de grafos particular que merece una menci\'on aparte dada
    la manera en la que implementamos su generador. Esta es la familia de los
    grafos \emph{planares}\footnote{Claude Berge. Planar Graphs. \emph{The Theory
    of Graphs}, 21:207-2013, 1966.}. La generaci\'on de esta familia convella
    una mayor dificultad, m\'as a\'un, es un problema en si mismo que ha sido
    y es estudiado a\'un\footnote{\'Eric Fusy. Uniform Random Sampling Of Planar
    Graphs in Linear Time.}. Por este motivo, se decidi\'o utilizar una
    biblioteca de funciones de \emph{C++} que ya nos provee dicho generador
    de grafos aleatorios\footnote{\url{http://www.ogdf.net/}}. Sus archivos
    fuentes son provistos junto con este trabajo como as\'i tambi\'en su
    compilaci\'on fue configurada en los \emph{Makefile}'s ya mencionados en
    esta secci\'on.\footnote{La versi\'on de \emph{python} 2.7 es requerida para
    que esto funcione sin ninguna modificaci\'on.}.

\phantomsection
\addcontentsline{toc}{subsection}{Sobre la experimentaci\'on}
\subsection*{Sobre la experimentaci\'on}
\par Llegados a esta instancia del ejercicio, lo que nos importa es poder
    comprobar emp\'iricamente las conclusiones te\'oricas, particularmente
    las relacionadas con la complejidad temporal y "cuan bueno" son los
    algoritmos/heur\'isticas.

\par Entonces nos encontramos en la situaci\'on de tratar de "medir"
    la complejidad del algoritmo propuesto e implementado. Solo que aqu\'i ya no podremos
    comprobar que el resultado devuelto por el algoritmo es correcto por dos
    motivos en especial: llegaremos a un punto donde probaremos con instancias
    muy grandes como para saber a ciencia cierta cual debe ser el resultado y
    pues decidimos darle un grado importante (sino principal) de aletoriedad
    a la selecci\'on/generaci\'on de estas intancias.

\par Para medir la complejidad se decidi\'o usar el \emph{clock} del sistema,
    el mismo es un n\'umero global relacionado al hardware/sistema operativo
    que se incrementa con cada instrucci\'on ejecutada por el ordenador. La idea
    es tomar este n\'umero \footnote{%
    \url{http://www.cplusplus.com/reference/chrono/high_resolution_clock/now/}}
    justo antes e inmediatamente despu\'es de que corra nuestro algoritmo
    (dejando de lado -siguiendo las directivas del enunciado de este TP-
    todas aquellas funcionalidades encargadas de cosas ajenas, como
    por ejemplo el \emph{parseo}\footnote{%
    \url{http://es.wikipedia.org/wiki/Analizador_sint\%C3\%A1ctico}} de los datos
    de entrada/salida). Luego, alcanza con calcular la diferencia entre estos dos
    valores para saber cu\'anto tiempo insumi\'o el programa para arribar a una soluci\'on.

\par El problema de esta metodolog\'ia para medir la complejidad es que no
    toma en cuenta que junto con nuestro algoritmo se encuentra ejecut\'ando
    muchas otras aplicaciones (\emph{Winning Eleven, Mortal Kombat, etc}, las
    cuales "comparten" el tiemp de CPU. Es decir, el sistema operativo podr\'ia
    (m\'as a\'un, lo hace) desalojar nuestra tarea del procesador y pasar a
    ejecutar otras cosas. Luego ser\'a nuevamente el turno de nuestra
    aplicaci\'on, la que podr\'a terminar. Pero mientras todo esto ocurr\'ia,
    el contador global de "clock" del sistema sigui\'o increment\'andose. Es
    decir, no llegamos a una medida exacta de lo que tar\'o nuestro programa.

\par Por estos motivos decidimos, para cada instancia del problema, ejecutarla
    m\'as de una vez (5 veces, en particular) y luego quedarnos con aquella
    ejecuci\'on que requiri\'o menos tiempo en terminar. As\'i
    conseguimos eliminar \emph{Outliers} y poder decir que tenemos una medida
    lo suficientemente precisa.

        \pagebreak

% ------------ Ejercicios ----------
    \section{Situaciones de la Vida Real}
        Situaciones de la vida en las que puede aplicarse el problema
de \emph{Clique de Frontera M\'axima}.

\subsubsection{Una empresa nacional y popular}

Tenemos un conjunto de plataformas en alta mar en los
que hay abundancia de distintos recursos que las grandes
empresas ans\'ian explotar. Estas plataformas estan conectadas 
entre sí por pasarelas. 

Lamentablemente para las empresas con capacidad de explotaci\'on, 
las plataformas en las que se permite explotar un tipo de recurso
en el que la empresa se especializa, rara vez comparten una pasarela
directa, sin embargo la l\'ogistica de su negocio requiere que 
transporten distintos tipos de recursos entre los nodos que
quieren adjudicarse.

Nuestra empresa particularmente se especializa solamente
en el chantaje y la extorsi\'on, recursos que a priori
ninguna de las plataformas permite explotar. Sin embargo
conociendo un poco de la historia reciente de nuestro pa\'is
se nos ocurre la idea de plantearle a la autoridad regulatoria
de la explotaci\'on, hacernos cargo del mantenimiento de las 
pasarelas bajo la contra-prestaci\'on de cobrar un peaje
a las otras empresas que utilizan la v\'ia.

Se nos autoriza a hacer este negocio, sin embargo como la 
autoridad regulatoria no es tan inocente/corrupta como los
sucesivos gobiernos argentinos (desde por lo menos 1810), 
nos exige que para poder efectuar nuestra maniobra seamos 
los concesionarios de al menos uno de los extremos de cada
una de estas pasarelas y hacernos cargo del mantenimiento
de dicha plataforma tambi\'en.

Este requerimiento, nos exige una inversi\'on mayor a la
que nos gustar\'ia efectuar, en principio el mantenimiento
de una plataforma es mucho mas oneroso que el de una pasarela,
ademas de tener que pagar el canon de la misma y estar limitados
a no poder cobrar peaje por pasar por la plataforma.

En nuestra desesperaci\'on se nos ocurre un plan viable, 
sub-alquilaremos las plataformas que tengamos que mantener
a empresas que deseen explotarlas (nosotros no tenemos el 
m\'as m\'inimo inter\'es en producir nada). 

Sin embargo dentro del mundillo empresario, tenemos muy 
mala reputaci\'on y los interesados potenciales en explotar
las plataformas s\'olo est\'an dispuestos a alquilarnos a
nosotros a precio reducido. 

Luego de hacer puntillosamente las predicciones econ\'omicas
vemos que aun podemos sacar una ventaja global pero en cada
una de las plataformas estaremos perdiendo dinero.

Ahi nos viene a la mente el trabajo pr\'actico de 
aquel infame segundo cuatrimestre de 2013 presentado por 
la c\'atedra de Algoritmos y Estructuras de Datos III 
en la facultad de Ciencias Exactas - U.B.A. \ldots
si s\'olo lo hubieramos hecho entonces \ldots


\subsubsection{Futuro posible o futuro ideal?}
Estamos en el a\~no $15000$ d.c y somos capitanes de una flota
de piratas espaciales. Tenemos un conocimiento muy limitado
de la geograf\'ia espacial, b\'asicamente estamos circunscriptos
a nuestro coto de saqu\'eos habitual y \'ultimamente
empezamos a sospechar que hemos sobre-explotado el mismo con
lo que se dificulta una proyecci\'on ec\'onomica sustentable
(adem\'as de ver seriamente disminuidas nuestras chances de
obtener placer por medio de las actividades de amenaza, coerci\'on
y en definitiva tortura que tanto disfrutamos), a pesar de esto
con los a\~nos hemos acumulado una interesante capacidad 
armament\'istica y somos expertos en el campo de batalla.

En uno de nuestros saqu\'eos descubrimos un mapa muy limitado
de otra regi\'on espacial inexplorada por nosotros pero cuyo 
amplio detalle cartogr\'afico indica habitabilidad y una
gran prosperidad de sus habitantes, con lo que se nos hace 
agua la boca.

Nos encontramos en un punto de quiebre, hasta el momento
suponiamos que el universo se circunscrib\'ia a nuestro 
universo conocido, pero en este momento avizoramos un 
crisol de oportunidades para la continuidad y sustentabilidad
de nuestras ilimitadas perversiones.

Como no somos especialistas en la navegaci\'on interestelar, nos
procuramos la colaboraci\'on de un experto en topolog\'ia espacial
(lease, secuestramos un cerebrito) para que nos explique en 
detalle el mapa y los conceptos subyacentes tales como los 
agujeros de gusano. Este accede a explicarnos, luego de marat\'onicas
sesiones de negociaci\'on (tortura unilateral), sobre la organizaci\'on
de la red de planetas descripta en el mapa.

B\'asicamente, la forma de llegar a ese c\'umulo de planetas es
la navegaci\'on con nuestras naves a traves de esos agujeros de
gusano hacia alguna de las zonas m\'as densamente populadas. Nos
explica tambi\'en que se detallan numerosos agujeros de gusano
sin destino conocido y nos revela su sospecha de que conducen 
a otros cuadrantes espaciales menos densos pero no por ello
menos apetitosos para nuestras actividades. 

El experto en topolog\'ia nos presenta un modelo donde los planetas 
habitados son asimilados a nodos y las rutas espaciales que los conectan
son asimiladas a aristas en un grafo no dirigido. Con este modelo
armado nos recomienda hacer un estudio de donde nos conviene 
desplegar nuestro ataque inicialmente, evaluando nuestro poder de
fuego en relaci\'on con el beneficio que podemos obtener de 
dominar todas las rutas desconocidas accesibles desde esos c\'umulos
de sociedades.

Si bien nos sobra el poder de fuego para dominar por completo 
uno de estos c\'umulos, somos realistas y sabemos que potencialmente
tendremos bajas, tenemos calculado intuitivamente que a menor
cantidad de batalla menor cantidad de bajas por lo tanto buscamos
un equilibrio donde tengamos la mayor cantidad rutas con el menor
despliegue posible

El experto en topolog\'ia espacial ya exhausto m\'as all\'a del punto
de no retorno, con su \'ultimo suspiro nos introduce el concepto de 
clique de frontera m\'axima como soluci\'on \'optima a nuestras ansias.

Lamentablemente, luego de soportar estoicamente varios meses de tortura
nuestro experto nos abandona para conocer a su creador y nos deja con el 
problema planteado pero sin haber sido resuelto.

Es ese el instante preciso en el que lamentamos no haber aprobado el 
tercer trabajo pr\'actido de Algoritmos y Estructuras de Datos III 
por haber estado demasiado ocupados fantaseando con el futuro hace
aproximadamente 13000 a\~nos. Sin embargo nuestra torva faz se contrae
en una media sonrisa al darnos cuenta de la iron\'ia que conlleva que 
a pesar de ello, en la actualidad monopolizamos el poder.

        \pagebreak

    \section{Algoritmo Exacto}
        \subsection{Resoluci\'on}
    \begin{pseudocodigo}[Algoritmo Exacto para \emph{CMF} - Descriptivo]
    \Require\Statex
        \begin{itemize}
            \item Un grafo $G$ de $n$ v\'ertices\footnote{Asumimos, sin p\'erdida 
                de generalidad, que los v\'ertices est\'an numerados de 1 a $n$.}
                y $m$ aristas.

            \item Una funci\'on $candidatos(K)$, que dado un conjunto de v\'ertices
                $K$, devuelve una secuencia de v\'ertices de $G$ ordenada por grado
                de mayor a menor que son adjacentes a todos los elementos de $K$ y
                cuyo grado es mayor $2|K|$.

            \item Una funci\'on $\delta(K)$, que dada una \emph{clique} $K$\footnote{Una
                \emph{clique} no es otra cosa que un conjunto de v\'ertices, con la
                caracter\'istica de inducir un grafo completo en $G$.} en $G$, calcula
                el cardinal de su frontera\footnote{Como ya se explic\'o, sea $K$ una
                \emph{clique} de $G$, entonces: $\delta(K) = - |K|(|K|-1) +
                \displaystyle\sum_{v \in K} d(v)$}.

            \item Una funci\'on $\delta_{cota}(K)$, que dada una \emph{clique} $K$
                en $G$, calcula la cota de su frontera seg\'un el m\'etodo
                ya explicado anteriormente.

        \end{itemize}
    \Statex
    \Ensure Una \emph{clique} $K$ de $G$ con una frontera $\delta(K)$ de m\'axima
        cardinalidad.

    \Statex

    \If{$m = \dfrac{n(n-1)}{2}$} \Comment{Si $G$ es un $K_n$}
        \State $K \gets \left\{1;\dots;\left\lfloor\dfrac{n}{2}\right\rfloor\right\}$

    \Else
        \State $\delta_{max} \gets \left\lfloor\dfrac{r+1}{2}\right\rfloor\cdot
            \left\lceil\dfrac{r+1}{2}\right\rceil$ para $r$ tal que:
            $\begin{pmatrix}
                \text{\Huge$\bigwedge$} &
                    \begin{matrix}
                        r \in [1..n)\\[0.5cm]
                        r < \dfrac{n^2}{n^2 - 2m}\\[0.5cm]
                        (\forall r' \in [1..n))$ $r' < \dfrac{n^2}{n^2 - 2m} \implies r'\leq r
                    \end{matrix}
            \end{pmatrix}$\footnote{Esta cota inferior para la cardinalidad de la
                frontera maximal en $G$ se deriv\'o a partir del \emph{Teorema de Tur\'an},
                como ya se explic\'o anteriormente.}

        \State $K \gets \emptyset$, $K' \gets \emptyset$, $YaProcesados(K')$\footnote{%
            $YaProcesados(K')$ representar\'a a los v\'ertices de $candidatos(K')$ que
            ya fueron tenidos en cuenta para la \emph{clique} $K'$.}$ \gets \emptyset$

        \While{$candidatos(K')$\footnote{Si $K'$ es vac\'io, $candidatos(K')$ son todos
            los nodos de $G$.}$\setminus YaProcesados(K')$\footnote{Peque\~no abuso
            de notaci\'on, ya que $candidatos(K')$ es una secuencia y $YaProcesados(K')$
            es un conjunto. Se asume que es la secuencia sin los elementos del conjunto.}$
            \neq \emptyset$ $\lor$ $K' \neq \emptyset$}

            \If{$candidatos(K') \setminus YaProcesados(K') \neq \emptyset$ $\land$ $\delta_{max} < \delta_{cota}(K')$%
                \footnote{Si $K'$ es vac\'io, $\delta_{cota}(K')$ es $m$.}}

                \State $v \gets cabeza(candidatos(K') \setminus YaProcesados(K'))$
                \State $K' \gets K' \cup \{v'\}$
                \If{$candidatos(K') = \emptyset$ $\land$ $\delta_{max} < \delta(K')$}
                    %\Comment{Llegu\'e al final de la rama y obtuve una mejor solucion}
                    \State $\delta_{max} \gets \delta(K')$
                    \State $K \gets K'$
                \EndIf

            \ElsIf{$K' \neq \emptyset$} \Comment{\emph{Backtracking}}
                \State $v \gets ultimo(K')$\footnote{Donde $ultimo(K')$
                    es el \'ulitmo v\'ertice agregado a $K'$.}
                \State $K' \gets K' \setminus \{v'\}$
                \State $YaProcesados(K') \gets YaProcesados(K') \cup \{v\}$

            \Else
                \State $YaProcesados(K') \gets YaProcesados(K') \cup candidatos(K')$
            \EndIf
        \EndWhile
    \EndIf

    \State \Return{$K$}
\end{pseudocodigo}


\subsection{Complejidad}
    \subsubsection{Estructuras de Datos}
\par Como ya se mencion\'o en~\nameref{grafo:estructuras}, ya contamos en el grafo
    de entrada con listas/vectores de adyacencias y una matriz de adyacencias.

\par Al igual que en la heur\'istica golosa y el algoritmo exacto, decidimos
    modelar las cliques con vectores%
    \footnote{\url{http://www.cplusplus.com/reference/vector/vector/}}.

\par A su vez, para mejorar la eficiencia al recorrer la vecindad, se
    utilizan un vector/clique como \emph{min heaps} seg\'un el grado
    de los nodos, permitiendonos acceder en $\mathcal O(1)$ al nodo
    de menor grado. Los motivos de este proceder se detallan en
    la siguiente secci\'on.

\subsubsection{Pseudoc\'odigo de complejidad}
\par Se presenta a continuaci\'on un pseudoc\'odigo m\'as espec\'ifico de la implementaci\'on
    de este algoritmo provista junto con este trabajo. El mismo tiene en cuenta
    las estructuras de datos explicadas en el punto anterior.

\par Luego del pseudoc\'odigo se justifican detalladamente las complejidades
    expuestas a continuaci\'on que no sean evidentes\footnote{Consideramos
    como "complejidades evidentes" las asignaciones de variables, operaciones
    m\'atematicas simples, asignaciones/inicializaci\'on de posiciones de
    un vector/\emph{deque} o cualquier contenedor de acceso aleatorio/arbitrario}.

\par Debemos aclarar, debido a las variantes ya presentadas sobre esta
    heur\'istica, que el siguiente an\'alisis de complejidad aplica
    sobre la heur\'istica que selecciona al primer vecino v\'alido encontrado,
    en una vecindad donde s\'olo se consideran las cliques con un nodo m\'as
    o un nodo menos y que selecciona como clique inicial a alg\'un nodo
    de grado mayor o igual al grado promedio del grafo.

\par El an\'alisis de complejidad de las variantes se realiza en la secci\'on
    \ref{bl:compl:variantes}.

\bigskip

\begin{pseudocodigo}[Heur\'istica de B\'usqueda Local para \emph{CMF} - Complejidad]
    \Require Un grafo $G$ con $n$ v\'ertices numerados de $1$ a $n$ y $m$ aristas. El mismo
        cuenta con las siguientes estructuras de datos que lo modelan:
        \begin{itemize}
            \item Vectores de adyacencia: Dado un vertice $v$, $vecinos(v)$ nos da todos los
                nodos adyacentes a $v$ en $G$.

            \item Matriz de adyacencia: Dados los v\'ertices $v$ y $w$, $adyacentes(v,w)$ y
                $adyacentes(w,v)$ nos devuelven $true$ si y s\'olo si $v$ es adyacente
                a $w$ en $G$.

            \item Vector de nodos de $G$.
        \end{itemize}
    \Ensure\Statex
        \begin{itemize}
            \item Un vector $K$ correspondiente a la \emph{clique} de m\'axima frontera
                encontrada por la heur\'istica.

            \item El cardinal de $\delta(K)$, siendo $K$ la \emph{clique} del item anterior.
        \end{itemize}
    \Statex
    \State $K \gets \emptyset$ \Compl{Brown}{}{$n$}{}
    \If{$m = \frac{n(n-1)}{2}$} \Compl{Blue}{}{$1$}{}
        \State $K \gets \left\{1;\dots;\left\lfloor\sfrac{n}{2}\right\rfloor\right\}$ \Compl{Blue}{}{$n$}{}
        \State $\delta_{max} \gets \left\lfloor\sfrac{n}{2}\right\rfloor\cdot
            \left\lceil\sfrac{n}{2}\right\rceil$ \Compl{Blue}{}{$1$}{}
        \Statex
    \Else
        \State $K \gets$ Primer nodo de grado mayor o igual al grado promedio \Compl{Blue}{}{$n$}{}
        \State $\delta(K) \gets d(v)$ \Compl{Blue}{}{$1$}{}
        \If{$\exists v \in vecinos(v)$ tal que $d(v) > 2|K|$} \Compl{Red}{}{$n$}{}
            \State $v_{add} \gets v$ \Compl{Red}{}{$1$}{}
            \State $\delta(K+v_{add}) \gets \delta(K) + d(v_{add}) - 2|K|$ \Compl{Red}{}{$1$}{}
        \EndIf \Compl{Blue}{Costo del \emph{si}: }{$n$}{}
        \Statex
        \State $v_{rem} \gets top(K)$ \Compl{Blue}{}{$1$}{}
        \State $\delta(K-v_{rem}) \gets \delta(K)-d(v_{rem})+2(|K|-1)$ \Compl{Blue}{}{$1$}{}
        \Statex
        \While{$\delta(K) < \delta(K+v_{add})$ $\lor$ $\delta(K) < \delta(K-v_{rem})$} \Compl{Red}{}{$1$}{}
            \If{$\delta(K+v_{add}) > \delta(K-v_{rem})$} \Compl{Fuchsia}{}{$1$}{}
                \State $push\_heap(K,v_{add})$ \Compl{Fuchsia}{}{$log(n)$}{}
                \State $\delta(K) \gets \delta(K+v_{add})$ \Compl{Fuchsia}{}{$1$}{}
                \Statex
            \Else
                \State $pop\_heap(K)$ \Compl{Fuchsia}{}{$log(n)$}{}
                \State $\delta(K) \gets \delta(K-v_{rem})$ \Compl{Fuchsia}{}{$1$}{}
                \Statex
            \EndIf \Compl{Red}{Costo del \emph{si}: }{$log(n)$}{}
            \Statex
            \Statex $\triangleright$ Ya salte a un vecino, recorro la nueva vecindad
            \If{$\exists v \in candidatos(K)$ tal que $d(v) > 2|K|$} \Compl{Fuchsia}{}{$n^2$}{}
                \State $v_{add} \gets v$ \Compl{Fuchsia}{}{$1$}{}
                \State $\delta(K+v_{add}) \gets \delta(K) + d(v_{add}) - 2|K|$ \Compl{Fuchsia}{}{$1$}{}
            \EndIf \Compl{Red}{Costo del \emph{si}: }{$n^2$}{}
            \Statex
            \State $v_{rem} \gets top(K)$ \Compl{Red}{}{$1$}{}
            \State $\delta(K-v_{rem}) \gets \delta(K)-d(v_{rem})+2(|K|-1)$ \Compl{Red}{}{$1$}{}
            \Statex
        \EndWhile \Compl{Blue}{Costo del \emph{mientras}: }{$n^2$}{veces $\times$ $\mathcal O(n^2 + log(n)) = \mathcal O(n^4)$ }
    \EndIf \Compl{Brown}{Costo \emph{si}: }{$n^4 + n$}{$= \mathcal O(n^4)$}
    \State \Return{$\delta(K)$, $K$} \Compl{Brown}{}{$1$\label{bl:return}}{}
    \Statex
    \Statex \Compl{Brown}{Costo Total de la Heur\'istica: }{$n^4 +n$}{$= \mathcal O(n^4)$}
\end{pseudocodigo}

\bigskip

\par La primera parte del algoritmo es exactamente igual a la de la
    heur\'istica golosa y el algoritmo exacto: reconoce en $\mathcal O(1)$
    si el grafo $G$ es un $K_n$ y en caso afirmativo, devuelve
    la \emph{CMF} con costo lineal. Esto ya fue justificado en
    los dos casos previos, as\'i pues se omite su justificaci\'on.

\par Luego, en caso de estar ante un $G$ que no es un grafo completo,
    el algoritmo busca a un nodo que tenga grado mayor o igual al
    promedio. Esta b\'usqueda se realiza sobre el vector de nodos
    (los cuales permiten el acceso a su grado en $\mathcal O(1)$),
    con lo cual el costo en el peor caso pasa a ser revisar
    todos los nodos del grafo\footnote{\url{http://www.cplusplus.com/reference/algorithm/find_if/}},
    siendo la complejidad entonces $\mathcal O(n)$. Habiendo
    escogido el nodo/clique inicial, se guarda la frontera
    parcial correspondiente (que no es otra cosa que el grado
    de dicho nodo, que como se dijo, es accedible en tiempo
    constante) y se asigna dicho nodo al vector que representa
    la clique (lo cual tiene un costo constante tambien\footnote{%
    \url{http://www.cplusplus.com/reference/vector/vector/push_back/}}.

\par Luego, antes de comenzar el ciclo, se inicializan los posibles
    valores de la vecindad. En s\'i mismo, se recorrer la vecindad
    \emph{inicial}, guardando las primeras cliques vecinas (una
    con un nodo m\'as y otra con un nodo menos\footnote{Esta \'ultima
    clique es trivial ya que en este momento la clique se compone
    de s\'olo un nodo.}) que incrementen la funci\'on objetivo.

\par Buscar una clique que incremente la frontera es lineal ya
    que basta con recorrer el \emph{deque} de vecinos del
    \'unico nodo de la clique (el cual es accesible en tiempo
    constante gracias a el vector de nodos de la estructura del
    grafo y a la propia estructura de estos nodos\footnote{Ver
    \emph{\nameref{grafo:estructuras}}, en la secci\'on
    \emph{\nameref{notas_preliminares}}.} y realizar una
    \'unica comparacion de tiempo constante (verificar que
    efectivamente su grado incrementa la frontera).

\par Encontrado este nodo, calcular cuanto incrementa la funci\'on
    es trivial y se realiza en tiempo constante, ya que alcanza
    con hacer las operaciones matem\'aticas sobre la frontera
    de la clique actual y el grado del nodo que se a\~nadir\'ia%
    \footnote{Ver la Secci\'on \emph{\nameref{notas:calc_front}},
    en \emph{\nameref{notas_preliminares}}.}.

\par Luego, para encontrar un nodo que incremente la frontera
    si fuese eliminado de la clique se hace aprovechando
    las estructuras de un \emph{min heap} sobre los grados
    de la clique, utilizando la representaci\'on de este
    que no es otra que un vector \footnote{\url{%
    http://www.cplusplus.com/reference/algorithm/make_heap/}}.
    Si bien el heap en este caso no fue inicializado mediante
    la funci\'on \emph{make\_heap}, esto no es necesario ya
    que en esta variante, s\'olo se tiene un nodo en la clique.
    Luego, si mantenemos esta estructura en el vector, las operaciones
    $push\_heap$ y $pop_heap$ tendr\'an un coste logar\'itmico
    mientras que $top$ sera de coste constante\footnote{%
    \url{http://www.cplusplus.com/reference/algorithm/push_heap/},%
    \url{http://www.cplusplus.com/reference/algorithm/pop_heap/},%
    \url{http://www.cplusplus.com/reference/vector/vector/front/},%
    \url{http://www.cplusplus.com/reference/vector/vector/pop_back/},%
    \url{http://www.cplusplus.com/reference/vector/vector/push_back/}}.

\par Teniendo ya un \emph{min heap}, podemos acceder al nodo
    de menor grado de la clique actual en $\mathcal O(1)$

\par Nuevamente, calcular la frontera si se quitase dicho
    nodo es trivial. Pero ac\'a debemos justificar el hecho
    de no buscar alg\'un otro nodo que al ser quitado incremente
    la frontera en caso de que este nodo (el tope del heap) no
    lo haga. El fundamento aqu\'i es que el tope del heap
    es el nodo de menor grado, por lo tanto, es el nodo
    que menos frontera le a\~nade a $K$, por lo tanto,
    quitar cualquier otro nodo implicar\'ia quitar un nodo
    de mayor grado, lo que significa que estar\'iamos
    considerando quitar un nodo que aporta m\'as frontera
    que el tope del heap y por lo tanto estariamos
    obteniendo una frontera menor que la que se obtiene
    de quitar el tope. Esto nos permite asegurar que si
    se tiene el nodo de menor grado de la clique, y este
    no incrementa la frontera si se quitase de la clique,
    entonces ning\'un otro nodo de la clique lo har\'a.

\par Luego, comienza el ciclo de la b\'usqueda local. En la
    guarda del mismo se realizan s\'olo comparaciones
    de enteros que son almacenados en variables de la implementaci\'on
    de esta funci\'on de \emph{C++}, por lo cual su acceso
    tiene un coste $\mathcal O(1)$.

\par Inmediatamente al comenzar el ciclo hay un condicional
    \emph{si}. El mismo en su guarda realiza una operaci\'on
    $\mathcal O(1)$ por los mismos motivos explicados en
    el p\'arrafo anterior. Dentro de sus dos ramas vemos
    que se realizan 1 asignaci\'on ($\mathcal O(1)$) y
    un $push\_heap$ o $pop\_heap$, que tienen un coste
    logar\'itmico como ya se explico m\'as arriba.

\par En la segunda mitad del ciclo, se
    revisa la nueva vecindad (en este punto, el ciclo
    habr\'a a\~nadido o quitado alg\'un nodo de la
    clique, por lo cual tenemos una nueva vecindad). En
    el caso de los vecinos que se componen de cliques con
    un nodo menos, el proceso es id\'entico al realizado
    antes de entrar al ciclo, con lo cual no hace falta
    justificar nada m\'as. En cuanto a las cliques que
    contienen un nodo m\'as, ahora tenemos una clique
    con (posiblemente) m\'as de un nodo, por lo cual,
    debemos ver todos sus candidatos, pero no nos podemos
    basar en los nodos de la clique antes de realizar
    el salto al vecino pues si se quito un nodo, los
    candidatos podr\'ian incrementarse. Por lo tanto,
    debemos recorrer (en el peor de los casos) todos
    los nodos adyacentes a alg\'un nodo de la clique
    (que son accesibles en $\mathcal O(1)$ gracias al vector
    de nodos de la estructura de $G$) y verificar 2
    cosas: que sean adyacentes a todos los dem\'as
    nodos de la clique y que incrementen la frontera%
    \footnote{Es decir, que $d(v) > 2|K|$.}. En caso
    de encontrar uno, podemos detener este recorrido de posibles
    nodos candidatos, ya que no estamos buscando saltar
    al mejor vecino sino al primero que encontremos
    que cumpla con nuestros requerimientos. Pero a\'un as\'i,
    en el peor de los casos deberemos recorrerlos a todos,
    es decir, $\mathcal O(n)$ nodos adyacentes, y tambi\'en
    como cota para $|K|$ debemos considerar que podr\'ia
    estar en el orden de $\mathcal O(n)$ elementos, siendo
    el costo de esta b\'usqueda de vecino candidato
    con un nodo m\'as, igual a $\mathcal O(n^2)$.

\par Por \'ultimo, ya teniendo calculado el coste del ciclo,
    debemos acotar la cantidad de iteraciones que tendr\'a.
    Esto es d\'ificil, ya que sin mayor informaci\'on sobre
    el grafo de entrada y su estructura, no se puede saber
    a ciencia cierta cuantas veces podr\'a "deplazarse" el
    algoritmo en la vecindad definida. Por lo tanto, se
    decidi\'o acotar mediante lo m\'inimo que crece la frontera
    entre iteraci\'on e iteraci\'on. Este m\'inimo es, simplemente,
    1. Considerando este peor caso, donde la funci\'on objetivo
    crece muy lentamente, y considerando que la frontera m\'axima
    no puede ser mayor a $m$ (ya que no hay m\'as aristas que $m$
    en $G$), podemos asegurar que la cantidad de iteraciones
    que har\'a el ciclo estar\'a acotada por $m$, y $m$ est\'a
    acotado por $n^2$.

\par Por lo tanto, sabiendo ya que el costo por ciclo es
    $\mathcal O(n^2)$, y sabieno que no tendremos m\'as
    de $n^2$ iteraciones, conclu\'imos que el costo del
    ciclo estar\'a acotado por $\mathcal O(n^4)$.

\par Al final de la heur\'istica, se devuelven la frontera
    ($\mathcal O(1)$ por ser un tipo primitivo del lenguaje)
    y la clique ($\mathcal O(1)$ por ser devuelta por
    referencia). Entonces, el coste final de todo el algoritmo
    heur\'istico recae sobre el ciclo y la inicializaci\'on
    mediante \emph{reserve}\footnote{\url{%
    http://www.cplusplus.com/reference/vector/vector/reserve/}} de
    $K$. Finalmente, el coste es $\mathcal O(n + n^4)$, que
    asint\'oticamente hablando es $\mathcal O(n^4)$.

\subsubsection{Complejidad de las Variantes\label{bl:compl:variantes}}
\par 

\bigskip

\begin{pseudocodigo}[Heur\'istica de B\'usqueda Local para \emph{CMF} con Intercambio - Complejidad]
    \Require Un grafo $G$ con $n$ v\'ertices numerados de $1$ a $n$ y $m$ aristas. El mismo
        cuenta con las siguientes estructuras de datos que lo modelan:
        \begin{itemize}
            \item Vectores de adyacencia: Dado un vertice $v$, $vecinos(v)$ nos da todos los
                nodos adyacentes a $v$ en $G$.

            \item Matriz de adyacencia: Dados los v\'ertices $v$ y $w$, $adyacentes(v,w)$ y
                $adyacentes(w,v)$ nos devuelven $true$ si y s\'olo si $v$ es adyacente
                a $w$ en $G$.

            \item Vector de nodos de $G$.
        \end{itemize}
    \Ensure\Statex
        \begin{itemize}
            \item Un vector $K$ correspondiente a la \emph{clique} de m\'axima frontera
                encontrada por la heur\'istica.

            \item El cardinal de $\delta(K)$, siendo $K$ la \emph{clique} del item anterior.
        \end{itemize}
    \Statex
    \State $K \gets \emptyset$ \Compl{Brown}{}{$n$}{}
    \If{$m = \frac{n(n-1)}{2}$} \Compl{Blue}{}{$1$}{}
        \State $K \gets \left\{1;\dots;\left\lfloor\sfrac{n}{2}\right\rfloor\right\}$ \Compl{Blue}{}{$n$}{}
        \State $\delta_{max} \gets \left\lfloor\sfrac{n}{2}\right\rfloor\cdot
            \left\lceil\sfrac{n}{2}\right\rceil$ \Compl{Blue}{}{$1$}{}
        \Statex
    \Else
        \State $K \gets$ Primer nodo de grado mayor o igual al grado promedio \Compl{Blue}{}{$n$}{}
        \State $\delta(K) \gets d(v)$ \Compl{Blue}{}{$1$}{}
        \If{$\exists v \in vecinos(v)$ tal que $d(v) > 2|K|$} \Compl{Red}{}{$n$}{}
            \State $v_{add} \gets v$ \Compl{Red}{}{$1$}{}
            \State $\delta(K+v_{add}) \gets \delta(K) + d(v_{add}) - 2|K|$ \Compl{Red}{}{$1$}{}
        \EndIf \Compl{Blue}{Costo del \emph{si}: }{$n$}{}
        \Statex
        \State $v_{rem} \gets top(K)$ \Compl{Blue}{}{$1$}{}
        \State $\delta(K-v_{rem}) \gets \delta(K)-d(v_{rem})+2(|K|-1)$ \Compl{Blue}{}{$1$}{}
        \Statex
        \State Busco $v_{exchIN}$ y $v_{exchOUT}$ tales que $\delta(K) < \delta(K-v_{exchOUT}+v_{exchIN})$ \Compl{Blue}{}{$n^3$}{}
        \State $\delta(K-v_{exchOUT}+v_{exchIN}) \gets \delta(K) - d(v_{exchOUT}) + d(v_{exchIN})$ \Compl{Blue}{}{$1$}{}
        \Statex
        \While{%
        $\begin{pmatrix}
            \text{\Huge{$\bigvee$}} &
            \begin{matrix}
                \delta(K) < \delta(K+v_{add})\\
                \delta(K) < \delta(K-v_{rem})\\
                \delta(K) < \delta(K-v_{exchOUT}+v_{exchIN})
            \end{matrix}
        \end{pmatrix}$%
        } \Compl{Red}{}{$1$}{}
            \Statex
            \If{$\delta(K+v_{add}) > \delta(K-v_{rem})$ $\land$ $\delta(K+v_{add}) > \delta(K-v_{exchOUT}+v_{exchIN})$} \Compl{Fuchsia}{}{$1$}{}
                \State $push\_heap(K,v_{add})$ \Compl{Fuchsia}{}{$log(n)$}{}
                \State $\delta(K) \gets \delta(K+v_{add})$ \Compl{Fuchsia}{}{$1$}{}
                \Statex
            \ElsIf{$\delta(K-v_{rem}) > \delta(K-v_{exchOUT}+v_{exchIN})$} \Compl{Fuchsia}{}{$1$}{}
                \State $pop\_heap(K)$ \Compl{Fuchsia}{}{$log(n)$}{}
                \State $\delta(K) \gets \delta(K-v_{rem})$ \Compl{Fuchsia}{}{$1$}{}
                \Statex
            \Else
                \State Intercambio en $K$ los nodos $v_{exchOUT}$ y $v_{exchIN}$ \Compl{Fuchsia}{}{$1$}{}
                \State Mantengo la estructura del \emph{heap} \Compl{Fuchsia}{}{$log(n)$}
                \State $\delta(K) \gets \delta(K) - d(v_{exchOUT}) + d(v_{exchIN})$ \Compl{Fuchsia}{}{$1$}{}
            \EndIf \Compl{Red}{Costo del \emph{si}: }{$log(n)$}{}
            \Statex
            \Statex $\triangleright$ Ya salte a un vecino, recorro la nueva vecindad
            \If{$\exists v \in candidatos(K)$ tal que $d(v) > 2|K|$} \Compl{Fuchsia}{}{$n^2$}{}
                \State $v_{add} \gets v$ \Compl{Fuchsia}{}{$1$}{}
                \State $\delta(K+v_{add}) \gets \delta(K) + d(v_{add}) - 2|K|$ \Compl{Fuchsia}{}{$1$}{}
            \EndIf \Compl{Red}{Costo del \emph{si}: }{$n^2$}{}
            \Statex
            \State $v_{rem} \gets top(K)$ \Compl{Red}{}{$1$}{}
            \State $\delta(K-v_{rem}) \gets \delta(K)-d(v_{rem})+2(|K|-1)$ \Compl{Red}{}{$1$}{}
            \Statex
            \State Busco $v_{exchIN}$ y $v_{exchOUT}$ tales que $\delta(K) < \delta(K-v_{exchOUT}+v_{exchIN})$ \Compl{Red}{}{$n^3$}{}
            \State $\delta(K-v_{exchOUT}+v_{exchIN}) \gets \delta(K) - d(v_{exchOUT}) + d(v_{exchIN})$ \Compl{Red}{}{$1$}{}
            \Statex
        \EndWhile \Compl{Blue}{Costo del \emph{mientras}: }{$n^2$}{veces $\times$ $\mathcal O(n^3 + n^2 + log(n)) = \mathcal O(n^5)$ }
    \EndIf \Compl{Brown}{Costo \emph{si}: }{$n^5 + n$}{$= \mathcal O(n^5)$}
    \State \Return{$\delta(K)$, $K$} \Compl{Brown}{}{$1$\label{bl:return}}{}
    \Statex
    \Statex \Compl{Brown}{Costo Total de la Heur\'istica: }{$n^5 + n$}{$= \mathcal O(n^5)$}
\end{pseudocodigo}

\bigskip


\subsection{Testeo y Casos Particulares de Estudio}
    Empezaremos mencionando y caracterizando algunas familias de grafos
para las que nuestra heurist\'ica golosa constructiva siempre encuentra
un resultado correcto, esto es, una clique de m\'axima frontera.

Teniendo en cuenta, como ya se menciono anteriormente, que la heur\'istica
empieza por uno de los nodos de mayor grado y en cada paso va 
agregando nodos que forman clique con los nodos actualmente escogidos
bajo la condici\'on de que el grado de los mismos sea mayor a dos veces
el tama\~no de la clique actual (de estos candidatos, tambi\'en elige
uno de los de mayor grado). Tenemos lo siguiente:

En todos los casos en los que hay varios candidatos para agregar a la
clique con el mismo grado (en particular el primero) la heur\'istica
puede proporcionar el resultado incorrecto, dado que el algoritmo
elige determin\'isticamente uno de los candidatos mientras que la 
entrada es aleatoria (la idea es poder resolver el problema para
cualquier grafo de entrada), veremos a continuaci\'on que excepto
en casos muy particulares, grafos isomorfos cuyas matrices de adyacencia
son distintas pueden, al ser sometidos a la heur\'istica, arrojar
resultados muy dispares (en particular, resultados correctos, versus
resultados incorrectos).

\subsubsection{Grafos completos}
	En estos grafos como ya se mencion\'o anteriormente, el algoritmo
	escoge nodos y va haciendo crecer una clique hasta llegar a una 
	clique de tama\~no $n/2$, notes\'e que todos los nodos en los 
	grafos de esta familia tienen grado $d(i) = n - 1$, por lo
	tanto la selecci\'on en cada paso del siguiente nodo de la
	clique es dependiente de la implementaci\'on

	Tambi\'en anal\'iticamente puede verse que las CMF para esta 
	familia de grafos son tanto las $K_{\left \lfloor{n/2}\right \rfloor}$ 
	como las $K_{\left \lceil{n/2}\right \rceil}$. Habiendo establecido como
	cota superior al tama\~no de la CMF el valor $n/2$, la posibilidad de que
	los $K_{\left \lceil{n/2}\right \rceil}$ tambi\'en cumplan parece una 
	anomal\'ia a priori. Veremos que no lo es y que est\'a directamente
	relacionado con la naturaleza de la familia en estudio. 

	La frontera es la cantidad de aristas que sale de cada uno de los
	nodos de la clique hacia nodos que no est\'an en la misma. Siendo
	$n$ la cantidad de nodos del grafo y habiendo establecido que la CMF
	tiene tama~no $\left \lfloor{n/2}\right \rfloor$, 
	si la clique en estudio tiene este tama\~no, los nodos que quedan afuera
	de la clique son $\left \lceil{n/2} \right \rceil$, luego: 

	\( 
	\delta(K_{\left \lfloor{n/2}\right \rfloor}) = 
	\left \lfloor{n/2} \right \rfloor \times 
	(n - \left \lfloor{n/2} \right \rfloor ) = 
	\left \lfloor{n/2} \right \rfloor \times 
	(\left \lfloor{n/2} \right \rfloor +
	\left \lceil{n/2} \right \rceil - 
	\left \lfloor{n/2} \right \rfloor) =
	\left \lfloor{n/2} \right \rfloor \times
	\left \lceil{n/2} \right \rceil = 
	\delta(K_{\left \lceil{n/2}\right \rceil})
	\)
\begin{figure}[H]
\caption{Ejemplos grafos completos - Entrada / Salida}
\centering
%\includegraphics[scale = 0.5]{img/ej2/k2.png}
%\includegraphics[scale = 0.5]{img/ej2/k3.png}
\includegraphics[scale = 0.5]{img/ej2/k4.png}
\includegraphics[scale = 0.5]{img/ej2/k5.png}
\includegraphics[scale = 0.5]{img/ej2/k7.png}
\end{figure}

\subsection{\'Arboles}
La familia de los \'arboles es una de las familias en las que la heur\'istica
se puede romper y no devolver el resultado correcto. 

Para los \'arboles es pr\'acticamente inmediato que el 
tama\~no de la clique de m\'axima frontera es a lo sumo igual a dos, dado
que por su misma definici\'on los arboles no contienen circuitos simples.

Nuevamente en este caso se toma alguno de los nodos de mayor grado 
y se intenta hacer crecer la clique con alg\'un nodo adyacente.

Como puede verse en los siguientes ejemplos en el caso de que 
implementativamente se seleccione uno de los nodos de mayor grado
adyacente a otro de los nodos de mayor grado, la heur\'istica encuentra
efectivamente una CMF, pero en el caso de que el nodo de partida sea 
uno de los de mayor grado en el grafo y dicho nodo no pertenezca a
ninguna de las CMF del grafo, la heur\'istica reportar\'a un falso 
positivo (una clique cuya frontera es maximal pero no m\'axima).

Desde un punto de vista implementativo el desempate entre nodos de 
mismo grado, depender\'a de la matriz de adyacencia (la posici\'on en
la que aparece un nodo) y dado que existe un arbol isomorfo al del 
ejemplo en el que el nodo incorrecto para empezar la clique est\'a
listado antes que el nodo correcto (el que permite generar la CMF),
el siguiente es un buen ejemplo para ilustrar la fragilidad de la
heur\'istica

\begin{figure}[H]
\caption{Ejemplos \'Arboles}
\centering
\includegraphics[scale = 0.5]{img/ej2/tree_st0.png}
\includegraphics[scale = 0.5]{img/ej2/tree_st01.png}
\includegraphics[scale = 0.5]{img/ej2/tree_st02.png}
\includegraphics[scale = 0.5]{img/ej2/tree_st11.png}
\includegraphics[scale = 0.5]{img/ej2/tree_st12.png}
\end{figure}

\subsection{Grafos bipartitos}

\subsection{Grafos circulares}

\subsection{Estrellas}

\subsection{Grafo de secuencias}

\subsection{Banana tree}
	




La familia de gr\'afos que asegura un resultado incorrecto de nuestra 
heur\'istica golosa puede ser caracterizada como aquella en la que el
nodo de mayor grado no forma parte de la clique de m\'axima frontera.
Esto sucede ya que la heur\'istica construye una clique a partir del 
nodo de grado m\'aximo y en cada paso se mantiene el invariante de 
de clique, agregando solamente nodos cuyo grado sea mayor al tama\~no
de la clique en ese paso, por lo tanto si alguno de los nodos de la 
clique de m\'axima frontera no forma clique con el nodo 
inicial (el de mayor grado) entonces la heuristica constructiva no 
puede llegar a incluir a ese nodo.

Ejemplos:

%\begin{tikzpicture}
%	\SetGraphUnit{1.5cm}
%	\GraphInit[vstyle=Normal]
%
%	\Vertices{circle}{2,5,6,7,8}
%	\WE(2){1}
%	\Vertices{circle}{3,10,9,11}
%	\WE(3){2}
%
%	\foreach \v in {2,5,6,7,8}{\Edge(1)(\v)};
%
%\end{tikzpicture}
	
\begin{tikzpicture}
	\path 	
			% Nodos
			(0,8) node [shape=circle, draw] (1) {1}
			(1,8) node [shape=circle, draw] (2) {2}
			(2,8) node [shape=circle, draw] (3) {3}
			(6,8) node [shape=circle, draw] (4) {4}
			(7,8) node [shape=circle, draw] (5) {5}
			(8,8) node [shape=circle, draw] (6) {6}
			(0,7) node [shape=circle, draw] (7) {7}
			(1,7) node [shape=circle, draw] (8) {8}
			(7,7) node [shape=circle, draw] (9) {9}
			(8,7) node [shape=circle, draw] (10) {10}
			(0,6) node [shape=circle, draw] (11) {11}
			(1,6) node [shape=circle, draw] (12) {12}
			(7,6) node [shape=circle, draw] (13) {13}
			(8,6) node [shape=circle, draw] (14) {14}
			(4,5) node [shape=circle, draw] (15) {15}
			(2,4) node [shape=circle, draw] (16) {16}
			(3,4) node [shape=circle, draw] (17) {17}
			(5,4) node [shape=circle, draw] (18) {18}
			(6,4) node [shape=circle, draw] (19) {19}
			(2,3) node [shape=circle, draw] (20) {20}
			(3,3) node [shape=circle, draw] (21) {21}
			(4,3) node [shape=circle, draw] (22) {22}
			(5,3) node [shape=circle, draw] (23) {23}
			(6,3) node [shape=circle, draw] (24) {24}
			(2,2) node [shape=circle, draw] (25) {25}
			(6,2) node [shape=circle, draw] (26) {26}
			(4, 1.5) node [shape=circle, draw] (27) {27}
			(2,1) node [shape=circle, draw] (28) {28}
			(6,1) node [shape=circle, draw] (29) {29}
			(2,0) node [shape=circle, draw] (30) {30}
			(3,0) node [shape=circle, draw] (31) {31}
			(5,0) node [shape=circle, draw] (32) {32}
			(6,0) node [shape=circle, draw] (33) {33};
			% Aristas de la semiestrella superior izq
			\draw[-] (1) -- (8);
			\draw[-] (2) -- (8);
			\draw[-] (3) -- (8);
			\draw[-] (7) -- (8);
			\draw[-] (11) -- (8);
			\draw[-] (12) -- (8);

			% Aristas de la semiestrella superior izq
			\draw[-] (4) -- (9);
			\draw[-] (5) -- (9);
			\draw[-] (6) -- (9);
			\draw[-] (10) -- (9);
			\draw[-] (13) -- (9);
			\draw[-] (14) -- (9);

			% Mas aristas
			
			\draw[-] (8) -- (9);
			\draw[-] (8) -- (15);
			\draw[-] (9) -- (15);
			\draw[-] (16) -- (15);
			\draw[-] (17) -- (15);
			\draw[-] (18) -- (15);
			\draw[-] (19) -- (15);
			\draw[-] (22) -- (15);

			% Aristas de la estrella inferior

			\draw[-] (20) -- (27);
			\draw[-] (21) -- (27);
			\draw[-] (22) -- (27);
			\draw[-] (23) -- (27);
			\draw[-] (24) -- (27);
			\draw[-] (25) -- (27);
			\draw[-] (26) -- (27);
			\draw[-] (28) -- (27);
			\draw[-] (29) -- (27);
			\draw[-] (30) -- (27);
			\draw[-] (31) -- (27);
			\draw[-] (32) -- (27);
			\draw[-] (33) -- (27);

			
\end{tikzpicture}


\subsection{Experimentaci\'on}
    \subsubsection{Resultados}

%Doble estrella
\begin{figure}[H]
    \centering
    \fontsize{8}{10}\selectfont
    \resizebox{0.87\textwidth}{!}{\input{img/ej3/tabu_search/ej3_nodos_nlogn_star+bridge+double_star.tex}}
    \caption{Complejidad temporal para grafos Estrella+Puente+Doble Estrella (Variante Max\_Iter=nlog(n))}
\end{figure}

\begin{figure}[H]
    \centering
    \fontsize{8}{10}\selectfont
    \resizebox{0.87\textwidth}{!}{\input{img/ej3/tabu_search/ej3_nodos_n_star+bridge+double_star.tex}}
    \caption{Complejidad temporal para grafos Estrella+Puente+Doble Estrella (Variante Max\_Iter=n)}
\end{figure}

\begin{figure}[H]
    \centering
    \fontsize{8}{10}\selectfont
    \resizebox{0.87\textwidth}{!}{\input{img/ej3/tabu_search/ej3_frontera_star_bridge_double_star.tex}}
    \caption{Frontera de grafos Estrella+Puente+Doble Estrella}
\end{figure}

%Doble estrella sin aspiracion
\begin{figure}[H]
    \centering
    \fontsize{8}{10}\selectfont
    \resizebox{0.87\textwidth}{!}{\input{img/ej3/tabu_search/ej3_nodos_nlogn_star+bridge+double_star_sin_aspiracion.tex}}
    \caption{Complejidad temporal para grafos Estrella+Puente+Doble Estrella (Variante Max\_Iter=nlog(n),sin aspiraci\'on)}
\end{figure}

\begin{figure}[H]
    \centering
    \fontsize{8}{10}\selectfont
    \resizebox{0.87\textwidth}{!}{\input{img/ej3/tabu_search/ej3_nodos_n_star+bridge+double_star_sin_aspiracion.tex}}
    \caption{Complejidad temporal para grafos Estrella+Puente+Doble Estrella (Variante Max\_Iter=n),sin aspiraci\'on)}
\end{figure}

\begin{figure}[H]
    \centering
    \fontsize{8}{10}\selectfont
    \resizebox{0.87\textwidth}{!}{\input{img/ej3/tabu_search/ej3_frontera_star_bridge_double_star_sin_aspiracion.tex}}
    \caption{Frontera de grafos Estrella+Puente+Doble Estrella (sin aspiracion)}
\end{figure}

%Doble estrella sin aspiracion golosa
\begin{figure}[H]
    \centering
    \fontsize{8}{10}\selectfont
    \resizebox{0.87\textwidth}{!}{\input{img/ej3/tabu_search/ej3_nodos_nlogn_star+bridge+double_star_sin_aspiracion_golosa.tex}}
    \caption{Complejidad temporal para grafos Estrella+Puente+Doble Estrella (Variante Max\_Iter=nlog(n),sin aspiraci\'on,golosa)}
\end{figure}

\begin{figure}[H]
    \centering
    \fontsize{8}{10}\selectfont
    \resizebox{0.87\textwidth}{!}{\input{img/ej3/tabu_search/ej3_nodos_n_star+bridge+double_star_sin_aspiracion_golosa.tex}}
    \caption{Complejidad temporal para grafos Estrella+Puente+Doble Estrella (Variante Max\_Iter=n),sin aspiraci\'on,golosa)}
\end{figure}

\begin{figure}[H]
    \centering
    \fontsize{8}{10}\selectfont
    \resizebox{0.87\textwidth}{!}{\input{img/ej3/tabu_search/ej3_frontera_star_bridge_double_star_sin_aspiracion_golosa.tex}}
    \caption{Frontera de grafos Estrella+Puente+Doble Estrella (sin aspiracion, golosa)}
\end{figure}

%Bipartito Completo
\begin{figure}[H]
    \centering
    \fontsize{8}{10}\selectfont
    \resizebox{0.87\textwidth}{!}{\input{img/ej3/tabu_search/ej3_nodos_nlogn_complete_bipartite.tex}}
    \caption{Complejidad temporal para grafos Bipartito Completo (Variante Max\_Iter=nlog(n))}
\end{figure}

\begin{figure}[H]
    \centering
    \fontsize{8}{10}\selectfont
    \resizebox{0.87\textwidth}{!}{\input{img/ej3/tabu_search/ej3_nodos_n_complete_bipartite.tex}}
    \caption{Complejidad temporal para grafos Bipartito Completo (Variante Max\_Iter=n)}
\end{figure}

\begin{figure}[H]
    \centering
    \fontsize{8}{10}\selectfont
    \resizebox{0.87\textwidth}{!}{\input{img/ej3/tabu_search/ej3_frontera_complete_bipartite.tex}}
    \caption{Frontera de grafos Bipartito Completo}
\end{figure}

%Bipartito Completo sin aspiracion
\begin{figure}[H]
    \centering
    \fontsize{8}{10}\selectfont
    \resizebox{0.87\textwidth}{!}{\input{img/ej3/tabu_search/ej3_nodos_nlogn_complete_bipartite_sin_aspiracion.tex}}
    \caption{Complejidad temporal para grafos Bipartito Completo (Variante Max\_Iter=nlog(n),sin aspiraci\'on)}
\end{figure}

\begin{figure}[H]
    \centering
    \fontsize{8}{10}\selectfont
    \resizebox{0.87\textwidth}{!}{\input{img/ej3/tabu_search/ej3_nodos_n_complete_bipartite_sin_aspiracion.tex}}
    \caption{Complejidad temporal para grafos Bipartito Completo (Variante Max\_Iter=n),sin aspiraci\'on)}
\end{figure}

\begin{figure}[H]
    \centering
    \fontsize{8}{10}\selectfont
    \resizebox{0.87\textwidth}{!}{\input{img/ej3/tabu_search/ej3_frontera_complete_bipartite_sin_aspiracion.tex}}
    \caption{Frontera de grafos Bipartito Completo (sin aspiracion)}
\end{figure}

%Bipartito Completo sin aspiracion golosa
\begin{figure}[H]
    \centering
    \fontsize{8}{10}\selectfont
    \resizebox{0.87\textwidth}{!}{\input{img/ej3/tabu_search/ej3_nodos_nlogn_complete_bipartite_sin_aspiracion_golosa.tex}}
    \caption{Complejidad temporal para grafos Bipartito Completo (Variante Max\_Iter=nlog(n),sin aspiraci\'on,golosa)}
\end{figure}

\begin{figure}[H]
    \centering
    \fontsize{8}{10}\selectfont
    \resizebox{0.87\textwidth}{!}{\input{img/ej3/tabu_search/ej3_nodos_n_complete_bipartite_sin_aspiracion_golosa.tex}}
    \caption{Complejidad temporal para grafos Bipartito Completo (Variante Max\_Iter=n),sin aspiraci\'on,golosa)}
\end{figure}

\begin{figure}[H]
    \centering
    \fontsize{8}{10}\selectfont
    \resizebox{0.87\textwidth}{!}{% GNUPLOT: LaTeX picture with Postscript
\begingroup
  \makeatletter
  \providecommand\color[2][]{%
    \GenericError{(gnuplot) \space\space\space\@spaces}{%
      Package color not loaded in conjunction with
      terminal option `colourtext'%
    }{See the gnuplot documentation for explanation.%
    }{Either use 'blacktext' in gnuplot or load the package
      color.sty in LaTeX.}%
    \renewcommand\color[2][]{}%
  }%
  \providecommand\includegraphics[2][]{%
    \GenericError{(gnuplot) \space\space\space\@spaces}{%
      Package graphicx or graphics not loaded%
    }{See the gnuplot documentation for explanation.%
    }{The gnuplot epslatex terminal needs graphicx.sty or graphics.sty.}%
    \renewcommand\includegraphics[2][]{}%
  }%
  \providecommand\rotatebox[2]{#2}%
  \@ifundefined{ifGPcolor}{%
    \newif\ifGPcolor
    \GPcolortrue
  }{}%
  \@ifundefined{ifGPblacktext}{%
    \newif\ifGPblacktext
    \GPblacktexttrue
  }{}%
  % define a \g@addto@macro without @ in the name:
  \let\gplgaddtomacro\g@addto@macro
  % define empty templates for all commands taking text:
  \gdef\gplbacktext{}%
  \gdef\gplfronttext{}%
  \makeatother
  \ifGPblacktext
    % no textcolor at all
    \def\colorrgb#1{}%
    \def\colorgray#1{}%
  \else
    % gray or color?
    \ifGPcolor
      \def\colorrgb#1{\color[rgb]{#1}}%
      \def\colorgray#1{\color[gray]{#1}}%
      \expandafter\def\csname LTw\endcsname{\color{white}}%
      \expandafter\def\csname LTb\endcsname{\color{black}}%
      \expandafter\def\csname LTa\endcsname{\color{black}}%
      \expandafter\def\csname LT0\endcsname{\color[rgb]{1,0,0}}%
      \expandafter\def\csname LT1\endcsname{\color[rgb]{0,1,0}}%
      \expandafter\def\csname LT2\endcsname{\color[rgb]{0,0,1}}%
      \expandafter\def\csname LT3\endcsname{\color[rgb]{1,0,1}}%
      \expandafter\def\csname LT4\endcsname{\color[rgb]{0,1,1}}%
      \expandafter\def\csname LT5\endcsname{\color[rgb]{1,1,0}}%
      \expandafter\def\csname LT6\endcsname{\color[rgb]{0,0,0}}%
      \expandafter\def\csname LT7\endcsname{\color[rgb]{1,0.3,0}}%
      \expandafter\def\csname LT8\endcsname{\color[rgb]{0.5,0.5,0.5}}%
    \else
      % gray
      \def\colorrgb#1{\color{black}}%
      \def\colorgray#1{\color[gray]{#1}}%
      \expandafter\def\csname LTw\endcsname{\color{white}}%
      \expandafter\def\csname LTb\endcsname{\color{black}}%
      \expandafter\def\csname LTa\endcsname{\color{black}}%
      \expandafter\def\csname LT0\endcsname{\color{black}}%
      \expandafter\def\csname LT1\endcsname{\color{black}}%
      \expandafter\def\csname LT2\endcsname{\color{black}}%
      \expandafter\def\csname LT3\endcsname{\color{black}}%
      \expandafter\def\csname LT4\endcsname{\color{black}}%
      \expandafter\def\csname LT5\endcsname{\color{black}}%
      \expandafter\def\csname LT6\endcsname{\color{black}}%
      \expandafter\def\csname LT7\endcsname{\color{black}}%
      \expandafter\def\csname LT8\endcsname{\color{black}}%
    \fi
  \fi
  \setlength{\unitlength}{0.0500bp}%
  \begin{picture}(7200.00,5040.00)%
    \gplgaddtomacro\gplbacktext{%
      \csname LTb\endcsname%
      \put(1298,2904){\makebox(0,0)[r]{\strut{} 0}}%
      \csname LTb\endcsname%
      \put(1298,3052){\makebox(0,0)[r]{\strut{} 500}}%
      \csname LTb\endcsname%
      \put(1298,3199){\makebox(0,0)[r]{\strut{} 1000}}%
      \csname LTb\endcsname%
      \put(1298,3347){\makebox(0,0)[r]{\strut{} 1500}}%
      \csname LTb\endcsname%
      \put(1298,3494){\makebox(0,0)[r]{\strut{} 2000}}%
      \csname LTb\endcsname%
      \put(1298,3642){\makebox(0,0)[r]{\strut{} 2500}}%
      \csname LTb\endcsname%
      \put(1298,3789){\makebox(0,0)[r]{\strut{} 3000}}%
      \csname LTb\endcsname%
      \put(1298,3937){\makebox(0,0)[r]{\strut{} 3500}}%
      \csname LTb\endcsname%
      \put(1298,4084){\makebox(0,0)[r]{\strut{} 4000}}%
      \csname LTb\endcsname%
      \put(1298,4232){\makebox(0,0)[r]{\strut{} 4500}}%
      \csname LTb\endcsname%
      \put(1298,4379){\makebox(0,0)[r]{\strut{} 5000}}%
      \csname LTb\endcsname%
      \put(1430,2684){\makebox(0,0){\strut{} 0}}%
      \csname LTb\endcsname%
      \put(1967,2684){\makebox(0,0){\strut{} 500}}%
      \csname LTb\endcsname%
      \put(2505,2684){\makebox(0,0){\strut{} 1000}}%
      \csname LTb\endcsname%
      \put(3042,2684){\makebox(0,0){\strut{} 1500}}%
      \csname LTb\endcsname%
      \put(3579,2684){\makebox(0,0){\strut{} 2000}}%
      \csname LTb\endcsname%
      \put(4117,2684){\makebox(0,0){\strut{} 2500}}%
      \csname LTb\endcsname%
      \put(4654,2684){\makebox(0,0){\strut{} 3000}}%
      \csname LTb\endcsname%
      \put(5191,2684){\makebox(0,0){\strut{} 3500}}%
      \csname LTb\endcsname%
      \put(5728,2684){\makebox(0,0){\strut{} 4000}}%
      \csname LTb\endcsname%
      \put(6266,2684){\makebox(0,0){\strut{} 4500}}%
      \csname LTb\endcsname%
      \put(6803,2684){\makebox(0,0){\strut{} 5000}}%
      \put(176,3641){\rotatebox{-270}{\makebox(0,0){\strut{}Frontera}}}%
      \put(396,3641){\rotatebox{-270}{\makebox(0,0){\strut{}(Escala Lineal)}}}%
      \put(4116,2354){\makebox(0,0){\strut{}Cantidad de Nodos}}%
      \put(4116,2134){\makebox(0,0){\strut{}(Escala Lineal)}}%
      \put(4116,4709){\makebox(0,0){\strut{}Frontera obtenida segun cantidad de nodos}}%
    }%
    \gplgaddtomacro\gplfronttext{%
      \csname LTb\endcsname%
      \put(6461,1713){\makebox(0,0)[r]{\strut{}Iter=nlog(n),Sin Mejorar=n,Tiempo Tabu=n}}%
      \csname LTb\endcsname%
      \put(6461,1493){\makebox(0,0)[r]{\strut{}Iter=nlog(n),Sin Mejorar=n,Tiempo Tabu=n2}}%
      \csname LTb\endcsname%
      \put(6461,1273){\makebox(0,0)[r]{\strut{}Iter=nlog(n),Sin Mejorar=n2,Tiempo Tabu=n}}%
      \csname LTb\endcsname%
      \put(6461,1053){\makebox(0,0)[r]{\strut{}Iter=nlog(n),Sin Mejorar=n2,Tiempo Tabu=n2}}%
      \csname LTb\endcsname%
      \put(6461,833){\makebox(0,0)[r]{\strut{}Iter=n,Sin Mejorar=n,Tiempo Tabu=n}}%
      \csname LTb\endcsname%
      \put(6461,613){\makebox(0,0)[r]{\strut{}Iter=n,Sin Mejorar=n,Tiempo Tabu=n2}}%
      \csname LTb\endcsname%
      \put(6461,393){\makebox(0,0)[r]{\strut{}Iter=n,Sin Mejorar=n2,Tiempo Tabu=n}}%
      \csname LTb\endcsname%
      \put(6461,173){\makebox(0,0)[r]{\strut{}Iter=n,Sin Mejorar=n2,Tiempo Tabu=n2}}%
    }%
    \gplbacktext
    \put(0,0){\includegraphics{ej3_frontera_complete_bipartite_sin_aspiracion_golosa}}%
    \gplfronttext
  \end{picture}%
\endgroup
}
    \caption{Frontera de grafos Bipartito Completo (sin aspiracion,golosa)}
\end{figure}

\subsubsection{Conclusiones}


        \pagebreak

    \section{Heur\'istica Constructiva Golosa}
        \subsection{Resoluci\'on}
    \begin{pseudocodigo}[Algoritmo Exacto para \emph{CMF} - Descriptivo]
    \Require\Statex
        \begin{itemize}
            \item Un grafo $G$ de $n$ v\'ertices\footnote{Asumimos, sin p\'erdida 
                de generalidad, que los v\'ertices est\'an numerados de 1 a $n$.}
                y $m$ aristas.

            \item Una funci\'on $candidatos(K)$, que dado un conjunto de v\'ertices
                $K$, devuelve una secuencia de v\'ertices de $G$ ordenada por grado
                de mayor a menor que son adjacentes a todos los elementos de $K$ y
                cuyo grado es mayor $2|K|$.

            \item Una funci\'on $\delta(K)$, que dada una \emph{clique} $K$\footnote{Una
                \emph{clique} no es otra cosa que un conjunto de v\'ertices, con la
                caracter\'istica de inducir un grafo completo en $G$.} en $G$, calcula
                el cardinal de su frontera\footnote{Como ya se explic\'o, sea $K$ una
                \emph{clique} de $G$, entonces: $\delta(K) = - |K|(|K|-1) +
                \displaystyle\sum_{v \in K} d(v)$}.

            \item Una funci\'on $\delta_{cota}(K)$, que dada una \emph{clique} $K$
                en $G$, calcula la cota de su frontera seg\'un el m\'etodo
                ya explicado anteriormente.

        \end{itemize}
    \Statex
    \Ensure Una \emph{clique} $K$ de $G$ con una frontera $\delta(K)$ de m\'axima
        cardinalidad.

    \Statex

    \If{$m = \dfrac{n(n-1)}{2}$} \Comment{Si $G$ es un $K_n$}
        \State $K \gets \left\{1;\dots;\left\lfloor\dfrac{n}{2}\right\rfloor\right\}$

    \Else
        \State $\delta_{max} \gets \left\lfloor\dfrac{r+1}{2}\right\rfloor\cdot
            \left\lceil\dfrac{r+1}{2}\right\rceil$ para $r$ tal que:
            $\begin{pmatrix}
                \text{\Huge$\bigwedge$} &
                    \begin{matrix}
                        r \in [1..n)\\[0.5cm]
                        r < \dfrac{n^2}{n^2 - 2m}\\[0.5cm]
                        (\forall r' \in [1..n))$ $r' < \dfrac{n^2}{n^2 - 2m} \implies r'\leq r
                    \end{matrix}
            \end{pmatrix}$\footnote{Esta cota inferior para la cardinalidad de la
                frontera maximal en $G$ se deriv\'o a partir del \emph{Teorema de Tur\'an},
                como ya se explic\'o anteriormente.}

        \State $K \gets \emptyset$, $K' \gets \emptyset$, $YaProcesados(K')$\footnote{%
            $YaProcesados(K')$ representar\'a a los v\'ertices de $candidatos(K')$ que
            ya fueron tenidos en cuenta para la \emph{clique} $K'$.}$ \gets \emptyset$

        \While{$candidatos(K')$\footnote{Si $K'$ es vac\'io, $candidatos(K')$ son todos
            los nodos de $G$.}$\setminus YaProcesados(K')$\footnote{Peque\~no abuso
            de notaci\'on, ya que $candidatos(K')$ es una secuencia y $YaProcesados(K')$
            es un conjunto. Se asume que es la secuencia sin los elementos del conjunto.}$
            \neq \emptyset$ $\lor$ $K' \neq \emptyset$}

            \If{$candidatos(K') \setminus YaProcesados(K') \neq \emptyset$ $\land$ $\delta_{max} < \delta_{cota}(K')$%
                \footnote{Si $K'$ es vac\'io, $\delta_{cota}(K')$ es $m$.}}

                \State $v \gets cabeza(candidatos(K') \setminus YaProcesados(K'))$
                \State $K' \gets K' \cup \{v'\}$
                \If{$candidatos(K') = \emptyset$ $\land$ $\delta_{max} < \delta(K')$}
                    %\Comment{Llegu\'e al final de la rama y obtuve una mejor solucion}
                    \State $\delta_{max} \gets \delta(K')$
                    \State $K \gets K'$
                \EndIf

            \ElsIf{$K' \neq \emptyset$} \Comment{\emph{Backtracking}}
                \State $v \gets ultimo(K')$\footnote{Donde $ultimo(K')$
                    es el \'ulitmo v\'ertice agregado a $K'$.}
                \State $K' \gets K' \setminus \{v'\}$
                \State $YaProcesados(K') \gets YaProcesados(K') \cup \{v\}$

            \Else
                \State $YaProcesados(K') \gets YaProcesados(K') \cup candidatos(K')$
            \EndIf
        \EndWhile
    \EndIf

    \State \Return{$K$}
\end{pseudocodigo}


\subsection{Complejidad}
    \subsubsection{Estructuras de Datos}
\par Como ya se mencion\'o en~\nameref{grafo:estructuras}, ya contamos en el grafo
    de entrada con listas/vectores de adyacencias y una matriz de adyacencias.

\par Al igual que en la heur\'istica golosa y el algoritmo exacto, decidimos
    modelar las cliques con vectores%
    \footnote{\url{http://www.cplusplus.com/reference/vector/vector/}}.

\par A su vez, para mejorar la eficiencia al recorrer la vecindad, se
    utilizan un vector/clique como \emph{min heaps} seg\'un el grado
    de los nodos, permitiendonos acceder en $\mathcal O(1)$ al nodo
    de menor grado. Los motivos de este proceder se detallan en
    la siguiente secci\'on.

\subsubsection{Pseudoc\'odigo de complejidad}
\par Se presenta a continuaci\'on un pseudoc\'odigo m\'as espec\'ifico de la implementaci\'on
    de este algoritmo provista junto con este trabajo. El mismo tiene en cuenta
    las estructuras de datos explicadas en el punto anterior.

\par Luego del pseudoc\'odigo se justifican detalladamente las complejidades
    expuestas a continuaci\'on que no sean evidentes\footnote{Consideramos
    como "complejidades evidentes" las asignaciones de variables, operaciones
    m\'atematicas simples, asignaciones/inicializaci\'on de posiciones de
    un vector/\emph{deque} o cualquier contenedor de acceso aleatorio/arbitrario}.

\par Debemos aclarar, debido a las variantes ya presentadas sobre esta
    heur\'istica, que el siguiente an\'alisis de complejidad aplica
    sobre la heur\'istica que selecciona al primer vecino v\'alido encontrado,
    en una vecindad donde s\'olo se consideran las cliques con un nodo m\'as
    o un nodo menos y que selecciona como clique inicial a alg\'un nodo
    de grado mayor o igual al grado promedio del grafo.

\par El an\'alisis de complejidad de las variantes se realiza en la secci\'on
    \ref{bl:compl:variantes}.

\bigskip

\begin{pseudocodigo}[Heur\'istica de B\'usqueda Local para \emph{CMF} - Complejidad]
    \Require Un grafo $G$ con $n$ v\'ertices numerados de $1$ a $n$ y $m$ aristas. El mismo
        cuenta con las siguientes estructuras de datos que lo modelan:
        \begin{itemize}
            \item Vectores de adyacencia: Dado un vertice $v$, $vecinos(v)$ nos da todos los
                nodos adyacentes a $v$ en $G$.

            \item Matriz de adyacencia: Dados los v\'ertices $v$ y $w$, $adyacentes(v,w)$ y
                $adyacentes(w,v)$ nos devuelven $true$ si y s\'olo si $v$ es adyacente
                a $w$ en $G$.

            \item Vector de nodos de $G$.
        \end{itemize}
    \Ensure\Statex
        \begin{itemize}
            \item Un vector $K$ correspondiente a la \emph{clique} de m\'axima frontera
                encontrada por la heur\'istica.

            \item El cardinal de $\delta(K)$, siendo $K$ la \emph{clique} del item anterior.
        \end{itemize}
    \Statex
    \State $K \gets \emptyset$ \Compl{Brown}{}{$n$}{}
    \If{$m = \frac{n(n-1)}{2}$} \Compl{Blue}{}{$1$}{}
        \State $K \gets \left\{1;\dots;\left\lfloor\sfrac{n}{2}\right\rfloor\right\}$ \Compl{Blue}{}{$n$}{}
        \State $\delta_{max} \gets \left\lfloor\sfrac{n}{2}\right\rfloor\cdot
            \left\lceil\sfrac{n}{2}\right\rceil$ \Compl{Blue}{}{$1$}{}
        \Statex
    \Else
        \State $K \gets$ Primer nodo de grado mayor o igual al grado promedio \Compl{Blue}{}{$n$}{}
        \State $\delta(K) \gets d(v)$ \Compl{Blue}{}{$1$}{}
        \If{$\exists v \in vecinos(v)$ tal que $d(v) > 2|K|$} \Compl{Red}{}{$n$}{}
            \State $v_{add} \gets v$ \Compl{Red}{}{$1$}{}
            \State $\delta(K+v_{add}) \gets \delta(K) + d(v_{add}) - 2|K|$ \Compl{Red}{}{$1$}{}
        \EndIf \Compl{Blue}{Costo del \emph{si}: }{$n$}{}
        \Statex
        \State $v_{rem} \gets top(K)$ \Compl{Blue}{}{$1$}{}
        \State $\delta(K-v_{rem}) \gets \delta(K)-d(v_{rem})+2(|K|-1)$ \Compl{Blue}{}{$1$}{}
        \Statex
        \While{$\delta(K) < \delta(K+v_{add})$ $\lor$ $\delta(K) < \delta(K-v_{rem})$} \Compl{Red}{}{$1$}{}
            \If{$\delta(K+v_{add}) > \delta(K-v_{rem})$} \Compl{Fuchsia}{}{$1$}{}
                \State $push\_heap(K,v_{add})$ \Compl{Fuchsia}{}{$log(n)$}{}
                \State $\delta(K) \gets \delta(K+v_{add})$ \Compl{Fuchsia}{}{$1$}{}
                \Statex
            \Else
                \State $pop\_heap(K)$ \Compl{Fuchsia}{}{$log(n)$}{}
                \State $\delta(K) \gets \delta(K-v_{rem})$ \Compl{Fuchsia}{}{$1$}{}
                \Statex
            \EndIf \Compl{Red}{Costo del \emph{si}: }{$log(n)$}{}
            \Statex
            \Statex $\triangleright$ Ya salte a un vecino, recorro la nueva vecindad
            \If{$\exists v \in candidatos(K)$ tal que $d(v) > 2|K|$} \Compl{Fuchsia}{}{$n^2$}{}
                \State $v_{add} \gets v$ \Compl{Fuchsia}{}{$1$}{}
                \State $\delta(K+v_{add}) \gets \delta(K) + d(v_{add}) - 2|K|$ \Compl{Fuchsia}{}{$1$}{}
            \EndIf \Compl{Red}{Costo del \emph{si}: }{$n^2$}{}
            \Statex
            \State $v_{rem} \gets top(K)$ \Compl{Red}{}{$1$}{}
            \State $\delta(K-v_{rem}) \gets \delta(K)-d(v_{rem})+2(|K|-1)$ \Compl{Red}{}{$1$}{}
            \Statex
        \EndWhile \Compl{Blue}{Costo del \emph{mientras}: }{$n^2$}{veces $\times$ $\mathcal O(n^2 + log(n)) = \mathcal O(n^4)$ }
    \EndIf \Compl{Brown}{Costo \emph{si}: }{$n^4 + n$}{$= \mathcal O(n^4)$}
    \State \Return{$\delta(K)$, $K$} \Compl{Brown}{}{$1$\label{bl:return}}{}
    \Statex
    \Statex \Compl{Brown}{Costo Total de la Heur\'istica: }{$n^4 +n$}{$= \mathcal O(n^4)$}
\end{pseudocodigo}

\bigskip

\par La primera parte del algoritmo es exactamente igual a la de la
    heur\'istica golosa y el algoritmo exacto: reconoce en $\mathcal O(1)$
    si el grafo $G$ es un $K_n$ y en caso afirmativo, devuelve
    la \emph{CMF} con costo lineal. Esto ya fue justificado en
    los dos casos previos, as\'i pues se omite su justificaci\'on.

\par Luego, en caso de estar ante un $G$ que no es un grafo completo,
    el algoritmo busca a un nodo que tenga grado mayor o igual al
    promedio. Esta b\'usqueda se realiza sobre el vector de nodos
    (los cuales permiten el acceso a su grado en $\mathcal O(1)$),
    con lo cual el costo en el peor caso pasa a ser revisar
    todos los nodos del grafo\footnote{\url{http://www.cplusplus.com/reference/algorithm/find_if/}},
    siendo la complejidad entonces $\mathcal O(n)$. Habiendo
    escogido el nodo/clique inicial, se guarda la frontera
    parcial correspondiente (que no es otra cosa que el grado
    de dicho nodo, que como se dijo, es accedible en tiempo
    constante) y se asigna dicho nodo al vector que representa
    la clique (lo cual tiene un costo constante tambien\footnote{%
    \url{http://www.cplusplus.com/reference/vector/vector/push_back/}}.

\par Luego, antes de comenzar el ciclo, se inicializan los posibles
    valores de la vecindad. En s\'i mismo, se recorrer la vecindad
    \emph{inicial}, guardando las primeras cliques vecinas (una
    con un nodo m\'as y otra con un nodo menos\footnote{Esta \'ultima
    clique es trivial ya que en este momento la clique se compone
    de s\'olo un nodo.}) que incrementen la funci\'on objetivo.

\par Buscar una clique que incremente la frontera es lineal ya
    que basta con recorrer el \emph{deque} de vecinos del
    \'unico nodo de la clique (el cual es accesible en tiempo
    constante gracias a el vector de nodos de la estructura del
    grafo y a la propia estructura de estos nodos\footnote{Ver
    \emph{\nameref{grafo:estructuras}}, en la secci\'on
    \emph{\nameref{notas_preliminares}}.} y realizar una
    \'unica comparacion de tiempo constante (verificar que
    efectivamente su grado incrementa la frontera).

\par Encontrado este nodo, calcular cuanto incrementa la funci\'on
    es trivial y se realiza en tiempo constante, ya que alcanza
    con hacer las operaciones matem\'aticas sobre la frontera
    de la clique actual y el grado del nodo que se a\~nadir\'ia%
    \footnote{Ver la Secci\'on \emph{\nameref{notas:calc_front}},
    en \emph{\nameref{notas_preliminares}}.}.

\par Luego, para encontrar un nodo que incremente la frontera
    si fuese eliminado de la clique se hace aprovechando
    las estructuras de un \emph{min heap} sobre los grados
    de la clique, utilizando la representaci\'on de este
    que no es otra que un vector \footnote{\url{%
    http://www.cplusplus.com/reference/algorithm/make_heap/}}.
    Si bien el heap en este caso no fue inicializado mediante
    la funci\'on \emph{make\_heap}, esto no es necesario ya
    que en esta variante, s\'olo se tiene un nodo en la clique.
    Luego, si mantenemos esta estructura en el vector, las operaciones
    $push\_heap$ y $pop_heap$ tendr\'an un coste logar\'itmico
    mientras que $top$ sera de coste constante\footnote{%
    \url{http://www.cplusplus.com/reference/algorithm/push_heap/},%
    \url{http://www.cplusplus.com/reference/algorithm/pop_heap/},%
    \url{http://www.cplusplus.com/reference/vector/vector/front/},%
    \url{http://www.cplusplus.com/reference/vector/vector/pop_back/},%
    \url{http://www.cplusplus.com/reference/vector/vector/push_back/}}.

\par Teniendo ya un \emph{min heap}, podemos acceder al nodo
    de menor grado de la clique actual en $\mathcal O(1)$

\par Nuevamente, calcular la frontera si se quitase dicho
    nodo es trivial. Pero ac\'a debemos justificar el hecho
    de no buscar alg\'un otro nodo que al ser quitado incremente
    la frontera en caso de que este nodo (el tope del heap) no
    lo haga. El fundamento aqu\'i es que el tope del heap
    es el nodo de menor grado, por lo tanto, es el nodo
    que menos frontera le a\~nade a $K$, por lo tanto,
    quitar cualquier otro nodo implicar\'ia quitar un nodo
    de mayor grado, lo que significa que estar\'iamos
    considerando quitar un nodo que aporta m\'as frontera
    que el tope del heap y por lo tanto estariamos
    obteniendo una frontera menor que la que se obtiene
    de quitar el tope. Esto nos permite asegurar que si
    se tiene el nodo de menor grado de la clique, y este
    no incrementa la frontera si se quitase de la clique,
    entonces ning\'un otro nodo de la clique lo har\'a.

\par Luego, comienza el ciclo de la b\'usqueda local. En la
    guarda del mismo se realizan s\'olo comparaciones
    de enteros que son almacenados en variables de la implementaci\'on
    de esta funci\'on de \emph{C++}, por lo cual su acceso
    tiene un coste $\mathcal O(1)$.

\par Inmediatamente al comenzar el ciclo hay un condicional
    \emph{si}. El mismo en su guarda realiza una operaci\'on
    $\mathcal O(1)$ por los mismos motivos explicados en
    el p\'arrafo anterior. Dentro de sus dos ramas vemos
    que se realizan 1 asignaci\'on ($\mathcal O(1)$) y
    un $push\_heap$ o $pop\_heap$, que tienen un coste
    logar\'itmico como ya se explico m\'as arriba.

\par En la segunda mitad del ciclo, se
    revisa la nueva vecindad (en este punto, el ciclo
    habr\'a a\~nadido o quitado alg\'un nodo de la
    clique, por lo cual tenemos una nueva vecindad). En
    el caso de los vecinos que se componen de cliques con
    un nodo menos, el proceso es id\'entico al realizado
    antes de entrar al ciclo, con lo cual no hace falta
    justificar nada m\'as. En cuanto a las cliques que
    contienen un nodo m\'as, ahora tenemos una clique
    con (posiblemente) m\'as de un nodo, por lo cual,
    debemos ver todos sus candidatos, pero no nos podemos
    basar en los nodos de la clique antes de realizar
    el salto al vecino pues si se quito un nodo, los
    candidatos podr\'ian incrementarse. Por lo tanto,
    debemos recorrer (en el peor de los casos) todos
    los nodos adyacentes a alg\'un nodo de la clique
    (que son accesibles en $\mathcal O(1)$ gracias al vector
    de nodos de la estructura de $G$) y verificar 2
    cosas: que sean adyacentes a todos los dem\'as
    nodos de la clique y que incrementen la frontera%
    \footnote{Es decir, que $d(v) > 2|K|$.}. En caso
    de encontrar uno, podemos detener este recorrido de posibles
    nodos candidatos, ya que no estamos buscando saltar
    al mejor vecino sino al primero que encontremos
    que cumpla con nuestros requerimientos. Pero a\'un as\'i,
    en el peor de los casos deberemos recorrerlos a todos,
    es decir, $\mathcal O(n)$ nodos adyacentes, y tambi\'en
    como cota para $|K|$ debemos considerar que podr\'ia
    estar en el orden de $\mathcal O(n)$ elementos, siendo
    el costo de esta b\'usqueda de vecino candidato
    con un nodo m\'as, igual a $\mathcal O(n^2)$.

\par Por \'ultimo, ya teniendo calculado el coste del ciclo,
    debemos acotar la cantidad de iteraciones que tendr\'a.
    Esto es d\'ificil, ya que sin mayor informaci\'on sobre
    el grafo de entrada y su estructura, no se puede saber
    a ciencia cierta cuantas veces podr\'a "deplazarse" el
    algoritmo en la vecindad definida. Por lo tanto, se
    decidi\'o acotar mediante lo m\'inimo que crece la frontera
    entre iteraci\'on e iteraci\'on. Este m\'inimo es, simplemente,
    1. Considerando este peor caso, donde la funci\'on objetivo
    crece muy lentamente, y considerando que la frontera m\'axima
    no puede ser mayor a $m$ (ya que no hay m\'as aristas que $m$
    en $G$), podemos asegurar que la cantidad de iteraciones
    que har\'a el ciclo estar\'a acotada por $m$, y $m$ est\'a
    acotado por $n^2$.

\par Por lo tanto, sabiendo ya que el costo por ciclo es
    $\mathcal O(n^2)$, y sabieno que no tendremos m\'as
    de $n^2$ iteraciones, conclu\'imos que el costo del
    ciclo estar\'a acotado por $\mathcal O(n^4)$.

\par Al final de la heur\'istica, se devuelven la frontera
    ($\mathcal O(1)$ por ser un tipo primitivo del lenguaje)
    y la clique ($\mathcal O(1)$ por ser devuelta por
    referencia). Entonces, el coste final de todo el algoritmo
    heur\'istico recae sobre el ciclo y la inicializaci\'on
    mediante \emph{reserve}\footnote{\url{%
    http://www.cplusplus.com/reference/vector/vector/reserve/}} de
    $K$. Finalmente, el coste es $\mathcal O(n + n^4)$, que
    asint\'oticamente hablando es $\mathcal O(n^4)$.

\subsubsection{Complejidad de las Variantes\label{bl:compl:variantes}}
\par 

\bigskip

\begin{pseudocodigo}[Heur\'istica de B\'usqueda Local para \emph{CMF} con Intercambio - Complejidad]
    \Require Un grafo $G$ con $n$ v\'ertices numerados de $1$ a $n$ y $m$ aristas. El mismo
        cuenta con las siguientes estructuras de datos que lo modelan:
        \begin{itemize}
            \item Vectores de adyacencia: Dado un vertice $v$, $vecinos(v)$ nos da todos los
                nodos adyacentes a $v$ en $G$.

            \item Matriz de adyacencia: Dados los v\'ertices $v$ y $w$, $adyacentes(v,w)$ y
                $adyacentes(w,v)$ nos devuelven $true$ si y s\'olo si $v$ es adyacente
                a $w$ en $G$.

            \item Vector de nodos de $G$.
        \end{itemize}
    \Ensure\Statex
        \begin{itemize}
            \item Un vector $K$ correspondiente a la \emph{clique} de m\'axima frontera
                encontrada por la heur\'istica.

            \item El cardinal de $\delta(K)$, siendo $K$ la \emph{clique} del item anterior.
        \end{itemize}
    \Statex
    \State $K \gets \emptyset$ \Compl{Brown}{}{$n$}{}
    \If{$m = \frac{n(n-1)}{2}$} \Compl{Blue}{}{$1$}{}
        \State $K \gets \left\{1;\dots;\left\lfloor\sfrac{n}{2}\right\rfloor\right\}$ \Compl{Blue}{}{$n$}{}
        \State $\delta_{max} \gets \left\lfloor\sfrac{n}{2}\right\rfloor\cdot
            \left\lceil\sfrac{n}{2}\right\rceil$ \Compl{Blue}{}{$1$}{}
        \Statex
    \Else
        \State $K \gets$ Primer nodo de grado mayor o igual al grado promedio \Compl{Blue}{}{$n$}{}
        \State $\delta(K) \gets d(v)$ \Compl{Blue}{}{$1$}{}
        \If{$\exists v \in vecinos(v)$ tal que $d(v) > 2|K|$} \Compl{Red}{}{$n$}{}
            \State $v_{add} \gets v$ \Compl{Red}{}{$1$}{}
            \State $\delta(K+v_{add}) \gets \delta(K) + d(v_{add}) - 2|K|$ \Compl{Red}{}{$1$}{}
        \EndIf \Compl{Blue}{Costo del \emph{si}: }{$n$}{}
        \Statex
        \State $v_{rem} \gets top(K)$ \Compl{Blue}{}{$1$}{}
        \State $\delta(K-v_{rem}) \gets \delta(K)-d(v_{rem})+2(|K|-1)$ \Compl{Blue}{}{$1$}{}
        \Statex
        \State Busco $v_{exchIN}$ y $v_{exchOUT}$ tales que $\delta(K) < \delta(K-v_{exchOUT}+v_{exchIN})$ \Compl{Blue}{}{$n^3$}{}
        \State $\delta(K-v_{exchOUT}+v_{exchIN}) \gets \delta(K) - d(v_{exchOUT}) + d(v_{exchIN})$ \Compl{Blue}{}{$1$}{}
        \Statex
        \While{%
        $\begin{pmatrix}
            \text{\Huge{$\bigvee$}} &
            \begin{matrix}
                \delta(K) < \delta(K+v_{add})\\
                \delta(K) < \delta(K-v_{rem})\\
                \delta(K) < \delta(K-v_{exchOUT}+v_{exchIN})
            \end{matrix}
        \end{pmatrix}$%
        } \Compl{Red}{}{$1$}{}
            \Statex
            \If{$\delta(K+v_{add}) > \delta(K-v_{rem})$ $\land$ $\delta(K+v_{add}) > \delta(K-v_{exchOUT}+v_{exchIN})$} \Compl{Fuchsia}{}{$1$}{}
                \State $push\_heap(K,v_{add})$ \Compl{Fuchsia}{}{$log(n)$}{}
                \State $\delta(K) \gets \delta(K+v_{add})$ \Compl{Fuchsia}{}{$1$}{}
                \Statex
            \ElsIf{$\delta(K-v_{rem}) > \delta(K-v_{exchOUT}+v_{exchIN})$} \Compl{Fuchsia}{}{$1$}{}
                \State $pop\_heap(K)$ \Compl{Fuchsia}{}{$log(n)$}{}
                \State $\delta(K) \gets \delta(K-v_{rem})$ \Compl{Fuchsia}{}{$1$}{}
                \Statex
            \Else
                \State Intercambio en $K$ los nodos $v_{exchOUT}$ y $v_{exchIN}$ \Compl{Fuchsia}{}{$1$}{}
                \State Mantengo la estructura del \emph{heap} \Compl{Fuchsia}{}{$log(n)$}
                \State $\delta(K) \gets \delta(K) - d(v_{exchOUT}) + d(v_{exchIN})$ \Compl{Fuchsia}{}{$1$}{}
            \EndIf \Compl{Red}{Costo del \emph{si}: }{$log(n)$}{}
            \Statex
            \Statex $\triangleright$ Ya salte a un vecino, recorro la nueva vecindad
            \If{$\exists v \in candidatos(K)$ tal que $d(v) > 2|K|$} \Compl{Fuchsia}{}{$n^2$}{}
                \State $v_{add} \gets v$ \Compl{Fuchsia}{}{$1$}{}
                \State $\delta(K+v_{add}) \gets \delta(K) + d(v_{add}) - 2|K|$ \Compl{Fuchsia}{}{$1$}{}
            \EndIf \Compl{Red}{Costo del \emph{si}: }{$n^2$}{}
            \Statex
            \State $v_{rem} \gets top(K)$ \Compl{Red}{}{$1$}{}
            \State $\delta(K-v_{rem}) \gets \delta(K)-d(v_{rem})+2(|K|-1)$ \Compl{Red}{}{$1$}{}
            \Statex
            \State Busco $v_{exchIN}$ y $v_{exchOUT}$ tales que $\delta(K) < \delta(K-v_{exchOUT}+v_{exchIN})$ \Compl{Red}{}{$n^3$}{}
            \State $\delta(K-v_{exchOUT}+v_{exchIN}) \gets \delta(K) - d(v_{exchOUT}) + d(v_{exchIN})$ \Compl{Red}{}{$1$}{}
            \Statex
        \EndWhile \Compl{Blue}{Costo del \emph{mientras}: }{$n^2$}{veces $\times$ $\mathcal O(n^3 + n^2 + log(n)) = \mathcal O(n^5)$ }
    \EndIf \Compl{Brown}{Costo \emph{si}: }{$n^5 + n$}{$= \mathcal O(n^5)$}
    \State \Return{$\delta(K)$, $K$} \Compl{Brown}{}{$1$\label{bl:return}}{}
    \Statex
    \Statex \Compl{Brown}{Costo Total de la Heur\'istica: }{$n^5 + n$}{$= \mathcal O(n^5)$}
\end{pseudocodigo}

\bigskip


\subsection{Casos Particulares de Estudio}
    Empezaremos mencionando y caracterizando algunas familias de grafos
para las que nuestra heurist\'ica golosa constructiva siempre encuentra
un resultado correcto, esto es, una clique de m\'axima frontera.

Teniendo en cuenta, como ya se menciono anteriormente, que la heur\'istica
empieza por uno de los nodos de mayor grado y en cada paso va 
agregando nodos que forman clique con los nodos actualmente escogidos
bajo la condici\'on de que el grado de los mismos sea mayor a dos veces
el tama\~no de la clique actual (de estos candidatos, tambi\'en elige
uno de los de mayor grado). Tenemos lo siguiente:

En todos los casos en los que hay varios candidatos para agregar a la
clique con el mismo grado (en particular el primero) la heur\'istica
puede proporcionar el resultado incorrecto, dado que el algoritmo
elige determin\'isticamente uno de los candidatos mientras que la 
entrada es aleatoria (la idea es poder resolver el problema para
cualquier grafo de entrada), veremos a continuaci\'on que excepto
en casos muy particulares, grafos isomorfos cuyas matrices de adyacencia
son distintas pueden, al ser sometidos a la heur\'istica, arrojar
resultados muy dispares (en particular, resultados correctos, versus
resultados incorrectos).

\subsubsection{Grafos completos}
	En estos grafos como ya se mencion\'o anteriormente, el algoritmo
	escoge nodos y va haciendo crecer una clique hasta llegar a una 
	clique de tama\~no $n/2$, notes\'e que todos los nodos en los 
	grafos de esta familia tienen grado $d(i) = n - 1$, por lo
	tanto la selecci\'on en cada paso del siguiente nodo de la
	clique es dependiente de la implementaci\'on

	Tambi\'en anal\'iticamente puede verse que las CMF para esta 
	familia de grafos son tanto las $K_{\left \lfloor{n/2}\right \rfloor}$ 
	como las $K_{\left \lceil{n/2}\right \rceil}$. Habiendo establecido como
	cota superior al tama\~no de la CMF el valor $n/2$, la posibilidad de que
	los $K_{\left \lceil{n/2}\right \rceil}$ tambi\'en cumplan parece una 
	anomal\'ia a priori. Veremos que no lo es y que est\'a directamente
	relacionado con la naturaleza de la familia en estudio. 

	La frontera es la cantidad de aristas que sale de cada uno de los
	nodos de la clique hacia nodos que no est\'an en la misma. Siendo
	$n$ la cantidad de nodos del grafo y habiendo establecido que la CMF
	tiene tama~no $\left \lfloor{n/2}\right \rfloor$, 
	si la clique en estudio tiene este tama\~no, los nodos que quedan afuera
	de la clique son $\left \lceil{n/2} \right \rceil$, luego: 

	\( 
	\delta(K_{\left \lfloor{n/2}\right \rfloor}) = 
	\left \lfloor{n/2} \right \rfloor \times 
	(n - \left \lfloor{n/2} \right \rfloor ) = 
	\left \lfloor{n/2} \right \rfloor \times 
	(\left \lfloor{n/2} \right \rfloor +
	\left \lceil{n/2} \right \rceil - 
	\left \lfloor{n/2} \right \rfloor) =
	\left \lfloor{n/2} \right \rfloor \times
	\left \lceil{n/2} \right \rceil = 
	\delta(K_{\left \lceil{n/2}\right \rceil})
	\)
\begin{figure}[H]
\caption{Ejemplos grafos completos - Entrada / Salida}
\centering
%\includegraphics[scale = 0.5]{img/ej2/k2.png}
%\includegraphics[scale = 0.5]{img/ej2/k3.png}
\includegraphics[scale = 0.5]{img/ej2/k4.png}
\includegraphics[scale = 0.5]{img/ej2/k5.png}
\includegraphics[scale = 0.5]{img/ej2/k7.png}
\end{figure}

\subsection{\'Arboles}
La familia de los \'arboles es una de las familias en las que la heur\'istica
se puede romper y no devolver el resultado correcto. 

Para los \'arboles es pr\'acticamente inmediato que el 
tama\~no de la clique de m\'axima frontera es a lo sumo igual a dos, dado
que por su misma definici\'on los arboles no contienen circuitos simples.

Nuevamente en este caso se toma alguno de los nodos de mayor grado 
y se intenta hacer crecer la clique con alg\'un nodo adyacente.

Como puede verse en los siguientes ejemplos en el caso de que 
implementativamente se seleccione uno de los nodos de mayor grado
adyacente a otro de los nodos de mayor grado, la heur\'istica encuentra
efectivamente una CMF, pero en el caso de que el nodo de partida sea 
uno de los de mayor grado en el grafo y dicho nodo no pertenezca a
ninguna de las CMF del grafo, la heur\'istica reportar\'a un falso 
positivo (una clique cuya frontera es maximal pero no m\'axima).

Desde un punto de vista implementativo el desempate entre nodos de 
mismo grado, depender\'a de la matriz de adyacencia (la posici\'on en
la que aparece un nodo) y dado que existe un arbol isomorfo al del 
ejemplo en el que el nodo incorrecto para empezar la clique est\'a
listado antes que el nodo correcto (el que permite generar la CMF),
el siguiente es un buen ejemplo para ilustrar la fragilidad de la
heur\'istica

\begin{figure}[H]
\caption{Ejemplos \'Arboles}
\centering
\includegraphics[scale = 0.5]{img/ej2/tree_st0.png}
\includegraphics[scale = 0.5]{img/ej2/tree_st01.png}
\includegraphics[scale = 0.5]{img/ej2/tree_st02.png}
\includegraphics[scale = 0.5]{img/ej2/tree_st11.png}
\includegraphics[scale = 0.5]{img/ej2/tree_st12.png}
\end{figure}

\subsection{Grafos bipartitos}

\subsection{Grafos circulares}

\subsection{Estrellas}

\subsection{Grafo de secuencias}

\subsection{Banana tree}
	




La familia de gr\'afos que asegura un resultado incorrecto de nuestra 
heur\'istica golosa puede ser caracterizada como aquella en la que el
nodo de mayor grado no forma parte de la clique de m\'axima frontera.
Esto sucede ya que la heur\'istica construye una clique a partir del 
nodo de grado m\'aximo y en cada paso se mantiene el invariante de 
de clique, agregando solamente nodos cuyo grado sea mayor al tama\~no
de la clique en ese paso, por lo tanto si alguno de los nodos de la 
clique de m\'axima frontera no forma clique con el nodo 
inicial (el de mayor grado) entonces la heuristica constructiva no 
puede llegar a incluir a ese nodo.

Ejemplos:

%\begin{tikzpicture}
%	\SetGraphUnit{1.5cm}
%	\GraphInit[vstyle=Normal]
%
%	\Vertices{circle}{2,5,6,7,8}
%	\WE(2){1}
%	\Vertices{circle}{3,10,9,11}
%	\WE(3){2}
%
%	\foreach \v in {2,5,6,7,8}{\Edge(1)(\v)};
%
%\end{tikzpicture}
	
\begin{tikzpicture}
	\path 	
			% Nodos
			(0,8) node [shape=circle, draw] (1) {1}
			(1,8) node [shape=circle, draw] (2) {2}
			(2,8) node [shape=circle, draw] (3) {3}
			(6,8) node [shape=circle, draw] (4) {4}
			(7,8) node [shape=circle, draw] (5) {5}
			(8,8) node [shape=circle, draw] (6) {6}
			(0,7) node [shape=circle, draw] (7) {7}
			(1,7) node [shape=circle, draw] (8) {8}
			(7,7) node [shape=circle, draw] (9) {9}
			(8,7) node [shape=circle, draw] (10) {10}
			(0,6) node [shape=circle, draw] (11) {11}
			(1,6) node [shape=circle, draw] (12) {12}
			(7,6) node [shape=circle, draw] (13) {13}
			(8,6) node [shape=circle, draw] (14) {14}
			(4,5) node [shape=circle, draw] (15) {15}
			(2,4) node [shape=circle, draw] (16) {16}
			(3,4) node [shape=circle, draw] (17) {17}
			(5,4) node [shape=circle, draw] (18) {18}
			(6,4) node [shape=circle, draw] (19) {19}
			(2,3) node [shape=circle, draw] (20) {20}
			(3,3) node [shape=circle, draw] (21) {21}
			(4,3) node [shape=circle, draw] (22) {22}
			(5,3) node [shape=circle, draw] (23) {23}
			(6,3) node [shape=circle, draw] (24) {24}
			(2,2) node [shape=circle, draw] (25) {25}
			(6,2) node [shape=circle, draw] (26) {26}
			(4, 1.5) node [shape=circle, draw] (27) {27}
			(2,1) node [shape=circle, draw] (28) {28}
			(6,1) node [shape=circle, draw] (29) {29}
			(2,0) node [shape=circle, draw] (30) {30}
			(3,0) node [shape=circle, draw] (31) {31}
			(5,0) node [shape=circle, draw] (32) {32}
			(6,0) node [shape=circle, draw] (33) {33};
			% Aristas de la semiestrella superior izq
			\draw[-] (1) -- (8);
			\draw[-] (2) -- (8);
			\draw[-] (3) -- (8);
			\draw[-] (7) -- (8);
			\draw[-] (11) -- (8);
			\draw[-] (12) -- (8);

			% Aristas de la semiestrella superior izq
			\draw[-] (4) -- (9);
			\draw[-] (5) -- (9);
			\draw[-] (6) -- (9);
			\draw[-] (10) -- (9);
			\draw[-] (13) -- (9);
			\draw[-] (14) -- (9);

			% Mas aristas
			
			\draw[-] (8) -- (9);
			\draw[-] (8) -- (15);
			\draw[-] (9) -- (15);
			\draw[-] (16) -- (15);
			\draw[-] (17) -- (15);
			\draw[-] (18) -- (15);
			\draw[-] (19) -- (15);
			\draw[-] (22) -- (15);

			% Aristas de la estrella inferior

			\draw[-] (20) -- (27);
			\draw[-] (21) -- (27);
			\draw[-] (22) -- (27);
			\draw[-] (23) -- (27);
			\draw[-] (24) -- (27);
			\draw[-] (25) -- (27);
			\draw[-] (26) -- (27);
			\draw[-] (28) -- (27);
			\draw[-] (29) -- (27);
			\draw[-] (30) -- (27);
			\draw[-] (31) -- (27);
			\draw[-] (32) -- (27);
			\draw[-] (33) -- (27);

			
\end{tikzpicture}


\subsection{Experimentaci\'on}
    \subsubsection{Resultados}

%Doble estrella
\begin{figure}[H]
    \centering
    \fontsize{8}{10}\selectfont
    \resizebox{0.87\textwidth}{!}{\input{img/ej3/tabu_search/ej3_nodos_nlogn_star+bridge+double_star.tex}}
    \caption{Complejidad temporal para grafos Estrella+Puente+Doble Estrella (Variante Max\_Iter=nlog(n))}
\end{figure}

\begin{figure}[H]
    \centering
    \fontsize{8}{10}\selectfont
    \resizebox{0.87\textwidth}{!}{\input{img/ej3/tabu_search/ej3_nodos_n_star+bridge+double_star.tex}}
    \caption{Complejidad temporal para grafos Estrella+Puente+Doble Estrella (Variante Max\_Iter=n)}
\end{figure}

\begin{figure}[H]
    \centering
    \fontsize{8}{10}\selectfont
    \resizebox{0.87\textwidth}{!}{\input{img/ej3/tabu_search/ej3_frontera_star_bridge_double_star.tex}}
    \caption{Frontera de grafos Estrella+Puente+Doble Estrella}
\end{figure}

%Doble estrella sin aspiracion
\begin{figure}[H]
    \centering
    \fontsize{8}{10}\selectfont
    \resizebox{0.87\textwidth}{!}{\input{img/ej3/tabu_search/ej3_nodos_nlogn_star+bridge+double_star_sin_aspiracion.tex}}
    \caption{Complejidad temporal para grafos Estrella+Puente+Doble Estrella (Variante Max\_Iter=nlog(n),sin aspiraci\'on)}
\end{figure}

\begin{figure}[H]
    \centering
    \fontsize{8}{10}\selectfont
    \resizebox{0.87\textwidth}{!}{\input{img/ej3/tabu_search/ej3_nodos_n_star+bridge+double_star_sin_aspiracion.tex}}
    \caption{Complejidad temporal para grafos Estrella+Puente+Doble Estrella (Variante Max\_Iter=n),sin aspiraci\'on)}
\end{figure}

\begin{figure}[H]
    \centering
    \fontsize{8}{10}\selectfont
    \resizebox{0.87\textwidth}{!}{\input{img/ej3/tabu_search/ej3_frontera_star_bridge_double_star_sin_aspiracion.tex}}
    \caption{Frontera de grafos Estrella+Puente+Doble Estrella (sin aspiracion)}
\end{figure}

%Doble estrella sin aspiracion golosa
\begin{figure}[H]
    \centering
    \fontsize{8}{10}\selectfont
    \resizebox{0.87\textwidth}{!}{\input{img/ej3/tabu_search/ej3_nodos_nlogn_star+bridge+double_star_sin_aspiracion_golosa.tex}}
    \caption{Complejidad temporal para grafos Estrella+Puente+Doble Estrella (Variante Max\_Iter=nlog(n),sin aspiraci\'on,golosa)}
\end{figure}

\begin{figure}[H]
    \centering
    \fontsize{8}{10}\selectfont
    \resizebox{0.87\textwidth}{!}{\input{img/ej3/tabu_search/ej3_nodos_n_star+bridge+double_star_sin_aspiracion_golosa.tex}}
    \caption{Complejidad temporal para grafos Estrella+Puente+Doble Estrella (Variante Max\_Iter=n),sin aspiraci\'on,golosa)}
\end{figure}

\begin{figure}[H]
    \centering
    \fontsize{8}{10}\selectfont
    \resizebox{0.87\textwidth}{!}{\input{img/ej3/tabu_search/ej3_frontera_star_bridge_double_star_sin_aspiracion_golosa.tex}}
    \caption{Frontera de grafos Estrella+Puente+Doble Estrella (sin aspiracion, golosa)}
\end{figure}

%Bipartito Completo
\begin{figure}[H]
    \centering
    \fontsize{8}{10}\selectfont
    \resizebox{0.87\textwidth}{!}{\input{img/ej3/tabu_search/ej3_nodos_nlogn_complete_bipartite.tex}}
    \caption{Complejidad temporal para grafos Bipartito Completo (Variante Max\_Iter=nlog(n))}
\end{figure}

\begin{figure}[H]
    \centering
    \fontsize{8}{10}\selectfont
    \resizebox{0.87\textwidth}{!}{\input{img/ej3/tabu_search/ej3_nodos_n_complete_bipartite.tex}}
    \caption{Complejidad temporal para grafos Bipartito Completo (Variante Max\_Iter=n)}
\end{figure}

\begin{figure}[H]
    \centering
    \fontsize{8}{10}\selectfont
    \resizebox{0.87\textwidth}{!}{\input{img/ej3/tabu_search/ej3_frontera_complete_bipartite.tex}}
    \caption{Frontera de grafos Bipartito Completo}
\end{figure}

%Bipartito Completo sin aspiracion
\begin{figure}[H]
    \centering
    \fontsize{8}{10}\selectfont
    \resizebox{0.87\textwidth}{!}{\input{img/ej3/tabu_search/ej3_nodos_nlogn_complete_bipartite_sin_aspiracion.tex}}
    \caption{Complejidad temporal para grafos Bipartito Completo (Variante Max\_Iter=nlog(n),sin aspiraci\'on)}
\end{figure}

\begin{figure}[H]
    \centering
    \fontsize{8}{10}\selectfont
    \resizebox{0.87\textwidth}{!}{\input{img/ej3/tabu_search/ej3_nodos_n_complete_bipartite_sin_aspiracion.tex}}
    \caption{Complejidad temporal para grafos Bipartito Completo (Variante Max\_Iter=n),sin aspiraci\'on)}
\end{figure}

\begin{figure}[H]
    \centering
    \fontsize{8}{10}\selectfont
    \resizebox{0.87\textwidth}{!}{\input{img/ej3/tabu_search/ej3_frontera_complete_bipartite_sin_aspiracion.tex}}
    \caption{Frontera de grafos Bipartito Completo (sin aspiracion)}
\end{figure}

%Bipartito Completo sin aspiracion golosa
\begin{figure}[H]
    \centering
    \fontsize{8}{10}\selectfont
    \resizebox{0.87\textwidth}{!}{\input{img/ej3/tabu_search/ej3_nodos_nlogn_complete_bipartite_sin_aspiracion_golosa.tex}}
    \caption{Complejidad temporal para grafos Bipartito Completo (Variante Max\_Iter=nlog(n),sin aspiraci\'on,golosa)}
\end{figure}

\begin{figure}[H]
    \centering
    \fontsize{8}{10}\selectfont
    \resizebox{0.87\textwidth}{!}{\input{img/ej3/tabu_search/ej3_nodos_n_complete_bipartite_sin_aspiracion_golosa.tex}}
    \caption{Complejidad temporal para grafos Bipartito Completo (Variante Max\_Iter=n),sin aspiraci\'on,golosa)}
\end{figure}

\begin{figure}[H]
    \centering
    \fontsize{8}{10}\selectfont
    \resizebox{0.87\textwidth}{!}{% GNUPLOT: LaTeX picture with Postscript
\begingroup
  \makeatletter
  \providecommand\color[2][]{%
    \GenericError{(gnuplot) \space\space\space\@spaces}{%
      Package color not loaded in conjunction with
      terminal option `colourtext'%
    }{See the gnuplot documentation for explanation.%
    }{Either use 'blacktext' in gnuplot or load the package
      color.sty in LaTeX.}%
    \renewcommand\color[2][]{}%
  }%
  \providecommand\includegraphics[2][]{%
    \GenericError{(gnuplot) \space\space\space\@spaces}{%
      Package graphicx or graphics not loaded%
    }{See the gnuplot documentation for explanation.%
    }{The gnuplot epslatex terminal needs graphicx.sty or graphics.sty.}%
    \renewcommand\includegraphics[2][]{}%
  }%
  \providecommand\rotatebox[2]{#2}%
  \@ifundefined{ifGPcolor}{%
    \newif\ifGPcolor
    \GPcolortrue
  }{}%
  \@ifundefined{ifGPblacktext}{%
    \newif\ifGPblacktext
    \GPblacktexttrue
  }{}%
  % define a \g@addto@macro without @ in the name:
  \let\gplgaddtomacro\g@addto@macro
  % define empty templates for all commands taking text:
  \gdef\gplbacktext{}%
  \gdef\gplfronttext{}%
  \makeatother
  \ifGPblacktext
    % no textcolor at all
    \def\colorrgb#1{}%
    \def\colorgray#1{}%
  \else
    % gray or color?
    \ifGPcolor
      \def\colorrgb#1{\color[rgb]{#1}}%
      \def\colorgray#1{\color[gray]{#1}}%
      \expandafter\def\csname LTw\endcsname{\color{white}}%
      \expandafter\def\csname LTb\endcsname{\color{black}}%
      \expandafter\def\csname LTa\endcsname{\color{black}}%
      \expandafter\def\csname LT0\endcsname{\color[rgb]{1,0,0}}%
      \expandafter\def\csname LT1\endcsname{\color[rgb]{0,1,0}}%
      \expandafter\def\csname LT2\endcsname{\color[rgb]{0,0,1}}%
      \expandafter\def\csname LT3\endcsname{\color[rgb]{1,0,1}}%
      \expandafter\def\csname LT4\endcsname{\color[rgb]{0,1,1}}%
      \expandafter\def\csname LT5\endcsname{\color[rgb]{1,1,0}}%
      \expandafter\def\csname LT6\endcsname{\color[rgb]{0,0,0}}%
      \expandafter\def\csname LT7\endcsname{\color[rgb]{1,0.3,0}}%
      \expandafter\def\csname LT8\endcsname{\color[rgb]{0.5,0.5,0.5}}%
    \else
      % gray
      \def\colorrgb#1{\color{black}}%
      \def\colorgray#1{\color[gray]{#1}}%
      \expandafter\def\csname LTw\endcsname{\color{white}}%
      \expandafter\def\csname LTb\endcsname{\color{black}}%
      \expandafter\def\csname LTa\endcsname{\color{black}}%
      \expandafter\def\csname LT0\endcsname{\color{black}}%
      \expandafter\def\csname LT1\endcsname{\color{black}}%
      \expandafter\def\csname LT2\endcsname{\color{black}}%
      \expandafter\def\csname LT3\endcsname{\color{black}}%
      \expandafter\def\csname LT4\endcsname{\color{black}}%
      \expandafter\def\csname LT5\endcsname{\color{black}}%
      \expandafter\def\csname LT6\endcsname{\color{black}}%
      \expandafter\def\csname LT7\endcsname{\color{black}}%
      \expandafter\def\csname LT8\endcsname{\color{black}}%
    \fi
  \fi
  \setlength{\unitlength}{0.0500bp}%
  \begin{picture}(7200.00,5040.00)%
    \gplgaddtomacro\gplbacktext{%
      \csname LTb\endcsname%
      \put(1298,2904){\makebox(0,0)[r]{\strut{} 0}}%
      \csname LTb\endcsname%
      \put(1298,3052){\makebox(0,0)[r]{\strut{} 500}}%
      \csname LTb\endcsname%
      \put(1298,3199){\makebox(0,0)[r]{\strut{} 1000}}%
      \csname LTb\endcsname%
      \put(1298,3347){\makebox(0,0)[r]{\strut{} 1500}}%
      \csname LTb\endcsname%
      \put(1298,3494){\makebox(0,0)[r]{\strut{} 2000}}%
      \csname LTb\endcsname%
      \put(1298,3642){\makebox(0,0)[r]{\strut{} 2500}}%
      \csname LTb\endcsname%
      \put(1298,3789){\makebox(0,0)[r]{\strut{} 3000}}%
      \csname LTb\endcsname%
      \put(1298,3937){\makebox(0,0)[r]{\strut{} 3500}}%
      \csname LTb\endcsname%
      \put(1298,4084){\makebox(0,0)[r]{\strut{} 4000}}%
      \csname LTb\endcsname%
      \put(1298,4232){\makebox(0,0)[r]{\strut{} 4500}}%
      \csname LTb\endcsname%
      \put(1298,4379){\makebox(0,0)[r]{\strut{} 5000}}%
      \csname LTb\endcsname%
      \put(1430,2684){\makebox(0,0){\strut{} 0}}%
      \csname LTb\endcsname%
      \put(1967,2684){\makebox(0,0){\strut{} 500}}%
      \csname LTb\endcsname%
      \put(2505,2684){\makebox(0,0){\strut{} 1000}}%
      \csname LTb\endcsname%
      \put(3042,2684){\makebox(0,0){\strut{} 1500}}%
      \csname LTb\endcsname%
      \put(3579,2684){\makebox(0,0){\strut{} 2000}}%
      \csname LTb\endcsname%
      \put(4117,2684){\makebox(0,0){\strut{} 2500}}%
      \csname LTb\endcsname%
      \put(4654,2684){\makebox(0,0){\strut{} 3000}}%
      \csname LTb\endcsname%
      \put(5191,2684){\makebox(0,0){\strut{} 3500}}%
      \csname LTb\endcsname%
      \put(5728,2684){\makebox(0,0){\strut{} 4000}}%
      \csname LTb\endcsname%
      \put(6266,2684){\makebox(0,0){\strut{} 4500}}%
      \csname LTb\endcsname%
      \put(6803,2684){\makebox(0,0){\strut{} 5000}}%
      \put(176,3641){\rotatebox{-270}{\makebox(0,0){\strut{}Frontera}}}%
      \put(396,3641){\rotatebox{-270}{\makebox(0,0){\strut{}(Escala Lineal)}}}%
      \put(4116,2354){\makebox(0,0){\strut{}Cantidad de Nodos}}%
      \put(4116,2134){\makebox(0,0){\strut{}(Escala Lineal)}}%
      \put(4116,4709){\makebox(0,0){\strut{}Frontera obtenida segun cantidad de nodos}}%
    }%
    \gplgaddtomacro\gplfronttext{%
      \csname LTb\endcsname%
      \put(6461,1713){\makebox(0,0)[r]{\strut{}Iter=nlog(n),Sin Mejorar=n,Tiempo Tabu=n}}%
      \csname LTb\endcsname%
      \put(6461,1493){\makebox(0,0)[r]{\strut{}Iter=nlog(n),Sin Mejorar=n,Tiempo Tabu=n2}}%
      \csname LTb\endcsname%
      \put(6461,1273){\makebox(0,0)[r]{\strut{}Iter=nlog(n),Sin Mejorar=n2,Tiempo Tabu=n}}%
      \csname LTb\endcsname%
      \put(6461,1053){\makebox(0,0)[r]{\strut{}Iter=nlog(n),Sin Mejorar=n2,Tiempo Tabu=n2}}%
      \csname LTb\endcsname%
      \put(6461,833){\makebox(0,0)[r]{\strut{}Iter=n,Sin Mejorar=n,Tiempo Tabu=n}}%
      \csname LTb\endcsname%
      \put(6461,613){\makebox(0,0)[r]{\strut{}Iter=n,Sin Mejorar=n,Tiempo Tabu=n2}}%
      \csname LTb\endcsname%
      \put(6461,393){\makebox(0,0)[r]{\strut{}Iter=n,Sin Mejorar=n2,Tiempo Tabu=n}}%
      \csname LTb\endcsname%
      \put(6461,173){\makebox(0,0)[r]{\strut{}Iter=n,Sin Mejorar=n2,Tiempo Tabu=n2}}%
    }%
    \gplbacktext
    \put(0,0){\includegraphics{ej3_frontera_complete_bipartite_sin_aspiracion_golosa}}%
    \gplfronttext
  \end{picture}%
\endgroup
}
    \caption{Frontera de grafos Bipartito Completo (sin aspiracion,golosa)}
\end{figure}

\subsubsection{Conclusiones}


        \pagebreak

    \section{Heur\'istica de B\'usqueda Local}
        \subsection{Resoluci\'on}
    \begin{pseudocodigo}[Algoritmo Exacto para \emph{CMF} - Descriptivo]
    \Require\Statex
        \begin{itemize}
            \item Un grafo $G$ de $n$ v\'ertices\footnote{Asumimos, sin p\'erdida 
                de generalidad, que los v\'ertices est\'an numerados de 1 a $n$.}
                y $m$ aristas.

            \item Una funci\'on $candidatos(K)$, que dado un conjunto de v\'ertices
                $K$, devuelve una secuencia de v\'ertices de $G$ ordenada por grado
                de mayor a menor que son adjacentes a todos los elementos de $K$ y
                cuyo grado es mayor $2|K|$.

            \item Una funci\'on $\delta(K)$, que dada una \emph{clique} $K$\footnote{Una
                \emph{clique} no es otra cosa que un conjunto de v\'ertices, con la
                caracter\'istica de inducir un grafo completo en $G$.} en $G$, calcula
                el cardinal de su frontera\footnote{Como ya se explic\'o, sea $K$ una
                \emph{clique} de $G$, entonces: $\delta(K) = - |K|(|K|-1) +
                \displaystyle\sum_{v \in K} d(v)$}.

            \item Una funci\'on $\delta_{cota}(K)$, que dada una \emph{clique} $K$
                en $G$, calcula la cota de su frontera seg\'un el m\'etodo
                ya explicado anteriormente.

        \end{itemize}
    \Statex
    \Ensure Una \emph{clique} $K$ de $G$ con una frontera $\delta(K)$ de m\'axima
        cardinalidad.

    \Statex

    \If{$m = \dfrac{n(n-1)}{2}$} \Comment{Si $G$ es un $K_n$}
        \State $K \gets \left\{1;\dots;\left\lfloor\dfrac{n}{2}\right\rfloor\right\}$

    \Else
        \State $\delta_{max} \gets \left\lfloor\dfrac{r+1}{2}\right\rfloor\cdot
            \left\lceil\dfrac{r+1}{2}\right\rceil$ para $r$ tal que:
            $\begin{pmatrix}
                \text{\Huge$\bigwedge$} &
                    \begin{matrix}
                        r \in [1..n)\\[0.5cm]
                        r < \dfrac{n^2}{n^2 - 2m}\\[0.5cm]
                        (\forall r' \in [1..n))$ $r' < \dfrac{n^2}{n^2 - 2m} \implies r'\leq r
                    \end{matrix}
            \end{pmatrix}$\footnote{Esta cota inferior para la cardinalidad de la
                frontera maximal en $G$ se deriv\'o a partir del \emph{Teorema de Tur\'an},
                como ya se explic\'o anteriormente.}

        \State $K \gets \emptyset$, $K' \gets \emptyset$, $YaProcesados(K')$\footnote{%
            $YaProcesados(K')$ representar\'a a los v\'ertices de $candidatos(K')$ que
            ya fueron tenidos en cuenta para la \emph{clique} $K'$.}$ \gets \emptyset$

        \While{$candidatos(K')$\footnote{Si $K'$ es vac\'io, $candidatos(K')$ son todos
            los nodos de $G$.}$\setminus YaProcesados(K')$\footnote{Peque\~no abuso
            de notaci\'on, ya que $candidatos(K')$ es una secuencia y $YaProcesados(K')$
            es un conjunto. Se asume que es la secuencia sin los elementos del conjunto.}$
            \neq \emptyset$ $\lor$ $K' \neq \emptyset$}

            \If{$candidatos(K') \setminus YaProcesados(K') \neq \emptyset$ $\land$ $\delta_{max} < \delta_{cota}(K')$%
                \footnote{Si $K'$ es vac\'io, $\delta_{cota}(K')$ es $m$.}}

                \State $v \gets cabeza(candidatos(K') \setminus YaProcesados(K'))$
                \State $K' \gets K' \cup \{v'\}$
                \If{$candidatos(K') = \emptyset$ $\land$ $\delta_{max} < \delta(K')$}
                    %\Comment{Llegu\'e al final de la rama y obtuve una mejor solucion}
                    \State $\delta_{max} \gets \delta(K')$
                    \State $K \gets K'$
                \EndIf

            \ElsIf{$K' \neq \emptyset$} \Comment{\emph{Backtracking}}
                \State $v \gets ultimo(K')$\footnote{Donde $ultimo(K')$
                    es el \'ulitmo v\'ertice agregado a $K'$.}
                \State $K' \gets K' \setminus \{v'\}$
                \State $YaProcesados(K') \gets YaProcesados(K') \cup \{v\}$

            \Else
                \State $YaProcesados(K') \gets YaProcesados(K') \cup candidatos(K')$
            \EndIf
        \EndWhile
    \EndIf

    \State \Return{$K$}
\end{pseudocodigo}


\subsection{Complejidad}
    \subsubsection{Estructuras de Datos}
\par Como ya se mencion\'o en~\nameref{grafo:estructuras}, ya contamos en el grafo
    de entrada con listas/vectores de adyacencias y una matriz de adyacencias.

\par Al igual que en la heur\'istica golosa y el algoritmo exacto, decidimos
    modelar las cliques con vectores%
    \footnote{\url{http://www.cplusplus.com/reference/vector/vector/}}.

\par A su vez, para mejorar la eficiencia al recorrer la vecindad, se
    utilizan un vector/clique como \emph{min heaps} seg\'un el grado
    de los nodos, permitiendonos acceder en $\mathcal O(1)$ al nodo
    de menor grado. Los motivos de este proceder se detallan en
    la siguiente secci\'on.

\subsubsection{Pseudoc\'odigo de complejidad}
\par Se presenta a continuaci\'on un pseudoc\'odigo m\'as espec\'ifico de la implementaci\'on
    de este algoritmo provista junto con este trabajo. El mismo tiene en cuenta
    las estructuras de datos explicadas en el punto anterior.

\par Luego del pseudoc\'odigo se justifican detalladamente las complejidades
    expuestas a continuaci\'on que no sean evidentes\footnote{Consideramos
    como "complejidades evidentes" las asignaciones de variables, operaciones
    m\'atematicas simples, asignaciones/inicializaci\'on de posiciones de
    un vector/\emph{deque} o cualquier contenedor de acceso aleatorio/arbitrario}.

\par Debemos aclarar, debido a las variantes ya presentadas sobre esta
    heur\'istica, que el siguiente an\'alisis de complejidad aplica
    sobre la heur\'istica que selecciona al primer vecino v\'alido encontrado,
    en una vecindad donde s\'olo se consideran las cliques con un nodo m\'as
    o un nodo menos y que selecciona como clique inicial a alg\'un nodo
    de grado mayor o igual al grado promedio del grafo.

\par El an\'alisis de complejidad de las variantes se realiza en la secci\'on
    \ref{bl:compl:variantes}.

\bigskip

\begin{pseudocodigo}[Heur\'istica de B\'usqueda Local para \emph{CMF} - Complejidad]
    \Require Un grafo $G$ con $n$ v\'ertices numerados de $1$ a $n$ y $m$ aristas. El mismo
        cuenta con las siguientes estructuras de datos que lo modelan:
        \begin{itemize}
            \item Vectores de adyacencia: Dado un vertice $v$, $vecinos(v)$ nos da todos los
                nodos adyacentes a $v$ en $G$.

            \item Matriz de adyacencia: Dados los v\'ertices $v$ y $w$, $adyacentes(v,w)$ y
                $adyacentes(w,v)$ nos devuelven $true$ si y s\'olo si $v$ es adyacente
                a $w$ en $G$.

            \item Vector de nodos de $G$.
        \end{itemize}
    \Ensure\Statex
        \begin{itemize}
            \item Un vector $K$ correspondiente a la \emph{clique} de m\'axima frontera
                encontrada por la heur\'istica.

            \item El cardinal de $\delta(K)$, siendo $K$ la \emph{clique} del item anterior.
        \end{itemize}
    \Statex
    \State $K \gets \emptyset$ \Compl{Brown}{}{$n$}{}
    \If{$m = \frac{n(n-1)}{2}$} \Compl{Blue}{}{$1$}{}
        \State $K \gets \left\{1;\dots;\left\lfloor\sfrac{n}{2}\right\rfloor\right\}$ \Compl{Blue}{}{$n$}{}
        \State $\delta_{max} \gets \left\lfloor\sfrac{n}{2}\right\rfloor\cdot
            \left\lceil\sfrac{n}{2}\right\rceil$ \Compl{Blue}{}{$1$}{}
        \Statex
    \Else
        \State $K \gets$ Primer nodo de grado mayor o igual al grado promedio \Compl{Blue}{}{$n$}{}
        \State $\delta(K) \gets d(v)$ \Compl{Blue}{}{$1$}{}
        \If{$\exists v \in vecinos(v)$ tal que $d(v) > 2|K|$} \Compl{Red}{}{$n$}{}
            \State $v_{add} \gets v$ \Compl{Red}{}{$1$}{}
            \State $\delta(K+v_{add}) \gets \delta(K) + d(v_{add}) - 2|K|$ \Compl{Red}{}{$1$}{}
        \EndIf \Compl{Blue}{Costo del \emph{si}: }{$n$}{}
        \Statex
        \State $v_{rem} \gets top(K)$ \Compl{Blue}{}{$1$}{}
        \State $\delta(K-v_{rem}) \gets \delta(K)-d(v_{rem})+2(|K|-1)$ \Compl{Blue}{}{$1$}{}
        \Statex
        \While{$\delta(K) < \delta(K+v_{add})$ $\lor$ $\delta(K) < \delta(K-v_{rem})$} \Compl{Red}{}{$1$}{}
            \If{$\delta(K+v_{add}) > \delta(K-v_{rem})$} \Compl{Fuchsia}{}{$1$}{}
                \State $push\_heap(K,v_{add})$ \Compl{Fuchsia}{}{$log(n)$}{}
                \State $\delta(K) \gets \delta(K+v_{add})$ \Compl{Fuchsia}{}{$1$}{}
                \Statex
            \Else
                \State $pop\_heap(K)$ \Compl{Fuchsia}{}{$log(n)$}{}
                \State $\delta(K) \gets \delta(K-v_{rem})$ \Compl{Fuchsia}{}{$1$}{}
                \Statex
            \EndIf \Compl{Red}{Costo del \emph{si}: }{$log(n)$}{}
            \Statex
            \Statex $\triangleright$ Ya salte a un vecino, recorro la nueva vecindad
            \If{$\exists v \in candidatos(K)$ tal que $d(v) > 2|K|$} \Compl{Fuchsia}{}{$n^2$}{}
                \State $v_{add} \gets v$ \Compl{Fuchsia}{}{$1$}{}
                \State $\delta(K+v_{add}) \gets \delta(K) + d(v_{add}) - 2|K|$ \Compl{Fuchsia}{}{$1$}{}
            \EndIf \Compl{Red}{Costo del \emph{si}: }{$n^2$}{}
            \Statex
            \State $v_{rem} \gets top(K)$ \Compl{Red}{}{$1$}{}
            \State $\delta(K-v_{rem}) \gets \delta(K)-d(v_{rem})+2(|K|-1)$ \Compl{Red}{}{$1$}{}
            \Statex
        \EndWhile \Compl{Blue}{Costo del \emph{mientras}: }{$n^2$}{veces $\times$ $\mathcal O(n^2 + log(n)) = \mathcal O(n^4)$ }
    \EndIf \Compl{Brown}{Costo \emph{si}: }{$n^4 + n$}{$= \mathcal O(n^4)$}
    \State \Return{$\delta(K)$, $K$} \Compl{Brown}{}{$1$\label{bl:return}}{}
    \Statex
    \Statex \Compl{Brown}{Costo Total de la Heur\'istica: }{$n^4 +n$}{$= \mathcal O(n^4)$}
\end{pseudocodigo}

\bigskip

\par La primera parte del algoritmo es exactamente igual a la de la
    heur\'istica golosa y el algoritmo exacto: reconoce en $\mathcal O(1)$
    si el grafo $G$ es un $K_n$ y en caso afirmativo, devuelve
    la \emph{CMF} con costo lineal. Esto ya fue justificado en
    los dos casos previos, as\'i pues se omite su justificaci\'on.

\par Luego, en caso de estar ante un $G$ que no es un grafo completo,
    el algoritmo busca a un nodo que tenga grado mayor o igual al
    promedio. Esta b\'usqueda se realiza sobre el vector de nodos
    (los cuales permiten el acceso a su grado en $\mathcal O(1)$),
    con lo cual el costo en el peor caso pasa a ser revisar
    todos los nodos del grafo\footnote{\url{http://www.cplusplus.com/reference/algorithm/find_if/}},
    siendo la complejidad entonces $\mathcal O(n)$. Habiendo
    escogido el nodo/clique inicial, se guarda la frontera
    parcial correspondiente (que no es otra cosa que el grado
    de dicho nodo, que como se dijo, es accedible en tiempo
    constante) y se asigna dicho nodo al vector que representa
    la clique (lo cual tiene un costo constante tambien\footnote{%
    \url{http://www.cplusplus.com/reference/vector/vector/push_back/}}.

\par Luego, antes de comenzar el ciclo, se inicializan los posibles
    valores de la vecindad. En s\'i mismo, se recorrer la vecindad
    \emph{inicial}, guardando las primeras cliques vecinas (una
    con un nodo m\'as y otra con un nodo menos\footnote{Esta \'ultima
    clique es trivial ya que en este momento la clique se compone
    de s\'olo un nodo.}) que incrementen la funci\'on objetivo.

\par Buscar una clique que incremente la frontera es lineal ya
    que basta con recorrer el \emph{deque} de vecinos del
    \'unico nodo de la clique (el cual es accesible en tiempo
    constante gracias a el vector de nodos de la estructura del
    grafo y a la propia estructura de estos nodos\footnote{Ver
    \emph{\nameref{grafo:estructuras}}, en la secci\'on
    \emph{\nameref{notas_preliminares}}.} y realizar una
    \'unica comparacion de tiempo constante (verificar que
    efectivamente su grado incrementa la frontera).

\par Encontrado este nodo, calcular cuanto incrementa la funci\'on
    es trivial y se realiza en tiempo constante, ya que alcanza
    con hacer las operaciones matem\'aticas sobre la frontera
    de la clique actual y el grado del nodo que se a\~nadir\'ia%
    \footnote{Ver la Secci\'on \emph{\nameref{notas:calc_front}},
    en \emph{\nameref{notas_preliminares}}.}.

\par Luego, para encontrar un nodo que incremente la frontera
    si fuese eliminado de la clique se hace aprovechando
    las estructuras de un \emph{min heap} sobre los grados
    de la clique, utilizando la representaci\'on de este
    que no es otra que un vector \footnote{\url{%
    http://www.cplusplus.com/reference/algorithm/make_heap/}}.
    Si bien el heap en este caso no fue inicializado mediante
    la funci\'on \emph{make\_heap}, esto no es necesario ya
    que en esta variante, s\'olo se tiene un nodo en la clique.
    Luego, si mantenemos esta estructura en el vector, las operaciones
    $push\_heap$ y $pop_heap$ tendr\'an un coste logar\'itmico
    mientras que $top$ sera de coste constante\footnote{%
    \url{http://www.cplusplus.com/reference/algorithm/push_heap/},%
    \url{http://www.cplusplus.com/reference/algorithm/pop_heap/},%
    \url{http://www.cplusplus.com/reference/vector/vector/front/},%
    \url{http://www.cplusplus.com/reference/vector/vector/pop_back/},%
    \url{http://www.cplusplus.com/reference/vector/vector/push_back/}}.

\par Teniendo ya un \emph{min heap}, podemos acceder al nodo
    de menor grado de la clique actual en $\mathcal O(1)$

\par Nuevamente, calcular la frontera si se quitase dicho
    nodo es trivial. Pero ac\'a debemos justificar el hecho
    de no buscar alg\'un otro nodo que al ser quitado incremente
    la frontera en caso de que este nodo (el tope del heap) no
    lo haga. El fundamento aqu\'i es que el tope del heap
    es el nodo de menor grado, por lo tanto, es el nodo
    que menos frontera le a\~nade a $K$, por lo tanto,
    quitar cualquier otro nodo implicar\'ia quitar un nodo
    de mayor grado, lo que significa que estar\'iamos
    considerando quitar un nodo que aporta m\'as frontera
    que el tope del heap y por lo tanto estariamos
    obteniendo una frontera menor que la que se obtiene
    de quitar el tope. Esto nos permite asegurar que si
    se tiene el nodo de menor grado de la clique, y este
    no incrementa la frontera si se quitase de la clique,
    entonces ning\'un otro nodo de la clique lo har\'a.

\par Luego, comienza el ciclo de la b\'usqueda local. En la
    guarda del mismo se realizan s\'olo comparaciones
    de enteros que son almacenados en variables de la implementaci\'on
    de esta funci\'on de \emph{C++}, por lo cual su acceso
    tiene un coste $\mathcal O(1)$.

\par Inmediatamente al comenzar el ciclo hay un condicional
    \emph{si}. El mismo en su guarda realiza una operaci\'on
    $\mathcal O(1)$ por los mismos motivos explicados en
    el p\'arrafo anterior. Dentro de sus dos ramas vemos
    que se realizan 1 asignaci\'on ($\mathcal O(1)$) y
    un $push\_heap$ o $pop\_heap$, que tienen un coste
    logar\'itmico como ya se explico m\'as arriba.

\par En la segunda mitad del ciclo, se
    revisa la nueva vecindad (en este punto, el ciclo
    habr\'a a\~nadido o quitado alg\'un nodo de la
    clique, por lo cual tenemos una nueva vecindad). En
    el caso de los vecinos que se componen de cliques con
    un nodo menos, el proceso es id\'entico al realizado
    antes de entrar al ciclo, con lo cual no hace falta
    justificar nada m\'as. En cuanto a las cliques que
    contienen un nodo m\'as, ahora tenemos una clique
    con (posiblemente) m\'as de un nodo, por lo cual,
    debemos ver todos sus candidatos, pero no nos podemos
    basar en los nodos de la clique antes de realizar
    el salto al vecino pues si se quito un nodo, los
    candidatos podr\'ian incrementarse. Por lo tanto,
    debemos recorrer (en el peor de los casos) todos
    los nodos adyacentes a alg\'un nodo de la clique
    (que son accesibles en $\mathcal O(1)$ gracias al vector
    de nodos de la estructura de $G$) y verificar 2
    cosas: que sean adyacentes a todos los dem\'as
    nodos de la clique y que incrementen la frontera%
    \footnote{Es decir, que $d(v) > 2|K|$.}. En caso
    de encontrar uno, podemos detener este recorrido de posibles
    nodos candidatos, ya que no estamos buscando saltar
    al mejor vecino sino al primero que encontremos
    que cumpla con nuestros requerimientos. Pero a\'un as\'i,
    en el peor de los casos deberemos recorrerlos a todos,
    es decir, $\mathcal O(n)$ nodos adyacentes, y tambi\'en
    como cota para $|K|$ debemos considerar que podr\'ia
    estar en el orden de $\mathcal O(n)$ elementos, siendo
    el costo de esta b\'usqueda de vecino candidato
    con un nodo m\'as, igual a $\mathcal O(n^2)$.

\par Por \'ultimo, ya teniendo calculado el coste del ciclo,
    debemos acotar la cantidad de iteraciones que tendr\'a.
    Esto es d\'ificil, ya que sin mayor informaci\'on sobre
    el grafo de entrada y su estructura, no se puede saber
    a ciencia cierta cuantas veces podr\'a "deplazarse" el
    algoritmo en la vecindad definida. Por lo tanto, se
    decidi\'o acotar mediante lo m\'inimo que crece la frontera
    entre iteraci\'on e iteraci\'on. Este m\'inimo es, simplemente,
    1. Considerando este peor caso, donde la funci\'on objetivo
    crece muy lentamente, y considerando que la frontera m\'axima
    no puede ser mayor a $m$ (ya que no hay m\'as aristas que $m$
    en $G$), podemos asegurar que la cantidad de iteraciones
    que har\'a el ciclo estar\'a acotada por $m$, y $m$ est\'a
    acotado por $n^2$.

\par Por lo tanto, sabiendo ya que el costo por ciclo es
    $\mathcal O(n^2)$, y sabieno que no tendremos m\'as
    de $n^2$ iteraciones, conclu\'imos que el costo del
    ciclo estar\'a acotado por $\mathcal O(n^4)$.

\par Al final de la heur\'istica, se devuelven la frontera
    ($\mathcal O(1)$ por ser un tipo primitivo del lenguaje)
    y la clique ($\mathcal O(1)$ por ser devuelta por
    referencia). Entonces, el coste final de todo el algoritmo
    heur\'istico recae sobre el ciclo y la inicializaci\'on
    mediante \emph{reserve}\footnote{\url{%
    http://www.cplusplus.com/reference/vector/vector/reserve/}} de
    $K$. Finalmente, el coste es $\mathcal O(n + n^4)$, que
    asint\'oticamente hablando es $\mathcal O(n^4)$.

\subsubsection{Complejidad de las Variantes\label{bl:compl:variantes}}
\par 

\bigskip

\begin{pseudocodigo}[Heur\'istica de B\'usqueda Local para \emph{CMF} con Intercambio - Complejidad]
    \Require Un grafo $G$ con $n$ v\'ertices numerados de $1$ a $n$ y $m$ aristas. El mismo
        cuenta con las siguientes estructuras de datos que lo modelan:
        \begin{itemize}
            \item Vectores de adyacencia: Dado un vertice $v$, $vecinos(v)$ nos da todos los
                nodos adyacentes a $v$ en $G$.

            \item Matriz de adyacencia: Dados los v\'ertices $v$ y $w$, $adyacentes(v,w)$ y
                $adyacentes(w,v)$ nos devuelven $true$ si y s\'olo si $v$ es adyacente
                a $w$ en $G$.

            \item Vector de nodos de $G$.
        \end{itemize}
    \Ensure\Statex
        \begin{itemize}
            \item Un vector $K$ correspondiente a la \emph{clique} de m\'axima frontera
                encontrada por la heur\'istica.

            \item El cardinal de $\delta(K)$, siendo $K$ la \emph{clique} del item anterior.
        \end{itemize}
    \Statex
    \State $K \gets \emptyset$ \Compl{Brown}{}{$n$}{}
    \If{$m = \frac{n(n-1)}{2}$} \Compl{Blue}{}{$1$}{}
        \State $K \gets \left\{1;\dots;\left\lfloor\sfrac{n}{2}\right\rfloor\right\}$ \Compl{Blue}{}{$n$}{}
        \State $\delta_{max} \gets \left\lfloor\sfrac{n}{2}\right\rfloor\cdot
            \left\lceil\sfrac{n}{2}\right\rceil$ \Compl{Blue}{}{$1$}{}
        \Statex
    \Else
        \State $K \gets$ Primer nodo de grado mayor o igual al grado promedio \Compl{Blue}{}{$n$}{}
        \State $\delta(K) \gets d(v)$ \Compl{Blue}{}{$1$}{}
        \If{$\exists v \in vecinos(v)$ tal que $d(v) > 2|K|$} \Compl{Red}{}{$n$}{}
            \State $v_{add} \gets v$ \Compl{Red}{}{$1$}{}
            \State $\delta(K+v_{add}) \gets \delta(K) + d(v_{add}) - 2|K|$ \Compl{Red}{}{$1$}{}
        \EndIf \Compl{Blue}{Costo del \emph{si}: }{$n$}{}
        \Statex
        \State $v_{rem} \gets top(K)$ \Compl{Blue}{}{$1$}{}
        \State $\delta(K-v_{rem}) \gets \delta(K)-d(v_{rem})+2(|K|-1)$ \Compl{Blue}{}{$1$}{}
        \Statex
        \State Busco $v_{exchIN}$ y $v_{exchOUT}$ tales que $\delta(K) < \delta(K-v_{exchOUT}+v_{exchIN})$ \Compl{Blue}{}{$n^3$}{}
        \State $\delta(K-v_{exchOUT}+v_{exchIN}) \gets \delta(K) - d(v_{exchOUT}) + d(v_{exchIN})$ \Compl{Blue}{}{$1$}{}
        \Statex
        \While{%
        $\begin{pmatrix}
            \text{\Huge{$\bigvee$}} &
            \begin{matrix}
                \delta(K) < \delta(K+v_{add})\\
                \delta(K) < \delta(K-v_{rem})\\
                \delta(K) < \delta(K-v_{exchOUT}+v_{exchIN})
            \end{matrix}
        \end{pmatrix}$%
        } \Compl{Red}{}{$1$}{}
            \Statex
            \If{$\delta(K+v_{add}) > \delta(K-v_{rem})$ $\land$ $\delta(K+v_{add}) > \delta(K-v_{exchOUT}+v_{exchIN})$} \Compl{Fuchsia}{}{$1$}{}
                \State $push\_heap(K,v_{add})$ \Compl{Fuchsia}{}{$log(n)$}{}
                \State $\delta(K) \gets \delta(K+v_{add})$ \Compl{Fuchsia}{}{$1$}{}
                \Statex
            \ElsIf{$\delta(K-v_{rem}) > \delta(K-v_{exchOUT}+v_{exchIN})$} \Compl{Fuchsia}{}{$1$}{}
                \State $pop\_heap(K)$ \Compl{Fuchsia}{}{$log(n)$}{}
                \State $\delta(K) \gets \delta(K-v_{rem})$ \Compl{Fuchsia}{}{$1$}{}
                \Statex
            \Else
                \State Intercambio en $K$ los nodos $v_{exchOUT}$ y $v_{exchIN}$ \Compl{Fuchsia}{}{$1$}{}
                \State Mantengo la estructura del \emph{heap} \Compl{Fuchsia}{}{$log(n)$}
                \State $\delta(K) \gets \delta(K) - d(v_{exchOUT}) + d(v_{exchIN})$ \Compl{Fuchsia}{}{$1$}{}
            \EndIf \Compl{Red}{Costo del \emph{si}: }{$log(n)$}{}
            \Statex
            \Statex $\triangleright$ Ya salte a un vecino, recorro la nueva vecindad
            \If{$\exists v \in candidatos(K)$ tal que $d(v) > 2|K|$} \Compl{Fuchsia}{}{$n^2$}{}
                \State $v_{add} \gets v$ \Compl{Fuchsia}{}{$1$}{}
                \State $\delta(K+v_{add}) \gets \delta(K) + d(v_{add}) - 2|K|$ \Compl{Fuchsia}{}{$1$}{}
            \EndIf \Compl{Red}{Costo del \emph{si}: }{$n^2$}{}
            \Statex
            \State $v_{rem} \gets top(K)$ \Compl{Red}{}{$1$}{}
            \State $\delta(K-v_{rem}) \gets \delta(K)-d(v_{rem})+2(|K|-1)$ \Compl{Red}{}{$1$}{}
            \Statex
            \State Busco $v_{exchIN}$ y $v_{exchOUT}$ tales que $\delta(K) < \delta(K-v_{exchOUT}+v_{exchIN})$ \Compl{Red}{}{$n^3$}{}
            \State $\delta(K-v_{exchOUT}+v_{exchIN}) \gets \delta(K) - d(v_{exchOUT}) + d(v_{exchIN})$ \Compl{Red}{}{$1$}{}
            \Statex
        \EndWhile \Compl{Blue}{Costo del \emph{mientras}: }{$n^2$}{veces $\times$ $\mathcal O(n^3 + n^2 + log(n)) = \mathcal O(n^5)$ }
    \EndIf \Compl{Brown}{Costo \emph{si}: }{$n^5 + n$}{$= \mathcal O(n^5)$}
    \State \Return{$\delta(K)$, $K$} \Compl{Brown}{}{$1$\label{bl:return}}{}
    \Statex
    \Statex \Compl{Brown}{Costo Total de la Heur\'istica: }{$n^5 + n$}{$= \mathcal O(n^5)$}
\end{pseudocodigo}

\bigskip


\subsection{Casos Particulares de Estudio}
    Empezaremos mencionando y caracterizando algunas familias de grafos
para las que nuestra heurist\'ica golosa constructiva siempre encuentra
un resultado correcto, esto es, una clique de m\'axima frontera.

Teniendo en cuenta, como ya se menciono anteriormente, que la heur\'istica
empieza por uno de los nodos de mayor grado y en cada paso va 
agregando nodos que forman clique con los nodos actualmente escogidos
bajo la condici\'on de que el grado de los mismos sea mayor a dos veces
el tama\~no de la clique actual (de estos candidatos, tambi\'en elige
uno de los de mayor grado). Tenemos lo siguiente:

En todos los casos en los que hay varios candidatos para agregar a la
clique con el mismo grado (en particular el primero) la heur\'istica
puede proporcionar el resultado incorrecto, dado que el algoritmo
elige determin\'isticamente uno de los candidatos mientras que la 
entrada es aleatoria (la idea es poder resolver el problema para
cualquier grafo de entrada), veremos a continuaci\'on que excepto
en casos muy particulares, grafos isomorfos cuyas matrices de adyacencia
son distintas pueden, al ser sometidos a la heur\'istica, arrojar
resultados muy dispares (en particular, resultados correctos, versus
resultados incorrectos).

\subsubsection{Grafos completos}
	En estos grafos como ya se mencion\'o anteriormente, el algoritmo
	escoge nodos y va haciendo crecer una clique hasta llegar a una 
	clique de tama\~no $n/2$, notes\'e que todos los nodos en los 
	grafos de esta familia tienen grado $d(i) = n - 1$, por lo
	tanto la selecci\'on en cada paso del siguiente nodo de la
	clique es dependiente de la implementaci\'on

	Tambi\'en anal\'iticamente puede verse que las CMF para esta 
	familia de grafos son tanto las $K_{\left \lfloor{n/2}\right \rfloor}$ 
	como las $K_{\left \lceil{n/2}\right \rceil}$. Habiendo establecido como
	cota superior al tama\~no de la CMF el valor $n/2$, la posibilidad de que
	los $K_{\left \lceil{n/2}\right \rceil}$ tambi\'en cumplan parece una 
	anomal\'ia a priori. Veremos que no lo es y que est\'a directamente
	relacionado con la naturaleza de la familia en estudio. 

	La frontera es la cantidad de aristas que sale de cada uno de los
	nodos de la clique hacia nodos que no est\'an en la misma. Siendo
	$n$ la cantidad de nodos del grafo y habiendo establecido que la CMF
	tiene tama~no $\left \lfloor{n/2}\right \rfloor$, 
	si la clique en estudio tiene este tama\~no, los nodos que quedan afuera
	de la clique son $\left \lceil{n/2} \right \rceil$, luego: 

	\( 
	\delta(K_{\left \lfloor{n/2}\right \rfloor}) = 
	\left \lfloor{n/2} \right \rfloor \times 
	(n - \left \lfloor{n/2} \right \rfloor ) = 
	\left \lfloor{n/2} \right \rfloor \times 
	(\left \lfloor{n/2} \right \rfloor +
	\left \lceil{n/2} \right \rceil - 
	\left \lfloor{n/2} \right \rfloor) =
	\left \lfloor{n/2} \right \rfloor \times
	\left \lceil{n/2} \right \rceil = 
	\delta(K_{\left \lceil{n/2}\right \rceil})
	\)
\begin{figure}[H]
\caption{Ejemplos grafos completos - Entrada / Salida}
\centering
%\includegraphics[scale = 0.5]{img/ej2/k2.png}
%\includegraphics[scale = 0.5]{img/ej2/k3.png}
\includegraphics[scale = 0.5]{img/ej2/k4.png}
\includegraphics[scale = 0.5]{img/ej2/k5.png}
\includegraphics[scale = 0.5]{img/ej2/k7.png}
\end{figure}

\subsection{\'Arboles}
La familia de los \'arboles es una de las familias en las que la heur\'istica
se puede romper y no devolver el resultado correcto. 

Para los \'arboles es pr\'acticamente inmediato que el 
tama\~no de la clique de m\'axima frontera es a lo sumo igual a dos, dado
que por su misma definici\'on los arboles no contienen circuitos simples.

Nuevamente en este caso se toma alguno de los nodos de mayor grado 
y se intenta hacer crecer la clique con alg\'un nodo adyacente.

Como puede verse en los siguientes ejemplos en el caso de que 
implementativamente se seleccione uno de los nodos de mayor grado
adyacente a otro de los nodos de mayor grado, la heur\'istica encuentra
efectivamente una CMF, pero en el caso de que el nodo de partida sea 
uno de los de mayor grado en el grafo y dicho nodo no pertenezca a
ninguna de las CMF del grafo, la heur\'istica reportar\'a un falso 
positivo (una clique cuya frontera es maximal pero no m\'axima).

Desde un punto de vista implementativo el desempate entre nodos de 
mismo grado, depender\'a de la matriz de adyacencia (la posici\'on en
la que aparece un nodo) y dado que existe un arbol isomorfo al del 
ejemplo en el que el nodo incorrecto para empezar la clique est\'a
listado antes que el nodo correcto (el que permite generar la CMF),
el siguiente es un buen ejemplo para ilustrar la fragilidad de la
heur\'istica

\begin{figure}[H]
\caption{Ejemplos \'Arboles}
\centering
\includegraphics[scale = 0.5]{img/ej2/tree_st0.png}
\includegraphics[scale = 0.5]{img/ej2/tree_st01.png}
\includegraphics[scale = 0.5]{img/ej2/tree_st02.png}
\includegraphics[scale = 0.5]{img/ej2/tree_st11.png}
\includegraphics[scale = 0.5]{img/ej2/tree_st12.png}
\end{figure}

\subsection{Grafos bipartitos}

\subsection{Grafos circulares}

\subsection{Estrellas}

\subsection{Grafo de secuencias}

\subsection{Banana tree}
	




La familia de gr\'afos que asegura un resultado incorrecto de nuestra 
heur\'istica golosa puede ser caracterizada como aquella en la que el
nodo de mayor grado no forma parte de la clique de m\'axima frontera.
Esto sucede ya que la heur\'istica construye una clique a partir del 
nodo de grado m\'aximo y en cada paso se mantiene el invariante de 
de clique, agregando solamente nodos cuyo grado sea mayor al tama\~no
de la clique en ese paso, por lo tanto si alguno de los nodos de la 
clique de m\'axima frontera no forma clique con el nodo 
inicial (el de mayor grado) entonces la heuristica constructiva no 
puede llegar a incluir a ese nodo.

Ejemplos:

%\begin{tikzpicture}
%	\SetGraphUnit{1.5cm}
%	\GraphInit[vstyle=Normal]
%
%	\Vertices{circle}{2,5,6,7,8}
%	\WE(2){1}
%	\Vertices{circle}{3,10,9,11}
%	\WE(3){2}
%
%	\foreach \v in {2,5,6,7,8}{\Edge(1)(\v)};
%
%\end{tikzpicture}
	
\begin{tikzpicture}
	\path 	
			% Nodos
			(0,8) node [shape=circle, draw] (1) {1}
			(1,8) node [shape=circle, draw] (2) {2}
			(2,8) node [shape=circle, draw] (3) {3}
			(6,8) node [shape=circle, draw] (4) {4}
			(7,8) node [shape=circle, draw] (5) {5}
			(8,8) node [shape=circle, draw] (6) {6}
			(0,7) node [shape=circle, draw] (7) {7}
			(1,7) node [shape=circle, draw] (8) {8}
			(7,7) node [shape=circle, draw] (9) {9}
			(8,7) node [shape=circle, draw] (10) {10}
			(0,6) node [shape=circle, draw] (11) {11}
			(1,6) node [shape=circle, draw] (12) {12}
			(7,6) node [shape=circle, draw] (13) {13}
			(8,6) node [shape=circle, draw] (14) {14}
			(4,5) node [shape=circle, draw] (15) {15}
			(2,4) node [shape=circle, draw] (16) {16}
			(3,4) node [shape=circle, draw] (17) {17}
			(5,4) node [shape=circle, draw] (18) {18}
			(6,4) node [shape=circle, draw] (19) {19}
			(2,3) node [shape=circle, draw] (20) {20}
			(3,3) node [shape=circle, draw] (21) {21}
			(4,3) node [shape=circle, draw] (22) {22}
			(5,3) node [shape=circle, draw] (23) {23}
			(6,3) node [shape=circle, draw] (24) {24}
			(2,2) node [shape=circle, draw] (25) {25}
			(6,2) node [shape=circle, draw] (26) {26}
			(4, 1.5) node [shape=circle, draw] (27) {27}
			(2,1) node [shape=circle, draw] (28) {28}
			(6,1) node [shape=circle, draw] (29) {29}
			(2,0) node [shape=circle, draw] (30) {30}
			(3,0) node [shape=circle, draw] (31) {31}
			(5,0) node [shape=circle, draw] (32) {32}
			(6,0) node [shape=circle, draw] (33) {33};
			% Aristas de la semiestrella superior izq
			\draw[-] (1) -- (8);
			\draw[-] (2) -- (8);
			\draw[-] (3) -- (8);
			\draw[-] (7) -- (8);
			\draw[-] (11) -- (8);
			\draw[-] (12) -- (8);

			% Aristas de la semiestrella superior izq
			\draw[-] (4) -- (9);
			\draw[-] (5) -- (9);
			\draw[-] (6) -- (9);
			\draw[-] (10) -- (9);
			\draw[-] (13) -- (9);
			\draw[-] (14) -- (9);

			% Mas aristas
			
			\draw[-] (8) -- (9);
			\draw[-] (8) -- (15);
			\draw[-] (9) -- (15);
			\draw[-] (16) -- (15);
			\draw[-] (17) -- (15);
			\draw[-] (18) -- (15);
			\draw[-] (19) -- (15);
			\draw[-] (22) -- (15);

			% Aristas de la estrella inferior

			\draw[-] (20) -- (27);
			\draw[-] (21) -- (27);
			\draw[-] (22) -- (27);
			\draw[-] (23) -- (27);
			\draw[-] (24) -- (27);
			\draw[-] (25) -- (27);
			\draw[-] (26) -- (27);
			\draw[-] (28) -- (27);
			\draw[-] (29) -- (27);
			\draw[-] (30) -- (27);
			\draw[-] (31) -- (27);
			\draw[-] (32) -- (27);
			\draw[-] (33) -- (27);

			
\end{tikzpicture}


\subsection{Experimentaci\'on}
    \subsubsection{Resultados}

%Doble estrella
\begin{figure}[H]
    \centering
    \fontsize{8}{10}\selectfont
    \resizebox{0.87\textwidth}{!}{\input{img/ej3/tabu_search/ej3_nodos_nlogn_star+bridge+double_star.tex}}
    \caption{Complejidad temporal para grafos Estrella+Puente+Doble Estrella (Variante Max\_Iter=nlog(n))}
\end{figure}

\begin{figure}[H]
    \centering
    \fontsize{8}{10}\selectfont
    \resizebox{0.87\textwidth}{!}{\input{img/ej3/tabu_search/ej3_nodos_n_star+bridge+double_star.tex}}
    \caption{Complejidad temporal para grafos Estrella+Puente+Doble Estrella (Variante Max\_Iter=n)}
\end{figure}

\begin{figure}[H]
    \centering
    \fontsize{8}{10}\selectfont
    \resizebox{0.87\textwidth}{!}{\input{img/ej3/tabu_search/ej3_frontera_star_bridge_double_star.tex}}
    \caption{Frontera de grafos Estrella+Puente+Doble Estrella}
\end{figure}

%Doble estrella sin aspiracion
\begin{figure}[H]
    \centering
    \fontsize{8}{10}\selectfont
    \resizebox{0.87\textwidth}{!}{\input{img/ej3/tabu_search/ej3_nodos_nlogn_star+bridge+double_star_sin_aspiracion.tex}}
    \caption{Complejidad temporal para grafos Estrella+Puente+Doble Estrella (Variante Max\_Iter=nlog(n),sin aspiraci\'on)}
\end{figure}

\begin{figure}[H]
    \centering
    \fontsize{8}{10}\selectfont
    \resizebox{0.87\textwidth}{!}{\input{img/ej3/tabu_search/ej3_nodos_n_star+bridge+double_star_sin_aspiracion.tex}}
    \caption{Complejidad temporal para grafos Estrella+Puente+Doble Estrella (Variante Max\_Iter=n),sin aspiraci\'on)}
\end{figure}

\begin{figure}[H]
    \centering
    \fontsize{8}{10}\selectfont
    \resizebox{0.87\textwidth}{!}{\input{img/ej3/tabu_search/ej3_frontera_star_bridge_double_star_sin_aspiracion.tex}}
    \caption{Frontera de grafos Estrella+Puente+Doble Estrella (sin aspiracion)}
\end{figure}

%Doble estrella sin aspiracion golosa
\begin{figure}[H]
    \centering
    \fontsize{8}{10}\selectfont
    \resizebox{0.87\textwidth}{!}{\input{img/ej3/tabu_search/ej3_nodos_nlogn_star+bridge+double_star_sin_aspiracion_golosa.tex}}
    \caption{Complejidad temporal para grafos Estrella+Puente+Doble Estrella (Variante Max\_Iter=nlog(n),sin aspiraci\'on,golosa)}
\end{figure}

\begin{figure}[H]
    \centering
    \fontsize{8}{10}\selectfont
    \resizebox{0.87\textwidth}{!}{\input{img/ej3/tabu_search/ej3_nodos_n_star+bridge+double_star_sin_aspiracion_golosa.tex}}
    \caption{Complejidad temporal para grafos Estrella+Puente+Doble Estrella (Variante Max\_Iter=n),sin aspiraci\'on,golosa)}
\end{figure}

\begin{figure}[H]
    \centering
    \fontsize{8}{10}\selectfont
    \resizebox{0.87\textwidth}{!}{\input{img/ej3/tabu_search/ej3_frontera_star_bridge_double_star_sin_aspiracion_golosa.tex}}
    \caption{Frontera de grafos Estrella+Puente+Doble Estrella (sin aspiracion, golosa)}
\end{figure}

%Bipartito Completo
\begin{figure}[H]
    \centering
    \fontsize{8}{10}\selectfont
    \resizebox{0.87\textwidth}{!}{\input{img/ej3/tabu_search/ej3_nodos_nlogn_complete_bipartite.tex}}
    \caption{Complejidad temporal para grafos Bipartito Completo (Variante Max\_Iter=nlog(n))}
\end{figure}

\begin{figure}[H]
    \centering
    \fontsize{8}{10}\selectfont
    \resizebox{0.87\textwidth}{!}{\input{img/ej3/tabu_search/ej3_nodos_n_complete_bipartite.tex}}
    \caption{Complejidad temporal para grafos Bipartito Completo (Variante Max\_Iter=n)}
\end{figure}

\begin{figure}[H]
    \centering
    \fontsize{8}{10}\selectfont
    \resizebox{0.87\textwidth}{!}{\input{img/ej3/tabu_search/ej3_frontera_complete_bipartite.tex}}
    \caption{Frontera de grafos Bipartito Completo}
\end{figure}

%Bipartito Completo sin aspiracion
\begin{figure}[H]
    \centering
    \fontsize{8}{10}\selectfont
    \resizebox{0.87\textwidth}{!}{\input{img/ej3/tabu_search/ej3_nodos_nlogn_complete_bipartite_sin_aspiracion.tex}}
    \caption{Complejidad temporal para grafos Bipartito Completo (Variante Max\_Iter=nlog(n),sin aspiraci\'on)}
\end{figure}

\begin{figure}[H]
    \centering
    \fontsize{8}{10}\selectfont
    \resizebox{0.87\textwidth}{!}{\input{img/ej3/tabu_search/ej3_nodos_n_complete_bipartite_sin_aspiracion.tex}}
    \caption{Complejidad temporal para grafos Bipartito Completo (Variante Max\_Iter=n),sin aspiraci\'on)}
\end{figure}

\begin{figure}[H]
    \centering
    \fontsize{8}{10}\selectfont
    \resizebox{0.87\textwidth}{!}{\input{img/ej3/tabu_search/ej3_frontera_complete_bipartite_sin_aspiracion.tex}}
    \caption{Frontera de grafos Bipartito Completo (sin aspiracion)}
\end{figure}

%Bipartito Completo sin aspiracion golosa
\begin{figure}[H]
    \centering
    \fontsize{8}{10}\selectfont
    \resizebox{0.87\textwidth}{!}{\input{img/ej3/tabu_search/ej3_nodos_nlogn_complete_bipartite_sin_aspiracion_golosa.tex}}
    \caption{Complejidad temporal para grafos Bipartito Completo (Variante Max\_Iter=nlog(n),sin aspiraci\'on,golosa)}
\end{figure}

\begin{figure}[H]
    \centering
    \fontsize{8}{10}\selectfont
    \resizebox{0.87\textwidth}{!}{\input{img/ej3/tabu_search/ej3_nodos_n_complete_bipartite_sin_aspiracion_golosa.tex}}
    \caption{Complejidad temporal para grafos Bipartito Completo (Variante Max\_Iter=n),sin aspiraci\'on,golosa)}
\end{figure}

\begin{figure}[H]
    \centering
    \fontsize{8}{10}\selectfont
    \resizebox{0.87\textwidth}{!}{% GNUPLOT: LaTeX picture with Postscript
\begingroup
  \makeatletter
  \providecommand\color[2][]{%
    \GenericError{(gnuplot) \space\space\space\@spaces}{%
      Package color not loaded in conjunction with
      terminal option `colourtext'%
    }{See the gnuplot documentation for explanation.%
    }{Either use 'blacktext' in gnuplot or load the package
      color.sty in LaTeX.}%
    \renewcommand\color[2][]{}%
  }%
  \providecommand\includegraphics[2][]{%
    \GenericError{(gnuplot) \space\space\space\@spaces}{%
      Package graphicx or graphics not loaded%
    }{See the gnuplot documentation for explanation.%
    }{The gnuplot epslatex terminal needs graphicx.sty or graphics.sty.}%
    \renewcommand\includegraphics[2][]{}%
  }%
  \providecommand\rotatebox[2]{#2}%
  \@ifundefined{ifGPcolor}{%
    \newif\ifGPcolor
    \GPcolortrue
  }{}%
  \@ifundefined{ifGPblacktext}{%
    \newif\ifGPblacktext
    \GPblacktexttrue
  }{}%
  % define a \g@addto@macro without @ in the name:
  \let\gplgaddtomacro\g@addto@macro
  % define empty templates for all commands taking text:
  \gdef\gplbacktext{}%
  \gdef\gplfronttext{}%
  \makeatother
  \ifGPblacktext
    % no textcolor at all
    \def\colorrgb#1{}%
    \def\colorgray#1{}%
  \else
    % gray or color?
    \ifGPcolor
      \def\colorrgb#1{\color[rgb]{#1}}%
      \def\colorgray#1{\color[gray]{#1}}%
      \expandafter\def\csname LTw\endcsname{\color{white}}%
      \expandafter\def\csname LTb\endcsname{\color{black}}%
      \expandafter\def\csname LTa\endcsname{\color{black}}%
      \expandafter\def\csname LT0\endcsname{\color[rgb]{1,0,0}}%
      \expandafter\def\csname LT1\endcsname{\color[rgb]{0,1,0}}%
      \expandafter\def\csname LT2\endcsname{\color[rgb]{0,0,1}}%
      \expandafter\def\csname LT3\endcsname{\color[rgb]{1,0,1}}%
      \expandafter\def\csname LT4\endcsname{\color[rgb]{0,1,1}}%
      \expandafter\def\csname LT5\endcsname{\color[rgb]{1,1,0}}%
      \expandafter\def\csname LT6\endcsname{\color[rgb]{0,0,0}}%
      \expandafter\def\csname LT7\endcsname{\color[rgb]{1,0.3,0}}%
      \expandafter\def\csname LT8\endcsname{\color[rgb]{0.5,0.5,0.5}}%
    \else
      % gray
      \def\colorrgb#1{\color{black}}%
      \def\colorgray#1{\color[gray]{#1}}%
      \expandafter\def\csname LTw\endcsname{\color{white}}%
      \expandafter\def\csname LTb\endcsname{\color{black}}%
      \expandafter\def\csname LTa\endcsname{\color{black}}%
      \expandafter\def\csname LT0\endcsname{\color{black}}%
      \expandafter\def\csname LT1\endcsname{\color{black}}%
      \expandafter\def\csname LT2\endcsname{\color{black}}%
      \expandafter\def\csname LT3\endcsname{\color{black}}%
      \expandafter\def\csname LT4\endcsname{\color{black}}%
      \expandafter\def\csname LT5\endcsname{\color{black}}%
      \expandafter\def\csname LT6\endcsname{\color{black}}%
      \expandafter\def\csname LT7\endcsname{\color{black}}%
      \expandafter\def\csname LT8\endcsname{\color{black}}%
    \fi
  \fi
  \setlength{\unitlength}{0.0500bp}%
  \begin{picture}(7200.00,5040.00)%
    \gplgaddtomacro\gplbacktext{%
      \csname LTb\endcsname%
      \put(1298,2904){\makebox(0,0)[r]{\strut{} 0}}%
      \csname LTb\endcsname%
      \put(1298,3052){\makebox(0,0)[r]{\strut{} 500}}%
      \csname LTb\endcsname%
      \put(1298,3199){\makebox(0,0)[r]{\strut{} 1000}}%
      \csname LTb\endcsname%
      \put(1298,3347){\makebox(0,0)[r]{\strut{} 1500}}%
      \csname LTb\endcsname%
      \put(1298,3494){\makebox(0,0)[r]{\strut{} 2000}}%
      \csname LTb\endcsname%
      \put(1298,3642){\makebox(0,0)[r]{\strut{} 2500}}%
      \csname LTb\endcsname%
      \put(1298,3789){\makebox(0,0)[r]{\strut{} 3000}}%
      \csname LTb\endcsname%
      \put(1298,3937){\makebox(0,0)[r]{\strut{} 3500}}%
      \csname LTb\endcsname%
      \put(1298,4084){\makebox(0,0)[r]{\strut{} 4000}}%
      \csname LTb\endcsname%
      \put(1298,4232){\makebox(0,0)[r]{\strut{} 4500}}%
      \csname LTb\endcsname%
      \put(1298,4379){\makebox(0,0)[r]{\strut{} 5000}}%
      \csname LTb\endcsname%
      \put(1430,2684){\makebox(0,0){\strut{} 0}}%
      \csname LTb\endcsname%
      \put(1967,2684){\makebox(0,0){\strut{} 500}}%
      \csname LTb\endcsname%
      \put(2505,2684){\makebox(0,0){\strut{} 1000}}%
      \csname LTb\endcsname%
      \put(3042,2684){\makebox(0,0){\strut{} 1500}}%
      \csname LTb\endcsname%
      \put(3579,2684){\makebox(0,0){\strut{} 2000}}%
      \csname LTb\endcsname%
      \put(4117,2684){\makebox(0,0){\strut{} 2500}}%
      \csname LTb\endcsname%
      \put(4654,2684){\makebox(0,0){\strut{} 3000}}%
      \csname LTb\endcsname%
      \put(5191,2684){\makebox(0,0){\strut{} 3500}}%
      \csname LTb\endcsname%
      \put(5728,2684){\makebox(0,0){\strut{} 4000}}%
      \csname LTb\endcsname%
      \put(6266,2684){\makebox(0,0){\strut{} 4500}}%
      \csname LTb\endcsname%
      \put(6803,2684){\makebox(0,0){\strut{} 5000}}%
      \put(176,3641){\rotatebox{-270}{\makebox(0,0){\strut{}Frontera}}}%
      \put(396,3641){\rotatebox{-270}{\makebox(0,0){\strut{}(Escala Lineal)}}}%
      \put(4116,2354){\makebox(0,0){\strut{}Cantidad de Nodos}}%
      \put(4116,2134){\makebox(0,0){\strut{}(Escala Lineal)}}%
      \put(4116,4709){\makebox(0,0){\strut{}Frontera obtenida segun cantidad de nodos}}%
    }%
    \gplgaddtomacro\gplfronttext{%
      \csname LTb\endcsname%
      \put(6461,1713){\makebox(0,0)[r]{\strut{}Iter=nlog(n),Sin Mejorar=n,Tiempo Tabu=n}}%
      \csname LTb\endcsname%
      \put(6461,1493){\makebox(0,0)[r]{\strut{}Iter=nlog(n),Sin Mejorar=n,Tiempo Tabu=n2}}%
      \csname LTb\endcsname%
      \put(6461,1273){\makebox(0,0)[r]{\strut{}Iter=nlog(n),Sin Mejorar=n2,Tiempo Tabu=n}}%
      \csname LTb\endcsname%
      \put(6461,1053){\makebox(0,0)[r]{\strut{}Iter=nlog(n),Sin Mejorar=n2,Tiempo Tabu=n2}}%
      \csname LTb\endcsname%
      \put(6461,833){\makebox(0,0)[r]{\strut{}Iter=n,Sin Mejorar=n,Tiempo Tabu=n}}%
      \csname LTb\endcsname%
      \put(6461,613){\makebox(0,0)[r]{\strut{}Iter=n,Sin Mejorar=n,Tiempo Tabu=n2}}%
      \csname LTb\endcsname%
      \put(6461,393){\makebox(0,0)[r]{\strut{}Iter=n,Sin Mejorar=n2,Tiempo Tabu=n}}%
      \csname LTb\endcsname%
      \put(6461,173){\makebox(0,0)[r]{\strut{}Iter=n,Sin Mejorar=n2,Tiempo Tabu=n2}}%
    }%
    \gplbacktext
    \put(0,0){\includegraphics{ej3_frontera_complete_bipartite_sin_aspiracion_golosa}}%
    \gplfronttext
  \end{picture}%
\endgroup
}
    \caption{Frontera de grafos Bipartito Completo (sin aspiracion,golosa)}
\end{figure}

\subsubsection{Conclusiones}


        \pagebreak

    \section{Metaheur\'istica de B\'usqueda Tab\'u}
        \subsection{Resoluci\'on}
    \begin{pseudocodigo}[Algoritmo Exacto para \emph{CMF} - Descriptivo]
    \Require\Statex
        \begin{itemize}
            \item Un grafo $G$ de $n$ v\'ertices\footnote{Asumimos, sin p\'erdida 
                de generalidad, que los v\'ertices est\'an numerados de 1 a $n$.}
                y $m$ aristas.

            \item Una funci\'on $candidatos(K)$, que dado un conjunto de v\'ertices
                $K$, devuelve una secuencia de v\'ertices de $G$ ordenada por grado
                de mayor a menor que son adjacentes a todos los elementos de $K$ y
                cuyo grado es mayor $2|K|$.

            \item Una funci\'on $\delta(K)$, que dada una \emph{clique} $K$\footnote{Una
                \emph{clique} no es otra cosa que un conjunto de v\'ertices, con la
                caracter\'istica de inducir un grafo completo en $G$.} en $G$, calcula
                el cardinal de su frontera\footnote{Como ya se explic\'o, sea $K$ una
                \emph{clique} de $G$, entonces: $\delta(K) = - |K|(|K|-1) +
                \displaystyle\sum_{v \in K} d(v)$}.

            \item Una funci\'on $\delta_{cota}(K)$, que dada una \emph{clique} $K$
                en $G$, calcula la cota de su frontera seg\'un el m\'etodo
                ya explicado anteriormente.

        \end{itemize}
    \Statex
    \Ensure Una \emph{clique} $K$ de $G$ con una frontera $\delta(K)$ de m\'axima
        cardinalidad.

    \Statex

    \If{$m = \dfrac{n(n-1)}{2}$} \Comment{Si $G$ es un $K_n$}
        \State $K \gets \left\{1;\dots;\left\lfloor\dfrac{n}{2}\right\rfloor\right\}$

    \Else
        \State $\delta_{max} \gets \left\lfloor\dfrac{r+1}{2}\right\rfloor\cdot
            \left\lceil\dfrac{r+1}{2}\right\rceil$ para $r$ tal que:
            $\begin{pmatrix}
                \text{\Huge$\bigwedge$} &
                    \begin{matrix}
                        r \in [1..n)\\[0.5cm]
                        r < \dfrac{n^2}{n^2 - 2m}\\[0.5cm]
                        (\forall r' \in [1..n))$ $r' < \dfrac{n^2}{n^2 - 2m} \implies r'\leq r
                    \end{matrix}
            \end{pmatrix}$\footnote{Esta cota inferior para la cardinalidad de la
                frontera maximal en $G$ se deriv\'o a partir del \emph{Teorema de Tur\'an},
                como ya se explic\'o anteriormente.}

        \State $K \gets \emptyset$, $K' \gets \emptyset$, $YaProcesados(K')$\footnote{%
            $YaProcesados(K')$ representar\'a a los v\'ertices de $candidatos(K')$ que
            ya fueron tenidos en cuenta para la \emph{clique} $K'$.}$ \gets \emptyset$

        \While{$candidatos(K')$\footnote{Si $K'$ es vac\'io, $candidatos(K')$ son todos
            los nodos de $G$.}$\setminus YaProcesados(K')$\footnote{Peque\~no abuso
            de notaci\'on, ya que $candidatos(K')$ es una secuencia y $YaProcesados(K')$
            es un conjunto. Se asume que es la secuencia sin los elementos del conjunto.}$
            \neq \emptyset$ $\lor$ $K' \neq \emptyset$}

            \If{$candidatos(K') \setminus YaProcesados(K') \neq \emptyset$ $\land$ $\delta_{max} < \delta_{cota}(K')$%
                \footnote{Si $K'$ es vac\'io, $\delta_{cota}(K')$ es $m$.}}

                \State $v \gets cabeza(candidatos(K') \setminus YaProcesados(K'))$
                \State $K' \gets K' \cup \{v'\}$
                \If{$candidatos(K') = \emptyset$ $\land$ $\delta_{max} < \delta(K')$}
                    %\Comment{Llegu\'e al final de la rama y obtuve una mejor solucion}
                    \State $\delta_{max} \gets \delta(K')$
                    \State $K \gets K'$
                \EndIf

            \ElsIf{$K' \neq \emptyset$} \Comment{\emph{Backtracking}}
                \State $v \gets ultimo(K')$\footnote{Donde $ultimo(K')$
                    es el \'ulitmo v\'ertice agregado a $K'$.}
                \State $K' \gets K' \setminus \{v'\}$
                \State $YaProcesados(K') \gets YaProcesados(K') \cup \{v\}$

            \Else
                \State $YaProcesados(K') \gets YaProcesados(K') \cup candidatos(K')$
            \EndIf
        \EndWhile
    \EndIf

    \State \Return{$K$}
\end{pseudocodigo}


\subsection{Complejidad}
    \subsubsection{Estructuras de Datos}
\par Como ya se mencion\'o en~\nameref{grafo:estructuras}, ya contamos en el grafo
    de entrada con listas/vectores de adyacencias y una matriz de adyacencias.

\par Al igual que en la heur\'istica golosa y el algoritmo exacto, decidimos
    modelar las cliques con vectores%
    \footnote{\url{http://www.cplusplus.com/reference/vector/vector/}}.

\par A su vez, para mejorar la eficiencia al recorrer la vecindad, se
    utilizan un vector/clique como \emph{min heaps} seg\'un el grado
    de los nodos, permitiendonos acceder en $\mathcal O(1)$ al nodo
    de menor grado. Los motivos de este proceder se detallan en
    la siguiente secci\'on.

\subsubsection{Pseudoc\'odigo de complejidad}
\par Se presenta a continuaci\'on un pseudoc\'odigo m\'as espec\'ifico de la implementaci\'on
    de este algoritmo provista junto con este trabajo. El mismo tiene en cuenta
    las estructuras de datos explicadas en el punto anterior.

\par Luego del pseudoc\'odigo se justifican detalladamente las complejidades
    expuestas a continuaci\'on que no sean evidentes\footnote{Consideramos
    como "complejidades evidentes" las asignaciones de variables, operaciones
    m\'atematicas simples, asignaciones/inicializaci\'on de posiciones de
    un vector/\emph{deque} o cualquier contenedor de acceso aleatorio/arbitrario}.

\par Debemos aclarar, debido a las variantes ya presentadas sobre esta
    heur\'istica, que el siguiente an\'alisis de complejidad aplica
    sobre la heur\'istica que selecciona al primer vecino v\'alido encontrado,
    en una vecindad donde s\'olo se consideran las cliques con un nodo m\'as
    o un nodo menos y que selecciona como clique inicial a alg\'un nodo
    de grado mayor o igual al grado promedio del grafo.

\par El an\'alisis de complejidad de las variantes se realiza en la secci\'on
    \ref{bl:compl:variantes}.

\bigskip

\begin{pseudocodigo}[Heur\'istica de B\'usqueda Local para \emph{CMF} - Complejidad]
    \Require Un grafo $G$ con $n$ v\'ertices numerados de $1$ a $n$ y $m$ aristas. El mismo
        cuenta con las siguientes estructuras de datos que lo modelan:
        \begin{itemize}
            \item Vectores de adyacencia: Dado un vertice $v$, $vecinos(v)$ nos da todos los
                nodos adyacentes a $v$ en $G$.

            \item Matriz de adyacencia: Dados los v\'ertices $v$ y $w$, $adyacentes(v,w)$ y
                $adyacentes(w,v)$ nos devuelven $true$ si y s\'olo si $v$ es adyacente
                a $w$ en $G$.

            \item Vector de nodos de $G$.
        \end{itemize}
    \Ensure\Statex
        \begin{itemize}
            \item Un vector $K$ correspondiente a la \emph{clique} de m\'axima frontera
                encontrada por la heur\'istica.

            \item El cardinal de $\delta(K)$, siendo $K$ la \emph{clique} del item anterior.
        \end{itemize}
    \Statex
    \State $K \gets \emptyset$ \Compl{Brown}{}{$n$}{}
    \If{$m = \frac{n(n-1)}{2}$} \Compl{Blue}{}{$1$}{}
        \State $K \gets \left\{1;\dots;\left\lfloor\sfrac{n}{2}\right\rfloor\right\}$ \Compl{Blue}{}{$n$}{}
        \State $\delta_{max} \gets \left\lfloor\sfrac{n}{2}\right\rfloor\cdot
            \left\lceil\sfrac{n}{2}\right\rceil$ \Compl{Blue}{}{$1$}{}
        \Statex
    \Else
        \State $K \gets$ Primer nodo de grado mayor o igual al grado promedio \Compl{Blue}{}{$n$}{}
        \State $\delta(K) \gets d(v)$ \Compl{Blue}{}{$1$}{}
        \If{$\exists v \in vecinos(v)$ tal que $d(v) > 2|K|$} \Compl{Red}{}{$n$}{}
            \State $v_{add} \gets v$ \Compl{Red}{}{$1$}{}
            \State $\delta(K+v_{add}) \gets \delta(K) + d(v_{add}) - 2|K|$ \Compl{Red}{}{$1$}{}
        \EndIf \Compl{Blue}{Costo del \emph{si}: }{$n$}{}
        \Statex
        \State $v_{rem} \gets top(K)$ \Compl{Blue}{}{$1$}{}
        \State $\delta(K-v_{rem}) \gets \delta(K)-d(v_{rem})+2(|K|-1)$ \Compl{Blue}{}{$1$}{}
        \Statex
        \While{$\delta(K) < \delta(K+v_{add})$ $\lor$ $\delta(K) < \delta(K-v_{rem})$} \Compl{Red}{}{$1$}{}
            \If{$\delta(K+v_{add}) > \delta(K-v_{rem})$} \Compl{Fuchsia}{}{$1$}{}
                \State $push\_heap(K,v_{add})$ \Compl{Fuchsia}{}{$log(n)$}{}
                \State $\delta(K) \gets \delta(K+v_{add})$ \Compl{Fuchsia}{}{$1$}{}
                \Statex
            \Else
                \State $pop\_heap(K)$ \Compl{Fuchsia}{}{$log(n)$}{}
                \State $\delta(K) \gets \delta(K-v_{rem})$ \Compl{Fuchsia}{}{$1$}{}
                \Statex
            \EndIf \Compl{Red}{Costo del \emph{si}: }{$log(n)$}{}
            \Statex
            \Statex $\triangleright$ Ya salte a un vecino, recorro la nueva vecindad
            \If{$\exists v \in candidatos(K)$ tal que $d(v) > 2|K|$} \Compl{Fuchsia}{}{$n^2$}{}
                \State $v_{add} \gets v$ \Compl{Fuchsia}{}{$1$}{}
                \State $\delta(K+v_{add}) \gets \delta(K) + d(v_{add}) - 2|K|$ \Compl{Fuchsia}{}{$1$}{}
            \EndIf \Compl{Red}{Costo del \emph{si}: }{$n^2$}{}
            \Statex
            \State $v_{rem} \gets top(K)$ \Compl{Red}{}{$1$}{}
            \State $\delta(K-v_{rem}) \gets \delta(K)-d(v_{rem})+2(|K|-1)$ \Compl{Red}{}{$1$}{}
            \Statex
        \EndWhile \Compl{Blue}{Costo del \emph{mientras}: }{$n^2$}{veces $\times$ $\mathcal O(n^2 + log(n)) = \mathcal O(n^4)$ }
    \EndIf \Compl{Brown}{Costo \emph{si}: }{$n^4 + n$}{$= \mathcal O(n^4)$}
    \State \Return{$\delta(K)$, $K$} \Compl{Brown}{}{$1$\label{bl:return}}{}
    \Statex
    \Statex \Compl{Brown}{Costo Total de la Heur\'istica: }{$n^4 +n$}{$= \mathcal O(n^4)$}
\end{pseudocodigo}

\bigskip

\par La primera parte del algoritmo es exactamente igual a la de la
    heur\'istica golosa y el algoritmo exacto: reconoce en $\mathcal O(1)$
    si el grafo $G$ es un $K_n$ y en caso afirmativo, devuelve
    la \emph{CMF} con costo lineal. Esto ya fue justificado en
    los dos casos previos, as\'i pues se omite su justificaci\'on.

\par Luego, en caso de estar ante un $G$ que no es un grafo completo,
    el algoritmo busca a un nodo que tenga grado mayor o igual al
    promedio. Esta b\'usqueda se realiza sobre el vector de nodos
    (los cuales permiten el acceso a su grado en $\mathcal O(1)$),
    con lo cual el costo en el peor caso pasa a ser revisar
    todos los nodos del grafo\footnote{\url{http://www.cplusplus.com/reference/algorithm/find_if/}},
    siendo la complejidad entonces $\mathcal O(n)$. Habiendo
    escogido el nodo/clique inicial, se guarda la frontera
    parcial correspondiente (que no es otra cosa que el grado
    de dicho nodo, que como se dijo, es accedible en tiempo
    constante) y se asigna dicho nodo al vector que representa
    la clique (lo cual tiene un costo constante tambien\footnote{%
    \url{http://www.cplusplus.com/reference/vector/vector/push_back/}}.

\par Luego, antes de comenzar el ciclo, se inicializan los posibles
    valores de la vecindad. En s\'i mismo, se recorrer la vecindad
    \emph{inicial}, guardando las primeras cliques vecinas (una
    con un nodo m\'as y otra con un nodo menos\footnote{Esta \'ultima
    clique es trivial ya que en este momento la clique se compone
    de s\'olo un nodo.}) que incrementen la funci\'on objetivo.

\par Buscar una clique que incremente la frontera es lineal ya
    que basta con recorrer el \emph{deque} de vecinos del
    \'unico nodo de la clique (el cual es accesible en tiempo
    constante gracias a el vector de nodos de la estructura del
    grafo y a la propia estructura de estos nodos\footnote{Ver
    \emph{\nameref{grafo:estructuras}}, en la secci\'on
    \emph{\nameref{notas_preliminares}}.} y realizar una
    \'unica comparacion de tiempo constante (verificar que
    efectivamente su grado incrementa la frontera).

\par Encontrado este nodo, calcular cuanto incrementa la funci\'on
    es trivial y se realiza en tiempo constante, ya que alcanza
    con hacer las operaciones matem\'aticas sobre la frontera
    de la clique actual y el grado del nodo que se a\~nadir\'ia%
    \footnote{Ver la Secci\'on \emph{\nameref{notas:calc_front}},
    en \emph{\nameref{notas_preliminares}}.}.

\par Luego, para encontrar un nodo que incremente la frontera
    si fuese eliminado de la clique se hace aprovechando
    las estructuras de un \emph{min heap} sobre los grados
    de la clique, utilizando la representaci\'on de este
    que no es otra que un vector \footnote{\url{%
    http://www.cplusplus.com/reference/algorithm/make_heap/}}.
    Si bien el heap en este caso no fue inicializado mediante
    la funci\'on \emph{make\_heap}, esto no es necesario ya
    que en esta variante, s\'olo se tiene un nodo en la clique.
    Luego, si mantenemos esta estructura en el vector, las operaciones
    $push\_heap$ y $pop_heap$ tendr\'an un coste logar\'itmico
    mientras que $top$ sera de coste constante\footnote{%
    \url{http://www.cplusplus.com/reference/algorithm/push_heap/},%
    \url{http://www.cplusplus.com/reference/algorithm/pop_heap/},%
    \url{http://www.cplusplus.com/reference/vector/vector/front/},%
    \url{http://www.cplusplus.com/reference/vector/vector/pop_back/},%
    \url{http://www.cplusplus.com/reference/vector/vector/push_back/}}.

\par Teniendo ya un \emph{min heap}, podemos acceder al nodo
    de menor grado de la clique actual en $\mathcal O(1)$

\par Nuevamente, calcular la frontera si se quitase dicho
    nodo es trivial. Pero ac\'a debemos justificar el hecho
    de no buscar alg\'un otro nodo que al ser quitado incremente
    la frontera en caso de que este nodo (el tope del heap) no
    lo haga. El fundamento aqu\'i es que el tope del heap
    es el nodo de menor grado, por lo tanto, es el nodo
    que menos frontera le a\~nade a $K$, por lo tanto,
    quitar cualquier otro nodo implicar\'ia quitar un nodo
    de mayor grado, lo que significa que estar\'iamos
    considerando quitar un nodo que aporta m\'as frontera
    que el tope del heap y por lo tanto estariamos
    obteniendo una frontera menor que la que se obtiene
    de quitar el tope. Esto nos permite asegurar que si
    se tiene el nodo de menor grado de la clique, y este
    no incrementa la frontera si se quitase de la clique,
    entonces ning\'un otro nodo de la clique lo har\'a.

\par Luego, comienza el ciclo de la b\'usqueda local. En la
    guarda del mismo se realizan s\'olo comparaciones
    de enteros que son almacenados en variables de la implementaci\'on
    de esta funci\'on de \emph{C++}, por lo cual su acceso
    tiene un coste $\mathcal O(1)$.

\par Inmediatamente al comenzar el ciclo hay un condicional
    \emph{si}. El mismo en su guarda realiza una operaci\'on
    $\mathcal O(1)$ por los mismos motivos explicados en
    el p\'arrafo anterior. Dentro de sus dos ramas vemos
    que se realizan 1 asignaci\'on ($\mathcal O(1)$) y
    un $push\_heap$ o $pop\_heap$, que tienen un coste
    logar\'itmico como ya se explico m\'as arriba.

\par En la segunda mitad del ciclo, se
    revisa la nueva vecindad (en este punto, el ciclo
    habr\'a a\~nadido o quitado alg\'un nodo de la
    clique, por lo cual tenemos una nueva vecindad). En
    el caso de los vecinos que se componen de cliques con
    un nodo menos, el proceso es id\'entico al realizado
    antes de entrar al ciclo, con lo cual no hace falta
    justificar nada m\'as. En cuanto a las cliques que
    contienen un nodo m\'as, ahora tenemos una clique
    con (posiblemente) m\'as de un nodo, por lo cual,
    debemos ver todos sus candidatos, pero no nos podemos
    basar en los nodos de la clique antes de realizar
    el salto al vecino pues si se quito un nodo, los
    candidatos podr\'ian incrementarse. Por lo tanto,
    debemos recorrer (en el peor de los casos) todos
    los nodos adyacentes a alg\'un nodo de la clique
    (que son accesibles en $\mathcal O(1)$ gracias al vector
    de nodos de la estructura de $G$) y verificar 2
    cosas: que sean adyacentes a todos los dem\'as
    nodos de la clique y que incrementen la frontera%
    \footnote{Es decir, que $d(v) > 2|K|$.}. En caso
    de encontrar uno, podemos detener este recorrido de posibles
    nodos candidatos, ya que no estamos buscando saltar
    al mejor vecino sino al primero que encontremos
    que cumpla con nuestros requerimientos. Pero a\'un as\'i,
    en el peor de los casos deberemos recorrerlos a todos,
    es decir, $\mathcal O(n)$ nodos adyacentes, y tambi\'en
    como cota para $|K|$ debemos considerar que podr\'ia
    estar en el orden de $\mathcal O(n)$ elementos, siendo
    el costo de esta b\'usqueda de vecino candidato
    con un nodo m\'as, igual a $\mathcal O(n^2)$.

\par Por \'ultimo, ya teniendo calculado el coste del ciclo,
    debemos acotar la cantidad de iteraciones que tendr\'a.
    Esto es d\'ificil, ya que sin mayor informaci\'on sobre
    el grafo de entrada y su estructura, no se puede saber
    a ciencia cierta cuantas veces podr\'a "deplazarse" el
    algoritmo en la vecindad definida. Por lo tanto, se
    decidi\'o acotar mediante lo m\'inimo que crece la frontera
    entre iteraci\'on e iteraci\'on. Este m\'inimo es, simplemente,
    1. Considerando este peor caso, donde la funci\'on objetivo
    crece muy lentamente, y considerando que la frontera m\'axima
    no puede ser mayor a $m$ (ya que no hay m\'as aristas que $m$
    en $G$), podemos asegurar que la cantidad de iteraciones
    que har\'a el ciclo estar\'a acotada por $m$, y $m$ est\'a
    acotado por $n^2$.

\par Por lo tanto, sabiendo ya que el costo por ciclo es
    $\mathcal O(n^2)$, y sabieno que no tendremos m\'as
    de $n^2$ iteraciones, conclu\'imos que el costo del
    ciclo estar\'a acotado por $\mathcal O(n^4)$.

\par Al final de la heur\'istica, se devuelven la frontera
    ($\mathcal O(1)$ por ser un tipo primitivo del lenguaje)
    y la clique ($\mathcal O(1)$ por ser devuelta por
    referencia). Entonces, el coste final de todo el algoritmo
    heur\'istico recae sobre el ciclo y la inicializaci\'on
    mediante \emph{reserve}\footnote{\url{%
    http://www.cplusplus.com/reference/vector/vector/reserve/}} de
    $K$. Finalmente, el coste es $\mathcal O(n + n^4)$, que
    asint\'oticamente hablando es $\mathcal O(n^4)$.

\subsubsection{Complejidad de las Variantes\label{bl:compl:variantes}}
\par 

\bigskip

\begin{pseudocodigo}[Heur\'istica de B\'usqueda Local para \emph{CMF} con Intercambio - Complejidad]
    \Require Un grafo $G$ con $n$ v\'ertices numerados de $1$ a $n$ y $m$ aristas. El mismo
        cuenta con las siguientes estructuras de datos que lo modelan:
        \begin{itemize}
            \item Vectores de adyacencia: Dado un vertice $v$, $vecinos(v)$ nos da todos los
                nodos adyacentes a $v$ en $G$.

            \item Matriz de adyacencia: Dados los v\'ertices $v$ y $w$, $adyacentes(v,w)$ y
                $adyacentes(w,v)$ nos devuelven $true$ si y s\'olo si $v$ es adyacente
                a $w$ en $G$.

            \item Vector de nodos de $G$.
        \end{itemize}
    \Ensure\Statex
        \begin{itemize}
            \item Un vector $K$ correspondiente a la \emph{clique} de m\'axima frontera
                encontrada por la heur\'istica.

            \item El cardinal de $\delta(K)$, siendo $K$ la \emph{clique} del item anterior.
        \end{itemize}
    \Statex
    \State $K \gets \emptyset$ \Compl{Brown}{}{$n$}{}
    \If{$m = \frac{n(n-1)}{2}$} \Compl{Blue}{}{$1$}{}
        \State $K \gets \left\{1;\dots;\left\lfloor\sfrac{n}{2}\right\rfloor\right\}$ \Compl{Blue}{}{$n$}{}
        \State $\delta_{max} \gets \left\lfloor\sfrac{n}{2}\right\rfloor\cdot
            \left\lceil\sfrac{n}{2}\right\rceil$ \Compl{Blue}{}{$1$}{}
        \Statex
    \Else
        \State $K \gets$ Primer nodo de grado mayor o igual al grado promedio \Compl{Blue}{}{$n$}{}
        \State $\delta(K) \gets d(v)$ \Compl{Blue}{}{$1$}{}
        \If{$\exists v \in vecinos(v)$ tal que $d(v) > 2|K|$} \Compl{Red}{}{$n$}{}
            \State $v_{add} \gets v$ \Compl{Red}{}{$1$}{}
            \State $\delta(K+v_{add}) \gets \delta(K) + d(v_{add}) - 2|K|$ \Compl{Red}{}{$1$}{}
        \EndIf \Compl{Blue}{Costo del \emph{si}: }{$n$}{}
        \Statex
        \State $v_{rem} \gets top(K)$ \Compl{Blue}{}{$1$}{}
        \State $\delta(K-v_{rem}) \gets \delta(K)-d(v_{rem})+2(|K|-1)$ \Compl{Blue}{}{$1$}{}
        \Statex
        \State Busco $v_{exchIN}$ y $v_{exchOUT}$ tales que $\delta(K) < \delta(K-v_{exchOUT}+v_{exchIN})$ \Compl{Blue}{}{$n^3$}{}
        \State $\delta(K-v_{exchOUT}+v_{exchIN}) \gets \delta(K) - d(v_{exchOUT}) + d(v_{exchIN})$ \Compl{Blue}{}{$1$}{}
        \Statex
        \While{%
        $\begin{pmatrix}
            \text{\Huge{$\bigvee$}} &
            \begin{matrix}
                \delta(K) < \delta(K+v_{add})\\
                \delta(K) < \delta(K-v_{rem})\\
                \delta(K) < \delta(K-v_{exchOUT}+v_{exchIN})
            \end{matrix}
        \end{pmatrix}$%
        } \Compl{Red}{}{$1$}{}
            \Statex
            \If{$\delta(K+v_{add}) > \delta(K-v_{rem})$ $\land$ $\delta(K+v_{add}) > \delta(K-v_{exchOUT}+v_{exchIN})$} \Compl{Fuchsia}{}{$1$}{}
                \State $push\_heap(K,v_{add})$ \Compl{Fuchsia}{}{$log(n)$}{}
                \State $\delta(K) \gets \delta(K+v_{add})$ \Compl{Fuchsia}{}{$1$}{}
                \Statex
            \ElsIf{$\delta(K-v_{rem}) > \delta(K-v_{exchOUT}+v_{exchIN})$} \Compl{Fuchsia}{}{$1$}{}
                \State $pop\_heap(K)$ \Compl{Fuchsia}{}{$log(n)$}{}
                \State $\delta(K) \gets \delta(K-v_{rem})$ \Compl{Fuchsia}{}{$1$}{}
                \Statex
            \Else
                \State Intercambio en $K$ los nodos $v_{exchOUT}$ y $v_{exchIN}$ \Compl{Fuchsia}{}{$1$}{}
                \State Mantengo la estructura del \emph{heap} \Compl{Fuchsia}{}{$log(n)$}
                \State $\delta(K) \gets \delta(K) - d(v_{exchOUT}) + d(v_{exchIN})$ \Compl{Fuchsia}{}{$1$}{}
            \EndIf \Compl{Red}{Costo del \emph{si}: }{$log(n)$}{}
            \Statex
            \Statex $\triangleright$ Ya salte a un vecino, recorro la nueva vecindad
            \If{$\exists v \in candidatos(K)$ tal que $d(v) > 2|K|$} \Compl{Fuchsia}{}{$n^2$}{}
                \State $v_{add} \gets v$ \Compl{Fuchsia}{}{$1$}{}
                \State $\delta(K+v_{add}) \gets \delta(K) + d(v_{add}) - 2|K|$ \Compl{Fuchsia}{}{$1$}{}
            \EndIf \Compl{Red}{Costo del \emph{si}: }{$n^2$}{}
            \Statex
            \State $v_{rem} \gets top(K)$ \Compl{Red}{}{$1$}{}
            \State $\delta(K-v_{rem}) \gets \delta(K)-d(v_{rem})+2(|K|-1)$ \Compl{Red}{}{$1$}{}
            \Statex
            \State Busco $v_{exchIN}$ y $v_{exchOUT}$ tales que $\delta(K) < \delta(K-v_{exchOUT}+v_{exchIN})$ \Compl{Red}{}{$n^3$}{}
            \State $\delta(K-v_{exchOUT}+v_{exchIN}) \gets \delta(K) - d(v_{exchOUT}) + d(v_{exchIN})$ \Compl{Red}{}{$1$}{}
            \Statex
        \EndWhile \Compl{Blue}{Costo del \emph{mientras}: }{$n^2$}{veces $\times$ $\mathcal O(n^3 + n^2 + log(n)) = \mathcal O(n^5)$ }
    \EndIf \Compl{Brown}{Costo \emph{si}: }{$n^5 + n$}{$= \mathcal O(n^5)$}
    \State \Return{$\delta(K)$, $K$} \Compl{Brown}{}{$1$\label{bl:return}}{}
    \Statex
    \Statex \Compl{Brown}{Costo Total de la Heur\'istica: }{$n^5 + n$}{$= \mathcal O(n^5)$}
\end{pseudocodigo}

\bigskip


\subsection{Casos Particulares de Estudio}
    Empezaremos mencionando y caracterizando algunas familias de grafos
para las que nuestra heurist\'ica golosa constructiva siempre encuentra
un resultado correcto, esto es, una clique de m\'axima frontera.

Teniendo en cuenta, como ya se menciono anteriormente, que la heur\'istica
empieza por uno de los nodos de mayor grado y en cada paso va 
agregando nodos que forman clique con los nodos actualmente escogidos
bajo la condici\'on de que el grado de los mismos sea mayor a dos veces
el tama\~no de la clique actual (de estos candidatos, tambi\'en elige
uno de los de mayor grado). Tenemos lo siguiente:

En todos los casos en los que hay varios candidatos para agregar a la
clique con el mismo grado (en particular el primero) la heur\'istica
puede proporcionar el resultado incorrecto, dado que el algoritmo
elige determin\'isticamente uno de los candidatos mientras que la 
entrada es aleatoria (la idea es poder resolver el problema para
cualquier grafo de entrada), veremos a continuaci\'on que excepto
en casos muy particulares, grafos isomorfos cuyas matrices de adyacencia
son distintas pueden, al ser sometidos a la heur\'istica, arrojar
resultados muy dispares (en particular, resultados correctos, versus
resultados incorrectos).

\subsubsection{Grafos completos}
	En estos grafos como ya se mencion\'o anteriormente, el algoritmo
	escoge nodos y va haciendo crecer una clique hasta llegar a una 
	clique de tama\~no $n/2$, notes\'e que todos los nodos en los 
	grafos de esta familia tienen grado $d(i) = n - 1$, por lo
	tanto la selecci\'on en cada paso del siguiente nodo de la
	clique es dependiente de la implementaci\'on

	Tambi\'en anal\'iticamente puede verse que las CMF para esta 
	familia de grafos son tanto las $K_{\left \lfloor{n/2}\right \rfloor}$ 
	como las $K_{\left \lceil{n/2}\right \rceil}$. Habiendo establecido como
	cota superior al tama\~no de la CMF el valor $n/2$, la posibilidad de que
	los $K_{\left \lceil{n/2}\right \rceil}$ tambi\'en cumplan parece una 
	anomal\'ia a priori. Veremos que no lo es y que est\'a directamente
	relacionado con la naturaleza de la familia en estudio. 

	La frontera es la cantidad de aristas que sale de cada uno de los
	nodos de la clique hacia nodos que no est\'an en la misma. Siendo
	$n$ la cantidad de nodos del grafo y habiendo establecido que la CMF
	tiene tama~no $\left \lfloor{n/2}\right \rfloor$, 
	si la clique en estudio tiene este tama\~no, los nodos que quedan afuera
	de la clique son $\left \lceil{n/2} \right \rceil$, luego: 

	\( 
	\delta(K_{\left \lfloor{n/2}\right \rfloor}) = 
	\left \lfloor{n/2} \right \rfloor \times 
	(n - \left \lfloor{n/2} \right \rfloor ) = 
	\left \lfloor{n/2} \right \rfloor \times 
	(\left \lfloor{n/2} \right \rfloor +
	\left \lceil{n/2} \right \rceil - 
	\left \lfloor{n/2} \right \rfloor) =
	\left \lfloor{n/2} \right \rfloor \times
	\left \lceil{n/2} \right \rceil = 
	\delta(K_{\left \lceil{n/2}\right \rceil})
	\)
\begin{figure}[H]
\caption{Ejemplos grafos completos - Entrada / Salida}
\centering
%\includegraphics[scale = 0.5]{img/ej2/k2.png}
%\includegraphics[scale = 0.5]{img/ej2/k3.png}
\includegraphics[scale = 0.5]{img/ej2/k4.png}
\includegraphics[scale = 0.5]{img/ej2/k5.png}
\includegraphics[scale = 0.5]{img/ej2/k7.png}
\end{figure}

\subsection{\'Arboles}
La familia de los \'arboles es una de las familias en las que la heur\'istica
se puede romper y no devolver el resultado correcto. 

Para los \'arboles es pr\'acticamente inmediato que el 
tama\~no de la clique de m\'axima frontera es a lo sumo igual a dos, dado
que por su misma definici\'on los arboles no contienen circuitos simples.

Nuevamente en este caso se toma alguno de los nodos de mayor grado 
y se intenta hacer crecer la clique con alg\'un nodo adyacente.

Como puede verse en los siguientes ejemplos en el caso de que 
implementativamente se seleccione uno de los nodos de mayor grado
adyacente a otro de los nodos de mayor grado, la heur\'istica encuentra
efectivamente una CMF, pero en el caso de que el nodo de partida sea 
uno de los de mayor grado en el grafo y dicho nodo no pertenezca a
ninguna de las CMF del grafo, la heur\'istica reportar\'a un falso 
positivo (una clique cuya frontera es maximal pero no m\'axima).

Desde un punto de vista implementativo el desempate entre nodos de 
mismo grado, depender\'a de la matriz de adyacencia (la posici\'on en
la que aparece un nodo) y dado que existe un arbol isomorfo al del 
ejemplo en el que el nodo incorrecto para empezar la clique est\'a
listado antes que el nodo correcto (el que permite generar la CMF),
el siguiente es un buen ejemplo para ilustrar la fragilidad de la
heur\'istica

\begin{figure}[H]
\caption{Ejemplos \'Arboles}
\centering
\includegraphics[scale = 0.5]{img/ej2/tree_st0.png}
\includegraphics[scale = 0.5]{img/ej2/tree_st01.png}
\includegraphics[scale = 0.5]{img/ej2/tree_st02.png}
\includegraphics[scale = 0.5]{img/ej2/tree_st11.png}
\includegraphics[scale = 0.5]{img/ej2/tree_st12.png}
\end{figure}

\subsection{Grafos bipartitos}

\subsection{Grafos circulares}

\subsection{Estrellas}

\subsection{Grafo de secuencias}

\subsection{Banana tree}
	




La familia de gr\'afos que asegura un resultado incorrecto de nuestra 
heur\'istica golosa puede ser caracterizada como aquella en la que el
nodo de mayor grado no forma parte de la clique de m\'axima frontera.
Esto sucede ya que la heur\'istica construye una clique a partir del 
nodo de grado m\'aximo y en cada paso se mantiene el invariante de 
de clique, agregando solamente nodos cuyo grado sea mayor al tama\~no
de la clique en ese paso, por lo tanto si alguno de los nodos de la 
clique de m\'axima frontera no forma clique con el nodo 
inicial (el de mayor grado) entonces la heuristica constructiva no 
puede llegar a incluir a ese nodo.

Ejemplos:

%\begin{tikzpicture}
%	\SetGraphUnit{1.5cm}
%	\GraphInit[vstyle=Normal]
%
%	\Vertices{circle}{2,5,6,7,8}
%	\WE(2){1}
%	\Vertices{circle}{3,10,9,11}
%	\WE(3){2}
%
%	\foreach \v in {2,5,6,7,8}{\Edge(1)(\v)};
%
%\end{tikzpicture}
	
\begin{tikzpicture}
	\path 	
			% Nodos
			(0,8) node [shape=circle, draw] (1) {1}
			(1,8) node [shape=circle, draw] (2) {2}
			(2,8) node [shape=circle, draw] (3) {3}
			(6,8) node [shape=circle, draw] (4) {4}
			(7,8) node [shape=circle, draw] (5) {5}
			(8,8) node [shape=circle, draw] (6) {6}
			(0,7) node [shape=circle, draw] (7) {7}
			(1,7) node [shape=circle, draw] (8) {8}
			(7,7) node [shape=circle, draw] (9) {9}
			(8,7) node [shape=circle, draw] (10) {10}
			(0,6) node [shape=circle, draw] (11) {11}
			(1,6) node [shape=circle, draw] (12) {12}
			(7,6) node [shape=circle, draw] (13) {13}
			(8,6) node [shape=circle, draw] (14) {14}
			(4,5) node [shape=circle, draw] (15) {15}
			(2,4) node [shape=circle, draw] (16) {16}
			(3,4) node [shape=circle, draw] (17) {17}
			(5,4) node [shape=circle, draw] (18) {18}
			(6,4) node [shape=circle, draw] (19) {19}
			(2,3) node [shape=circle, draw] (20) {20}
			(3,3) node [shape=circle, draw] (21) {21}
			(4,3) node [shape=circle, draw] (22) {22}
			(5,3) node [shape=circle, draw] (23) {23}
			(6,3) node [shape=circle, draw] (24) {24}
			(2,2) node [shape=circle, draw] (25) {25}
			(6,2) node [shape=circle, draw] (26) {26}
			(4, 1.5) node [shape=circle, draw] (27) {27}
			(2,1) node [shape=circle, draw] (28) {28}
			(6,1) node [shape=circle, draw] (29) {29}
			(2,0) node [shape=circle, draw] (30) {30}
			(3,0) node [shape=circle, draw] (31) {31}
			(5,0) node [shape=circle, draw] (32) {32}
			(6,0) node [shape=circle, draw] (33) {33};
			% Aristas de la semiestrella superior izq
			\draw[-] (1) -- (8);
			\draw[-] (2) -- (8);
			\draw[-] (3) -- (8);
			\draw[-] (7) -- (8);
			\draw[-] (11) -- (8);
			\draw[-] (12) -- (8);

			% Aristas de la semiestrella superior izq
			\draw[-] (4) -- (9);
			\draw[-] (5) -- (9);
			\draw[-] (6) -- (9);
			\draw[-] (10) -- (9);
			\draw[-] (13) -- (9);
			\draw[-] (14) -- (9);

			% Mas aristas
			
			\draw[-] (8) -- (9);
			\draw[-] (8) -- (15);
			\draw[-] (9) -- (15);
			\draw[-] (16) -- (15);
			\draw[-] (17) -- (15);
			\draw[-] (18) -- (15);
			\draw[-] (19) -- (15);
			\draw[-] (22) -- (15);

			% Aristas de la estrella inferior

			\draw[-] (20) -- (27);
			\draw[-] (21) -- (27);
			\draw[-] (22) -- (27);
			\draw[-] (23) -- (27);
			\draw[-] (24) -- (27);
			\draw[-] (25) -- (27);
			\draw[-] (26) -- (27);
			\draw[-] (28) -- (27);
			\draw[-] (29) -- (27);
			\draw[-] (30) -- (27);
			\draw[-] (31) -- (27);
			\draw[-] (32) -- (27);
			\draw[-] (33) -- (27);

			
\end{tikzpicture}


\subsection{Experimentaci\'on\label{tabu:experimentacion}}
    \subsubsection{Resultados}

%Doble estrella
\begin{figure}[H]
    \centering
    \fontsize{8}{10}\selectfont
    \resizebox{0.87\textwidth}{!}{\input{img/ej3/tabu_search/ej3_nodos_nlogn_star+bridge+double_star.tex}}
    \caption{Complejidad temporal para grafos Estrella+Puente+Doble Estrella (Variante Max\_Iter=nlog(n))}
\end{figure}

\begin{figure}[H]
    \centering
    \fontsize{8}{10}\selectfont
    \resizebox{0.87\textwidth}{!}{\input{img/ej3/tabu_search/ej3_nodos_n_star+bridge+double_star.tex}}
    \caption{Complejidad temporal para grafos Estrella+Puente+Doble Estrella (Variante Max\_Iter=n)}
\end{figure}

\begin{figure}[H]
    \centering
    \fontsize{8}{10}\selectfont
    \resizebox{0.87\textwidth}{!}{\input{img/ej3/tabu_search/ej3_frontera_star_bridge_double_star.tex}}
    \caption{Frontera de grafos Estrella+Puente+Doble Estrella}
\end{figure}

%Doble estrella sin aspiracion
\begin{figure}[H]
    \centering
    \fontsize{8}{10}\selectfont
    \resizebox{0.87\textwidth}{!}{\input{img/ej3/tabu_search/ej3_nodos_nlogn_star+bridge+double_star_sin_aspiracion.tex}}
    \caption{Complejidad temporal para grafos Estrella+Puente+Doble Estrella (Variante Max\_Iter=nlog(n),sin aspiraci\'on)}
\end{figure}

\begin{figure}[H]
    \centering
    \fontsize{8}{10}\selectfont
    \resizebox{0.87\textwidth}{!}{\input{img/ej3/tabu_search/ej3_nodos_n_star+bridge+double_star_sin_aspiracion.tex}}
    \caption{Complejidad temporal para grafos Estrella+Puente+Doble Estrella (Variante Max\_Iter=n),sin aspiraci\'on)}
\end{figure}

\begin{figure}[H]
    \centering
    \fontsize{8}{10}\selectfont
    \resizebox{0.87\textwidth}{!}{\input{img/ej3/tabu_search/ej3_frontera_star_bridge_double_star_sin_aspiracion.tex}}
    \caption{Frontera de grafos Estrella+Puente+Doble Estrella (sin aspiracion)}
\end{figure}

%Doble estrella sin aspiracion golosa
\begin{figure}[H]
    \centering
    \fontsize{8}{10}\selectfont
    \resizebox{0.87\textwidth}{!}{\input{img/ej3/tabu_search/ej3_nodos_nlogn_star+bridge+double_star_sin_aspiracion_golosa.tex}}
    \caption{Complejidad temporal para grafos Estrella+Puente+Doble Estrella (Variante Max\_Iter=nlog(n),sin aspiraci\'on,golosa)}
\end{figure}

\begin{figure}[H]
    \centering
    \fontsize{8}{10}\selectfont
    \resizebox{0.87\textwidth}{!}{\input{img/ej3/tabu_search/ej3_nodos_n_star+bridge+double_star_sin_aspiracion_golosa.tex}}
    \caption{Complejidad temporal para grafos Estrella+Puente+Doble Estrella (Variante Max\_Iter=n),sin aspiraci\'on,golosa)}
\end{figure}

\begin{figure}[H]
    \centering
    \fontsize{8}{10}\selectfont
    \resizebox{0.87\textwidth}{!}{\input{img/ej3/tabu_search/ej3_frontera_star_bridge_double_star_sin_aspiracion_golosa.tex}}
    \caption{Frontera de grafos Estrella+Puente+Doble Estrella (sin aspiracion, golosa)}
\end{figure}

%Bipartito Completo
\begin{figure}[H]
    \centering
    \fontsize{8}{10}\selectfont
    \resizebox{0.87\textwidth}{!}{\input{img/ej3/tabu_search/ej3_nodos_nlogn_complete_bipartite.tex}}
    \caption{Complejidad temporal para grafos Bipartito Completo (Variante Max\_Iter=nlog(n))}
\end{figure}

\begin{figure}[H]
    \centering
    \fontsize{8}{10}\selectfont
    \resizebox{0.87\textwidth}{!}{\input{img/ej3/tabu_search/ej3_nodos_n_complete_bipartite.tex}}
    \caption{Complejidad temporal para grafos Bipartito Completo (Variante Max\_Iter=n)}
\end{figure}

\begin{figure}[H]
    \centering
    \fontsize{8}{10}\selectfont
    \resizebox{0.87\textwidth}{!}{\input{img/ej3/tabu_search/ej3_frontera_complete_bipartite.tex}}
    \caption{Frontera de grafos Bipartito Completo}
\end{figure}

%Bipartito Completo sin aspiracion
\begin{figure}[H]
    \centering
    \fontsize{8}{10}\selectfont
    \resizebox{0.87\textwidth}{!}{\input{img/ej3/tabu_search/ej3_nodos_nlogn_complete_bipartite_sin_aspiracion.tex}}
    \caption{Complejidad temporal para grafos Bipartito Completo (Variante Max\_Iter=nlog(n),sin aspiraci\'on)}
\end{figure}

\begin{figure}[H]
    \centering
    \fontsize{8}{10}\selectfont
    \resizebox{0.87\textwidth}{!}{\input{img/ej3/tabu_search/ej3_nodos_n_complete_bipartite_sin_aspiracion.tex}}
    \caption{Complejidad temporal para grafos Bipartito Completo (Variante Max\_Iter=n),sin aspiraci\'on)}
\end{figure}

\begin{figure}[H]
    \centering
    \fontsize{8}{10}\selectfont
    \resizebox{0.87\textwidth}{!}{\input{img/ej3/tabu_search/ej3_frontera_complete_bipartite_sin_aspiracion.tex}}
    \caption{Frontera de grafos Bipartito Completo (sin aspiracion)}
\end{figure}

%Bipartito Completo sin aspiracion golosa
\begin{figure}[H]
    \centering
    \fontsize{8}{10}\selectfont
    \resizebox{0.87\textwidth}{!}{\input{img/ej3/tabu_search/ej3_nodos_nlogn_complete_bipartite_sin_aspiracion_golosa.tex}}
    \caption{Complejidad temporal para grafos Bipartito Completo (Variante Max\_Iter=nlog(n),sin aspiraci\'on,golosa)}
\end{figure}

\begin{figure}[H]
    \centering
    \fontsize{8}{10}\selectfont
    \resizebox{0.87\textwidth}{!}{\input{img/ej3/tabu_search/ej3_nodos_n_complete_bipartite_sin_aspiracion_golosa.tex}}
    \caption{Complejidad temporal para grafos Bipartito Completo (Variante Max\_Iter=n),sin aspiraci\'on,golosa)}
\end{figure}

\begin{figure}[H]
    \centering
    \fontsize{8}{10}\selectfont
    \resizebox{0.87\textwidth}{!}{% GNUPLOT: LaTeX picture with Postscript
\begingroup
  \makeatletter
  \providecommand\color[2][]{%
    \GenericError{(gnuplot) \space\space\space\@spaces}{%
      Package color not loaded in conjunction with
      terminal option `colourtext'%
    }{See the gnuplot documentation for explanation.%
    }{Either use 'blacktext' in gnuplot or load the package
      color.sty in LaTeX.}%
    \renewcommand\color[2][]{}%
  }%
  \providecommand\includegraphics[2][]{%
    \GenericError{(gnuplot) \space\space\space\@spaces}{%
      Package graphicx or graphics not loaded%
    }{See the gnuplot documentation for explanation.%
    }{The gnuplot epslatex terminal needs graphicx.sty or graphics.sty.}%
    \renewcommand\includegraphics[2][]{}%
  }%
  \providecommand\rotatebox[2]{#2}%
  \@ifundefined{ifGPcolor}{%
    \newif\ifGPcolor
    \GPcolortrue
  }{}%
  \@ifundefined{ifGPblacktext}{%
    \newif\ifGPblacktext
    \GPblacktexttrue
  }{}%
  % define a \g@addto@macro without @ in the name:
  \let\gplgaddtomacro\g@addto@macro
  % define empty templates for all commands taking text:
  \gdef\gplbacktext{}%
  \gdef\gplfronttext{}%
  \makeatother
  \ifGPblacktext
    % no textcolor at all
    \def\colorrgb#1{}%
    \def\colorgray#1{}%
  \else
    % gray or color?
    \ifGPcolor
      \def\colorrgb#1{\color[rgb]{#1}}%
      \def\colorgray#1{\color[gray]{#1}}%
      \expandafter\def\csname LTw\endcsname{\color{white}}%
      \expandafter\def\csname LTb\endcsname{\color{black}}%
      \expandafter\def\csname LTa\endcsname{\color{black}}%
      \expandafter\def\csname LT0\endcsname{\color[rgb]{1,0,0}}%
      \expandafter\def\csname LT1\endcsname{\color[rgb]{0,1,0}}%
      \expandafter\def\csname LT2\endcsname{\color[rgb]{0,0,1}}%
      \expandafter\def\csname LT3\endcsname{\color[rgb]{1,0,1}}%
      \expandafter\def\csname LT4\endcsname{\color[rgb]{0,1,1}}%
      \expandafter\def\csname LT5\endcsname{\color[rgb]{1,1,0}}%
      \expandafter\def\csname LT6\endcsname{\color[rgb]{0,0,0}}%
      \expandafter\def\csname LT7\endcsname{\color[rgb]{1,0.3,0}}%
      \expandafter\def\csname LT8\endcsname{\color[rgb]{0.5,0.5,0.5}}%
    \else
      % gray
      \def\colorrgb#1{\color{black}}%
      \def\colorgray#1{\color[gray]{#1}}%
      \expandafter\def\csname LTw\endcsname{\color{white}}%
      \expandafter\def\csname LTb\endcsname{\color{black}}%
      \expandafter\def\csname LTa\endcsname{\color{black}}%
      \expandafter\def\csname LT0\endcsname{\color{black}}%
      \expandafter\def\csname LT1\endcsname{\color{black}}%
      \expandafter\def\csname LT2\endcsname{\color{black}}%
      \expandafter\def\csname LT3\endcsname{\color{black}}%
      \expandafter\def\csname LT4\endcsname{\color{black}}%
      \expandafter\def\csname LT5\endcsname{\color{black}}%
      \expandafter\def\csname LT6\endcsname{\color{black}}%
      \expandafter\def\csname LT7\endcsname{\color{black}}%
      \expandafter\def\csname LT8\endcsname{\color{black}}%
    \fi
  \fi
  \setlength{\unitlength}{0.0500bp}%
  \begin{picture}(7200.00,5040.00)%
    \gplgaddtomacro\gplbacktext{%
      \csname LTb\endcsname%
      \put(1298,2904){\makebox(0,0)[r]{\strut{} 0}}%
      \csname LTb\endcsname%
      \put(1298,3052){\makebox(0,0)[r]{\strut{} 500}}%
      \csname LTb\endcsname%
      \put(1298,3199){\makebox(0,0)[r]{\strut{} 1000}}%
      \csname LTb\endcsname%
      \put(1298,3347){\makebox(0,0)[r]{\strut{} 1500}}%
      \csname LTb\endcsname%
      \put(1298,3494){\makebox(0,0)[r]{\strut{} 2000}}%
      \csname LTb\endcsname%
      \put(1298,3642){\makebox(0,0)[r]{\strut{} 2500}}%
      \csname LTb\endcsname%
      \put(1298,3789){\makebox(0,0)[r]{\strut{} 3000}}%
      \csname LTb\endcsname%
      \put(1298,3937){\makebox(0,0)[r]{\strut{} 3500}}%
      \csname LTb\endcsname%
      \put(1298,4084){\makebox(0,0)[r]{\strut{} 4000}}%
      \csname LTb\endcsname%
      \put(1298,4232){\makebox(0,0)[r]{\strut{} 4500}}%
      \csname LTb\endcsname%
      \put(1298,4379){\makebox(0,0)[r]{\strut{} 5000}}%
      \csname LTb\endcsname%
      \put(1430,2684){\makebox(0,0){\strut{} 0}}%
      \csname LTb\endcsname%
      \put(1967,2684){\makebox(0,0){\strut{} 500}}%
      \csname LTb\endcsname%
      \put(2505,2684){\makebox(0,0){\strut{} 1000}}%
      \csname LTb\endcsname%
      \put(3042,2684){\makebox(0,0){\strut{} 1500}}%
      \csname LTb\endcsname%
      \put(3579,2684){\makebox(0,0){\strut{} 2000}}%
      \csname LTb\endcsname%
      \put(4117,2684){\makebox(0,0){\strut{} 2500}}%
      \csname LTb\endcsname%
      \put(4654,2684){\makebox(0,0){\strut{} 3000}}%
      \csname LTb\endcsname%
      \put(5191,2684){\makebox(0,0){\strut{} 3500}}%
      \csname LTb\endcsname%
      \put(5728,2684){\makebox(0,0){\strut{} 4000}}%
      \csname LTb\endcsname%
      \put(6266,2684){\makebox(0,0){\strut{} 4500}}%
      \csname LTb\endcsname%
      \put(6803,2684){\makebox(0,0){\strut{} 5000}}%
      \put(176,3641){\rotatebox{-270}{\makebox(0,0){\strut{}Frontera}}}%
      \put(396,3641){\rotatebox{-270}{\makebox(0,0){\strut{}(Escala Lineal)}}}%
      \put(4116,2354){\makebox(0,0){\strut{}Cantidad de Nodos}}%
      \put(4116,2134){\makebox(0,0){\strut{}(Escala Lineal)}}%
      \put(4116,4709){\makebox(0,0){\strut{}Frontera obtenida segun cantidad de nodos}}%
    }%
    \gplgaddtomacro\gplfronttext{%
      \csname LTb\endcsname%
      \put(6461,1713){\makebox(0,0)[r]{\strut{}Iter=nlog(n),Sin Mejorar=n,Tiempo Tabu=n}}%
      \csname LTb\endcsname%
      \put(6461,1493){\makebox(0,0)[r]{\strut{}Iter=nlog(n),Sin Mejorar=n,Tiempo Tabu=n2}}%
      \csname LTb\endcsname%
      \put(6461,1273){\makebox(0,0)[r]{\strut{}Iter=nlog(n),Sin Mejorar=n2,Tiempo Tabu=n}}%
      \csname LTb\endcsname%
      \put(6461,1053){\makebox(0,0)[r]{\strut{}Iter=nlog(n),Sin Mejorar=n2,Tiempo Tabu=n2}}%
      \csname LTb\endcsname%
      \put(6461,833){\makebox(0,0)[r]{\strut{}Iter=n,Sin Mejorar=n,Tiempo Tabu=n}}%
      \csname LTb\endcsname%
      \put(6461,613){\makebox(0,0)[r]{\strut{}Iter=n,Sin Mejorar=n,Tiempo Tabu=n2}}%
      \csname LTb\endcsname%
      \put(6461,393){\makebox(0,0)[r]{\strut{}Iter=n,Sin Mejorar=n2,Tiempo Tabu=n}}%
      \csname LTb\endcsname%
      \put(6461,173){\makebox(0,0)[r]{\strut{}Iter=n,Sin Mejorar=n2,Tiempo Tabu=n2}}%
    }%
    \gplbacktext
    \put(0,0){\includegraphics{ej3_frontera_complete_bipartite_sin_aspiracion_golosa}}%
    \gplfronttext
  \end{picture}%
\endgroup
}
    \caption{Frontera de grafos Bipartito Completo (sin aspiracion,golosa)}
\end{figure}

\subsubsection{Conclusiones}


        \pagebreak

    \section{Experimentaci\'on Computacional}
        \subsection{Resultados}
\par Llegados a el \'ultimo punto de este trabajo, ya se encuentran implementadas
    y experimentadas todas las heur\'isticas (y algorimto exacto) pedidas.

\par Ahora deseamos compara las heur\'isticas entre s\'i y llegar a nuevas
    conclusiones que no pudieron ser eval\'uadas al experimentar con cada
    una de ellas por separado.

\par De las mucha familias que se han presentado, hay muchas que tiene intersecciones
    nos vac\'ias (es decir, instancias en com\'un) y otras tantas que directamente
    son subfamilas de otras.

\par As\'i pues, sin necesidad de dar demasiados detalles al respecto, se
    puede afirmar que las \emph{Estrellas}, \emph{Banana Trees},
    \emph{Estrella+Puente+Doble Estrella} son subfamilias de los \emph{\'Arboles}.

\par A su vez, las estructuras internas de las familias \emph{Rueda}, \emph{Circular},
    son muy similiares a la de las \emph{Estrellas}, al menos para el contexto
    de las experimentaciones que fueron hechas y del problema sobre el que se est\'a
    trabajando. A su vez, estas 2 familias son de cierta ``estructura simple'' para
    la resoluci\'on de nuestro problema (al punto que el algoritmo exacto va a ser
    muy eficiente para instancias grandes de estos, a pesar de ser exponencial).

\par Por \'ultimo, y teniendo en cuenta lo que se dijo en \emph{\nameref{notas_preliminares}},
    como estamos trabajando sobre grafos conexos todos los grafos est\'an
    contenidos en las familias de grafos por densidad.

\par As\'i pues, con este panorama, decidimos elegir ciertas variantes de las
    heur\'isticas implementadas y realizar una experimentaci\'on de estas sobre
    los mismos conjuntos de instancias de ciertas familias. Como el problema
    no est\'a planteado para alg\'un tipo de grafo en com\'un, no se sabe
    si se desea obtener resultados particulares sobre ciertas familias, con
    lo cual decidimos utilizar para esta experimentaci\'on global a las familias
    de grafos por densidad (que incluyen a todas las dem\'as familias, sin limitarnos
    a ninguna estructura otra que ser conexas) y los \emph{\'Arboles}, estos \'ultimos
    por contener familias para los cuales se estuvo experimentando y a partir de
    cuyos resultados se eligen las variantes de las heur\'isticas que usaremos
    a continuaci\'on.

\par Sobre las variantes elegidas para esta experimentaci\'on, contamos con:

\begin{itemize}
    \item \textbf{El Algoritmo Exacto} Dicho algoritmo nos permitir\'a
        verificar cuan buenas son las soluciones dadas por las heur\'isticas,
        al menos para instancias chicas (pero que no son nada f\'aciles de
        resolver \emph{a mano}). Desde ya que est\'a descontado tratar
        de compara los tiempos de ejecuci\'on de las heur\'isticas con este
        algoritmo, requiriendo este evidentemente y sin demasiada necesidad de
        justificaci\'on, mucho m\'as tiempo para resolver los problemas.

    \item \textbf{Algoritmo Goloso} Siendo esta una heur\'istica para la
        cual no se plantearon variantes, pero habi\'endose usado como
        entrada de las otras dos, nos pareci\'o razonable utilizarla para
        verificar si las variantes que la utilizan como clique incial
        la mejoran o no.

    \item \textbf{B\'usqueda Local del mejor vecino con intercambio}, por
        lo explayado en la secci\'on de experimentaci\'on de esta
        heur\'istica, y sin tener ning\'un par\'ametro externo a utilizar
        para determinar cuando una heur\'istica es mejor que la otra (no
        sabemos el tiempo que se est\'a dispuesto a esperar por una respuesta
        ni cuan precisa debe ser esta), seleccionamos esta variante
        por ser una de las mejores para el \emph{tandem} tiempos de ejecuci\'on-%
        resultado devuelto.

    \item \textbf{B\'usqueda Tab\'u, \emph{MAX\_ITER = n, SIN\_MEJORAR = n,
        TIEMPO\_TABU = n/2}}, esta heur\'istica no recib\'e una entrada
        de la heur\'istica golosa y utiliza una funci\'on de aspiraci\'on.
        La seleccionamos por los mismos motivos que seleccionamos a la
        variante de b\'usqueda local.

    \item \textbf{B\'usqueda Tab\'u, \emph{MAX\_ITER = n, SIN\_MEJORAR = n,
        TIEMPO\_TABU = n/2}}, esta heur\'istica si recib\'e una entrada
        por parte de la heur\'istica golosa y no utiliza una funci\'on
        de aspiraci\'on. Fue seleccionada por los mismos motivoos
        que la otra variante de b\'usqueda tab\'u.
\end{itemize}

\subsubsection{\'Arboles}
%Tree
\begin{figure}[H]
    \centering
    \fontsize{7}{10}\selectfont
    \resizebox{0.80\textwidth}{!}{\input{img/ej4/ej4_nodos_tree.tex}}
    \caption{Complejidad temporal para \'Arboles}
\end{figure}

\begin{figure}[H]
    \centering
    \fontsize{7}{10}\selectfont
    \resizebox{0.80\textwidth}{!}{% GNUPLOT: LaTeX picture with Postscript
\begingroup
  \makeatletter
  \providecommand\color[2][]{%
    \GenericError{(gnuplot) \space\space\space\@spaces}{%
      Package color not loaded in conjunction with
      terminal option `colourtext'%
    }{See the gnuplot documentation for explanation.%
    }{Either use 'blacktext' in gnuplot or load the package
      color.sty in LaTeX.}%
    \renewcommand\color[2][]{}%
  }%
  \providecommand\includegraphics[2][]{%
    \GenericError{(gnuplot) \space\space\space\@spaces}{%
      Package graphicx or graphics not loaded%
    }{See the gnuplot documentation for explanation.%
    }{The gnuplot epslatex terminal needs graphicx.sty or graphics.sty.}%
    \renewcommand\includegraphics[2][]{}%
  }%
  \providecommand\rotatebox[2]{#2}%
  \@ifundefined{ifGPcolor}{%
    \newif\ifGPcolor
    \GPcolortrue
  }{}%
  \@ifundefined{ifGPblacktext}{%
    \newif\ifGPblacktext
    \GPblacktexttrue
  }{}%
  % define a \g@addto@macro without @ in the name:
  \let\gplgaddtomacro\g@addto@macro
  % define empty templates for all commands taking text:
  \gdef\gplbacktext{}%
  \gdef\gplfronttext{}%
  \makeatother
  \ifGPblacktext
    % no textcolor at all
    \def\colorrgb#1{}%
    \def\colorgray#1{}%
  \else
    % gray or color?
    \ifGPcolor
      \def\colorrgb#1{\color[rgb]{#1}}%
      \def\colorgray#1{\color[gray]{#1}}%
      \expandafter\def\csname LTw\endcsname{\color{white}}%
      \expandafter\def\csname LTb\endcsname{\color{black}}%
      \expandafter\def\csname LTa\endcsname{\color{black}}%
      \expandafter\def\csname LT0\endcsname{\color[rgb]{1,0,0}}%
      \expandafter\def\csname LT1\endcsname{\color[rgb]{0,1,0}}%
      \expandafter\def\csname LT2\endcsname{\color[rgb]{0,0,1}}%
      \expandafter\def\csname LT3\endcsname{\color[rgb]{1,0,1}}%
      \expandafter\def\csname LT4\endcsname{\color[rgb]{0,1,1}}%
      \expandafter\def\csname LT5\endcsname{\color[rgb]{1,1,0}}%
      \expandafter\def\csname LT6\endcsname{\color[rgb]{0,0,0}}%
      \expandafter\def\csname LT7\endcsname{\color[rgb]{1,0.3,0}}%
      \expandafter\def\csname LT8\endcsname{\color[rgb]{0.5,0.5,0.5}}%
    \else
      % gray
      \def\colorrgb#1{\color{black}}%
      \def\colorgray#1{\color[gray]{#1}}%
      \expandafter\def\csname LTw\endcsname{\color{white}}%
      \expandafter\def\csname LTb\endcsname{\color{black}}%
      \expandafter\def\csname LTa\endcsname{\color{black}}%
      \expandafter\def\csname LT0\endcsname{\color{black}}%
      \expandafter\def\csname LT1\endcsname{\color{black}}%
      \expandafter\def\csname LT2\endcsname{\color{black}}%
      \expandafter\def\csname LT3\endcsname{\color{black}}%
      \expandafter\def\csname LT4\endcsname{\color{black}}%
      \expandafter\def\csname LT5\endcsname{\color{black}}%
      \expandafter\def\csname LT6\endcsname{\color{black}}%
      \expandafter\def\csname LT7\endcsname{\color{black}}%
      \expandafter\def\csname LT8\endcsname{\color{black}}%
    \fi
  \fi
  \setlength{\unitlength}{0.0500bp}%
  \begin{picture}(7200.00,5040.00)%
    \gplgaddtomacro\gplbacktext{%
      \csname LTb\endcsname%
      \put(1034,2244){\makebox(0,0)[r]{\strut{} 0}}%
      \csname LTb\endcsname%
      \put(1034,2671){\makebox(0,0)[r]{\strut{} 5}}%
      \csname LTb\endcsname%
      \put(1034,3098){\makebox(0,0)[r]{\strut{} 10}}%
      \csname LTb\endcsname%
      \put(1034,3525){\makebox(0,0)[r]{\strut{} 15}}%
      \csname LTb\endcsname%
      \put(1034,3952){\makebox(0,0)[r]{\strut{} 20}}%
      \csname LTb\endcsname%
      \put(1034,4379){\makebox(0,0)[r]{\strut{} 25}}%
      \csname LTb\endcsname%
      \put(1166,2024){\makebox(0,0){\strut{} 1}}%
      \csname LTb\endcsname%
      \put(2575,2024){\makebox(0,0){\strut{} 10}}%
      \csname LTb\endcsname%
      \put(3985,2024){\makebox(0,0){\strut{} 100}}%
      \csname LTb\endcsname%
      \put(5394,2024){\makebox(0,0){\strut{} 1000}}%
      \csname LTb\endcsname%
      \put(6803,2024){\makebox(0,0){\strut{} 10000}}%
      \put(176,3311){\rotatebox{-270}{\makebox(0,0){\strut{}Frontera}}}%
      \put(396,3311){\rotatebox{-270}{\makebox(0,0){\strut{}(Escala Lineal)}}}%
      \put(3984,1694){\makebox(0,0){\strut{}Cantidad de Nodos}}%
      \put(3984,1474){\makebox(0,0){\strut{}(Escala Logaritmica)}}%
      \put(3984,4709){\makebox(0,0){\strut{}Frontera obtenida segun cantidad de nodos}}%
    }%
    \gplgaddtomacro\gplfronttext{%
      \csname LTb\endcsname%
      \put(5999,1053){\makebox(0,0)[r]{\strut{}Algoritmo Exacto}}%
      \csname LTb\endcsname%
      \put(5999,833){\makebox(0,0)[r]{\strut{}Tabu(n,n,n/2) (golosa,sin aspiracion)}}%
      \csname LTb\endcsname%
      \put(5999,613){\makebox(0,0)[r]{\strut{}Tabu(n,n,n/2)}}%
      \csname LTb\endcsname%
      \put(5999,393){\makebox(0,0)[r]{\strut{}BL (mejor vecino,intercambio)}}%
      \csname LTb\endcsname%
      \put(5999,173){\makebox(0,0)[r]{\strut{}Golosa}}%
    }%
    \gplbacktext
    \put(0,0){\includegraphics{ej4_frontera_tree}}%
    \gplfronttext
  \end{picture}%
\endgroup
}
    \caption{Frontera para \'Arboles}
\end{figure}

\bigskip

\par Para la familia de los grafos el primer resultado evidente que podemos
    indicar es que la golosa es la mejor opci\'on sin lugar a dudas.

\par Mientras que el resto de las heur\'isticas an\'alizadas oscila en un
    coste temporal de entre 100 a 100000 mil microsegundos de diferencia
    con la golosa (la cual se va incrementando a medida que se incrementa
    el tama\~no de la entrada), el resultado provisto por la golosa
    es id\'entico al ex\'acto o est\'a muy cerca de cerlo.

\par Sin lugar a dudas, en el caso de trabajar con \'arboles (y a pesar
    de la familia $Estrella+Puente+Doble Estrella$, que es un caso
    particular de la familia $Estrella+Puente+CMF$ (sobre la cual
    esta heur\'istica no funciona bien), est\'a heur\'istica
    tiene (para los datos que manejamos de requerimientos a la hora
    de resolver el problema) el mejor balance tiempo-resultado.

\subsubsection{Grafos poco densos}
%Connected Sparse
\begin{figure}[H]
    \centering
    \fontsize{7}{10}\selectfont
    \resizebox{0.80\textwidth}{!}{\input{img/ej4/ej4_nodos_connected_sparse.tex}}
    \caption{Complejidad temporal para Grafos Conexos Poco Densos}
\end{figure}

\begin{figure}[H]
    \centering
    \fontsize{7}{10}\selectfont
    \resizebox{0.80\textwidth}{!}{\input{img/ej4/ej4_frontera_connected_sparse.tex}}
    \caption{Frontera para Grafos Conexos Poco Densos}
\end{figure}

\bigskip

\par Al igual que con los \'arboles, se observa para este tipo
    de grafos un comportamiento similiar (sino id\'entico) en la
    familia de grafos poco densos.

\par Sin ir m\'as lejos, los \'arboles son grafos conexos poco densos
    (de hecho, son los grafos conexos menos densos que hay, ya que
    por ser \'arboles, si se le quita cualquier arista, deja de ser
    conexo).

\par Nuevamente se observa como la heur\'istica golosa requiere menos
    tiempo que todas las dem\'as para terminar (aunque est\'a seguida
    de cerca por la variante de la b\'usqueda local).

\par Y, en cuanto al resultado, vemos que no est\'a m\'as lejos de
    la soluci\'on del algoritmo exacto que las dem\'as heur\'isticas
    (en particular, para las instancias en las que se pudo correr
    el algoritmo exacto -por limitaciones de tiempo-, las heur\'isticas
    est\'an razonablemente cerca de la soluci\'on \'optima).

\subsubsection{Grafos regularmente densos}
%Connected Regular
\begin{figure}[H]
    \centering
    \fontsize{7}{10}\selectfont
    \resizebox{0.80\textwidth}{!}{% GNUPLOT: LaTeX picture with Postscript
\begingroup
  \makeatletter
  \providecommand\color[2][]{%
    \GenericError{(gnuplot) \space\space\space\@spaces}{%
      Package color not loaded in conjunction with
      terminal option `colourtext'%
    }{See the gnuplot documentation for explanation.%
    }{Either use 'blacktext' in gnuplot or load the package
      color.sty in LaTeX.}%
    \renewcommand\color[2][]{}%
  }%
  \providecommand\includegraphics[2][]{%
    \GenericError{(gnuplot) \space\space\space\@spaces}{%
      Package graphicx or graphics not loaded%
    }{See the gnuplot documentation for explanation.%
    }{The gnuplot epslatex terminal needs graphicx.sty or graphics.sty.}%
    \renewcommand\includegraphics[2][]{}%
  }%
  \providecommand\rotatebox[2]{#2}%
  \@ifundefined{ifGPcolor}{%
    \newif\ifGPcolor
    \GPcolortrue
  }{}%
  \@ifundefined{ifGPblacktext}{%
    \newif\ifGPblacktext
    \GPblacktexttrue
  }{}%
  % define a \g@addto@macro without @ in the name:
  \let\gplgaddtomacro\g@addto@macro
  % define empty templates for all commands taking text:
  \gdef\gplbacktext{}%
  \gdef\gplfronttext{}%
  \makeatother
  \ifGPblacktext
    % no textcolor at all
    \def\colorrgb#1{}%
    \def\colorgray#1{}%
  \else
    % gray or color?
    \ifGPcolor
      \def\colorrgb#1{\color[rgb]{#1}}%
      \def\colorgray#1{\color[gray]{#1}}%
      \expandafter\def\csname LTw\endcsname{\color{white}}%
      \expandafter\def\csname LTb\endcsname{\color{black}}%
      \expandafter\def\csname LTa\endcsname{\color{black}}%
      \expandafter\def\csname LT0\endcsname{\color[rgb]{1,0,0}}%
      \expandafter\def\csname LT1\endcsname{\color[rgb]{0,1,0}}%
      \expandafter\def\csname LT2\endcsname{\color[rgb]{0,0,1}}%
      \expandafter\def\csname LT3\endcsname{\color[rgb]{1,0,1}}%
      \expandafter\def\csname LT4\endcsname{\color[rgb]{0,1,1}}%
      \expandafter\def\csname LT5\endcsname{\color[rgb]{1,1,0}}%
      \expandafter\def\csname LT6\endcsname{\color[rgb]{0,0,0}}%
      \expandafter\def\csname LT7\endcsname{\color[rgb]{1,0.3,0}}%
      \expandafter\def\csname LT8\endcsname{\color[rgb]{0.5,0.5,0.5}}%
    \else
      % gray
      \def\colorrgb#1{\color{black}}%
      \def\colorgray#1{\color[gray]{#1}}%
      \expandafter\def\csname LTw\endcsname{\color{white}}%
      \expandafter\def\csname LTb\endcsname{\color{black}}%
      \expandafter\def\csname LTa\endcsname{\color{black}}%
      \expandafter\def\csname LT0\endcsname{\color{black}}%
      \expandafter\def\csname LT1\endcsname{\color{black}}%
      \expandafter\def\csname LT2\endcsname{\color{black}}%
      \expandafter\def\csname LT3\endcsname{\color{black}}%
      \expandafter\def\csname LT4\endcsname{\color{black}}%
      \expandafter\def\csname LT5\endcsname{\color{black}}%
      \expandafter\def\csname LT6\endcsname{\color{black}}%
      \expandafter\def\csname LT7\endcsname{\color{black}}%
      \expandafter\def\csname LT8\endcsname{\color{black}}%
    \fi
  \fi
  \setlength{\unitlength}{0.0500bp}%
  \begin{picture}(7200.00,5040.00)%
    \gplgaddtomacro\gplbacktext{%
      \csname LTb\endcsname%
      \put(1430,2244){\makebox(0,0)[r]{\strut{} 0.01}}%
      \csname LTb\endcsname%
      \put(1430,2600){\makebox(0,0)[r]{\strut{} 1}}%
      \csname LTb\endcsname%
      \put(1430,2956){\makebox(0,0)[r]{\strut{} 100}}%
      \csname LTb\endcsname%
      \put(1430,3312){\makebox(0,0)[r]{\strut{} 10000}}%
      \csname LTb\endcsname%
      \put(1430,3667){\makebox(0,0)[r]{\strut{} 1e+06}}%
      \csname LTb\endcsname%
      \put(1430,4023){\makebox(0,0)[r]{\strut{} 1e+08}}%
      \csname LTb\endcsname%
      \put(1430,4379){\makebox(0,0)[r]{\strut{} 1e+10}}%
      \csname LTb\endcsname%
      \put(1562,2024){\makebox(0,0){\strut{} 0}}%
      \csname LTb\endcsname%
      \put(2086,2024){\makebox(0,0){\strut{} 500}}%
      \csname LTb\endcsname%
      \put(2610,2024){\makebox(0,0){\strut{} 1000}}%
      \csname LTb\endcsname%
      \put(3134,2024){\makebox(0,0){\strut{} 1500}}%
      \csname LTb\endcsname%
      \put(3658,2024){\makebox(0,0){\strut{} 2000}}%
      \csname LTb\endcsname%
      \put(4183,2024){\makebox(0,0){\strut{} 2500}}%
      \csname LTb\endcsname%
      \put(4707,2024){\makebox(0,0){\strut{} 3000}}%
      \csname LTb\endcsname%
      \put(5231,2024){\makebox(0,0){\strut{} 3500}}%
      \csname LTb\endcsname%
      \put(5755,2024){\makebox(0,0){\strut{} 4000}}%
      \csname LTb\endcsname%
      \put(6279,2024){\makebox(0,0){\strut{} 4500}}%
      \csname LTb\endcsname%
      \put(6803,2024){\makebox(0,0){\strut{} 5000}}%
      \put(176,3311){\rotatebox{-270}{\makebox(0,0){\strut{}Tiempo (microsegundos)}}}%
      \put(396,3311){\rotatebox{-270}{\makebox(0,0){\strut{}(Escala Logaritmica)}}}%
      \put(4182,1694){\makebox(0,0){\strut{}Cantidad de Nodos}}%
      \put(4182,1474){\makebox(0,0){\strut{}(Escala Lineal)}}%
      \put(4182,4709){\makebox(0,0){\strut{}Tiempo de ejecucion conforme aumenta la cantidad de nodos}}%
    }%
    \gplgaddtomacro\gplfronttext{%
      \csname LTb\endcsname%
      \put(6725,1053){\makebox(0,0)[r]{\strut{}Algoritmo Exacto (Regular)}}%
      \csname LTb\endcsname%
      \put(6725,833){\makebox(0,0)[r]{\strut{}Tabu(n,n,n/2) (golosa,sin aspiracion,regular)}}%
      \csname LTb\endcsname%
      \put(6725,613){\makebox(0,0)[r]{\strut{}Tabu(n,n,n/2,regular)}}%
      \csname LTb\endcsname%
      \put(6725,393){\makebox(0,0)[r]{\strut{}BL (mejor vecino,intercambio,regular)}}%
      \csname LTb\endcsname%
      \put(6725,173){\makebox(0,0)[r]{\strut{}Golosa (regular)}}%
    }%
    \gplbacktext
    \put(0,0){\includegraphics{ej4_nodos_connected_regular}}%
    \gplfronttext
  \end{picture}%
\endgroup
}
    \caption{Complejidad temporal para Grafos Conexos Regulares}
\end{figure}

\begin{figure}[H]
    \centering
    \fontsize{7}{10}\selectfont
    \resizebox{0.80\textwidth}{!}{\input{img/ej4/ej4_frontera_connected_regular.tex}}
    \caption{Frontera para Grafos Conexos Regulares}
\end{figure}


\bigskip

\par Para esta familia de grafos m\'as densos, vemos que los
    resultados obtenidos se mantienen, si bien ya las heur\'isticas
    comienzan a estar cada vez m\'as cerca, tanto en resultado provisto
    como en sus tiempo s de ejecuci\'on. A\'un as\'i, la heur\'istica 
    golosa siguie teniendo la ventaja respecto de las dem\'as y mientras
    que el resultado provisto por todas las heur\'isticas son similares
    (y vale la pena mencionar, hasta los grafos de 200 nodos se
    puede observar como todas las variantes de las heur\'isticas
    que est\'an siendo evaluadas dan resultados sumamente aceptables).

\subsubsection{Grafos muy densos}
%Connected Dense
\begin{figure}[H]
    \centering
    \fontsize{7}{10}\selectfont
    \resizebox{0.80\textwidth}{!}{\input{img/ej4/ej4_nodos_connected_dense.tex}}
    \caption{Complejidad temporal para Grafos Conexos Densos}
\end{figure}

\begin{figure}[H]
    \centering
    \fontsize{7}{10}\selectfont
    \resizebox{0.80\textwidth}{!}{\input{img/ej4/ej4_frontera_connected_dense.tex}}
    \caption{Frontera para Grafos Conexos Densos}
\end{figure}

\bigskip

\par Por \'ultimo, observamos como los resultados, si bien se mantienen,
    siguen acerc\'andose. Al menos hasta los 5000 nodos, la soluci\'on
    golosa se ha eregido como la elegida a la hora de resolver dichos
    problemas si no se tiene informaci\'on sobre la estructura de los
    grafos de entrada.

\subsection{Conclusiones}
\par Para finalizar este trabajo, vale la pena mencionar que quedaron
    muchas variantes de las heur\'isticas (especialmente de la b\'usqueda
    tab\'u) y de muchas otras familias de grafo para las cuales los
    resultados, seguramente, hubieran sido diferentes.

\par Como conclusiones, debemos afirmar que este \'ultimo an\'alisis
    global realizado se hizo sin tener informaci\'on de los grafos
    que modelan los problemas que se desean resolver. Dicha informaci\'on
    puede ser (de hecho, es lo principal) muy importante, sino cr\'itico,
    a la hora de desarrollar una heur\'istica (cosa que es necesaria
    si llegar a una soluci\'on \'optima para el problema es muy caro
    computacionalmente).

\par Al no tener dicha informaci\'on, se decidi\'o hacer unas heur\'isticas
    un tanto g\'enericas para un problema particular, y los resultados
    obteniods de la experimentaci\'on no son, ni est\'an cerca de serlo,
    concluyentes sobre que heur\'istica es ``mejor'' para resolver el
    problema de \emph{CMF}\footnote{En particular, como no se saben
    cuan buena alcanza que sea la soluci\'on obtenida, ni el tiempo
    m\'aximo que se est\'a dispuesto a esperar por ella, la definici\'on
    de ``mejor heur\'istica'' carace, en parte, de sentido.}.

        \pagebreak
\end{document}
