Situaciones de la vida en las que puede aplicarse el problema
de \emph{Clique de Frontera M\'axima}.

\subsubsection{Una empresa nacional y popular}

Tenemos un conjunto de plataformas en alta mar en las
que hay abundancia de distintos recursos que las grandes
empresas ans\'ian explotar. Estas plataformas estan conectadas 
entre sí por pasarelas. 

Lamentablemente para las empresas con capacidad de explotaci\'on, 
las plataformas en las que se permite explotar el tipo de recurso
en el que la empresa se especializa, rara vez comparten una pasarela
directa, sin embargo la l\'ogistica de su negocio requiere que 
transporten distintos tipos de recursos (gente, maquinaria, divisas) 
entre los nodos que quieren adjudicarse.

Nuestra empresa particularmente se especializa solamente
en el chantaje y la extorsi\'on, recursos que a priori
ninguna de las plataformas permite explotar. Sin embargo
conociendo un poco de la historia reciente de nuestro pa\'is
se nos ocurre la idea de plantearle a la autoridad regulatoria
de la explotaci\'on, hacernos cargo del mantenimiento de las 
pasarelas bajo la contra-prestaci\'on de cobrar un peaje
a las otras empresas que utilicen la v\'ia.

Se nos autoriza a hacer este negocio, sin embargo como la 
autoridad regulatoria no es tan inocente/corrupta como los
sucesivos gobiernos argentinos (desde por lo menos 1810), 
nos exige que para poder efectuar nuestra maniobra seamos 
los concesionarios de al menos uno de los extremos de cada
una de estas pasarelas y hacernos cargo del mantenimiento
de dicha plataforma tambi\'en.

Este requerimiento, nos exige una inversi\'on mayor a la
que nos gustar\'ia efectuar, en principio el mantenimiento
de una plataforma es mucho mas oneroso que el de una pasarela,
ademas de tener que pagar el canon de la misma y estar limitados
a no poder cobrar peaje por el transito en la plataforma.

En nuestra desesperaci\'on se nos ocurre un plan viable:
sub-alquilaremos las plataformas que tengamos que mantener
a empresas que deseen explotarlas (nosotros no tenemos el 
m\'as m\'inimo inter\'es en, ni la capacidad para producir nada). 

Sin embargo dentro del mundillo empresario, tenemos muy 
mala reputaci\'on y los interesados potenciales en explotar
las plataformas s\'olo est\'an dispuestos a alquilarnos a
nosotros a precio reducido. 

Luego de hacer puntillosamente las predicciones econ\'omicas
vemos que aun podemos sacar una ventaja global pero en cada
una de las plataformas estaremos perdiendo dinero.

Ahi nos viene a la mente el trabajo pr\'actico de 
aquel infame segundo cuatrimestre de 2013 presentado por 
la c\'atedra de Algoritmos y Estructuras de Datos III 
en la facultad de Ciencias Exactas - U.B.A. \ldots
si s\'olo lo hubieramos hecho entonces \ldots


\subsubsection{Futuro posible o futuro ideal?}
Estamos en el a\~no $15000$ d.c y somos capitanes de una flota
de piratas espaciales. 

Tenemos un conocimiento muy limitado
de la geograf\'ia espacial, b\'asicamente estamos circunscriptos
a nuestro coto de saqu\'eos habitual. \'Ultimamente
empezamos a sospechar que hemos sobre-explotado el mismo con
lo que se dificulta una proyecci\'on ec\'onomica sustentable
(adem\'as de ver seriamente disminuidas nuestras chances de
obtener placer por medio de las actividades de amenaza, coerci\'on
y en definitiva tortura que tanto disfrutamos), a pesar de esto
con los a\~nos hemos acumulado una interesante capacidad 
armament\'istica y somos expertos en el campo de batalla.

En uno de nuestros saqu\'eos descubrimos un mapa muy limitado
de otra regi\'on espacial inexplorada por nosotros pero cuyo 
amplio detalle cartogr\'afico indica habitabilidad y una
gran prosperidad de sus habitantes, con lo que se nos hace 
agua la boca.

Nos encontramos en un punto de quiebre, hasta el momento
suponiamos que el universo se circunscrib\'ia a nuestro 
universo conocido, pero en este momento avizoramos un 
crisol de oportunidades para la continuidad y sustentabilidad
de nuestras ilimitadas perversiones.

Como no somos especialistas en la navegaci\'on interestelar, nos
procuramos la colaboraci\'on de un experto en topolog\'ia espacial
(lease, secuestramos un cerebrito) para que nos explique en 
detalle el mapa y los conceptos subyacentes tales como los 
agujeros de gusano indicados en el mismo. Este accede a explicarnos, 
luego de marat\'onicas sesiones de negociaci\'on (tortura unilateral), 
sobre la organizaci\'on de la red de planetas descripta en el mapa.

B\'asicamente, la forma de llegar a ese c\'umulo de planetas es
la navegaci\'on con nuestras naves a traves de esos agujeros de
gusano hacia alguna de las zonas m\'as densamente populadas. Nos
explica tambi\'en que se detallan numerosos agujeros de gusano
sin destino conocido y nos revela su sospecha de que conducen 
a otros cuadrantes espaciales menos densos pero no por ello
menos apetitosos para nuestras actividades. 

El experto en topolog\'ia nos presenta un modelo donde los planetas 
habitados son asimilados a nodos y las rutas espaciales que los conectan
son asimiladas a aristas en un grafo no dirigido. Con este modelo
armado nos recomienda hacer un estudio de donde nos conviene 
desplegar nuestro ataque inicialmente, evaluando nuestro poder de
fuego en relaci\'on con el beneficio que podemos obtener de 
dominar todas las rutas (tanto conodcidas como desconocidas) 
accesibles desde esos c\'umulos sociales.

Si bien nos sobra el poder de fuego para dominar por completo 
uno de estos c\'umulos, somos realistas y sabemos que potencialmente
tendremos bajas. Tenemos el conocimiento intuitivo de que a menor
cantidad de batallas menor cantidad de bajas por lo tanto buscamos
un equilibrio donde tengamos la mayor cantidad rutas con el menor
despliegue posible.

El experto en topolog\'ia espacial ya exhausto m\'as all\'a del punto
de no retorno, con su \'ultimo suspiro nos introduce el concepto de 
clique de frontera m\'axima como soluci\'on \'optima a nuestras ansias.

Lamentablemente, luego de soportar estoicamente varios meses de tortura
nuestro experto nos abandona para conocer a su creador y nos deja con el 
problema planteado pero sin haber sido resuelto.

Es ese el instante preciso en el que lamentamos no haber aprobado el 
tercer trabajo pr\'actido de Algoritmos y Estructuras de Datos III 
por haber estado demasiado ocupados fantaseando con el futuro hace
aproximadamente 13000 a\~nos. Sin embargo nuestra torva faz se contrae
en una media sonrisa al darnos cuenta de la iron\'ia que conlleva que 
a pesar de ello, en la actualidad detentamos el poder del miedo.
