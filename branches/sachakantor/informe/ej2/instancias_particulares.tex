\subsubsection{Testeo}
\par Luego de haber dise\~nado el algoritmo, se realiz\'o una breve estapa
    de testeo para verificar que la implementaci\'on efectivamente
    diera con la soluci\'on correcta. Claramente, dicho testeo no es realmente
    pausible para instancias grandes del problema (¿para que ser\'ia necesario
    sino un programa que lo resuelva?), as\'i pues se decidi\'o utilizar la
    \emph{suite} de instancias y respuestas correctas provista por la c\'atedra%
    \footnote{\url{http://www.dc.uba.ar/materias/aed3/2013/2c/laboratorio/tests_tp3.serieA1.n4_20.zip},
    \url{http://www.dc.uba.ar/materias/aed3/2013/2c/laboratorio/tests_tp3.serieA2.n25_125.zip},
    \url{http://www.dc.uba.ar/materias/aed3/2013/2c/laboratorio/tests_tp3.serieB.n100_280.zip}}.

\par Ya estando este informe en manos del corrector, se puede asumir correctamente
    que dichos testeos fueron satisfactorios. Sin m\'as, se presenta a continuaci\'on
    una tabla con los resultados de las tres \emph{suites} (\emph{A1,A2} y \emph{B})
    y sus reseltados esperados (tambi\'en provistos por la c\'atedra):

\begin{center}
    \begin{longtable}{ |c||c|c|L{4cm}||c|c|L{4cm}| }
        \hline
        \multicolumn{7}{ |c| }{Algorimo Exacto para \emph{CMF} - Testeo}\\
        \hline\hline
        \multicolumn{4}{ |c||}{ Resultado Obtenido }& \multicolumn{3}{ c| }{ Resultado esperado }\\
        \hline
        Test & Frontera & $|CMF|$ & Nodos en $CMF$ & Frontera & $|CMF|$ & Nodos en $CMF$\\
        \hline\hline

        %Comienzo Tabla
        \hline
        \input{ej2/ej2.testeo.tabla}
        %------------------------------------------------------------
    \end{longtable}
\end{center}

\subsubsection{Familias para experimentaci\'on}
\par Ya habi\'endonos convencido de que el algoritmo funciona correctamente
    y da la soluci\'on exacta para los problemas de \emph{CMF}, pasamos a
    tratar de identificar familias de grafo donde experimentar con el algoritmo
    implementado y poder sacar conclusiones \'utiles.

\par En particular, este algoritmo da una soluci\'on exacta, y su complejidad
    est\'a en el orden exponencial, por lo cual, comenzar a experimentar
    con grafos con estructuras particulares no nos va a dar mucha m\'as informaci\'on
    que si prob\'asemos para grafos cualesquiera\footnote{Siempre conexos, por lo
    explicado en \emph{\nameref{notas_preliminares}, \nameref{notas:conexos}}.}.

\par Lo que si se puede observar en los pseudoc\'odigos expuestos, es que las
    podas implementadas para mejorar la eficiencia trabajan mucho sobre la lista
    de $candidatos$. Y el tama\~no de la misma va a tener una relaci\'on directa
    con la cantidad de aristas del grafo (entre otros posibles par\'ametros que,
    en el marco de este trabajo, no son necesarios de mencionar).

\par Por lo tanto, decidimos utilizar los conceptos de \emph{densidad} para grafos
    m\'as habitualmente usados\footnote{\url{http://en.wikipedia.org/wiki/Dense_graph}}.
    El mismo define la densidad de un grafo como la relaci\'on entre la cantidad
    de aristas de un grafo $G=(V,E)$ dado y la cantidad de aristas del grafo completo
    $K_{|V|}$.

\par As\'i pues, un grafo poco denso tendr\'a pocas aristas y uno denso tendr\'a muchas
    aristas.

\par Decidimos entonces, mediante esta definici\'on, experimentar para las siguientes
    familias de grafos conexos:

\begin{description}
    \item[Grafos Poco Densos] Familia de grafos conexos con una densidad entre $0\%$%
        \footnote{Siendo la cantidad m\'inima nominal de aristas igual a $n-1$, condici\'on
        necesaria para que el grafo sea conexo (definici\'on de \'arbol). Esto se tuvo en
        cuenta en el generador de instancias aleatorias.} y $33\%$.

    \item[Grafos Regulares] Familia de grafos conexos con una densidad entre $34\%$ y
        $66\%$.

    \item[Grafos Densos] Familia de grafos conexos con una densidad entre $67\%$ y
        $99\%$.

    \item[Grafos Completos] Familia de grafos completos (los cuales deber\'ian
        resolverse en tiempo lineal seg\'un las podas explicadas).

    \item[Grafos Planares] Familia de grafos planares. La motivaci\'on de esta familia
        est\'a en que al ser un grafo planar, este no contiene ning\'un subgrafo
        homeomorfo a $K_5$ ni $K_{3,3}$, por lo tanto sabemos de antemano que en realidad
        no existir\'a ninguna clique de m\'as de 4 nodos, con lo cual nuestro algoritmo
        exacto pasa de ser exponencial a ser lineal:

        \begin{equation*}
        \mathcal O(n\cdot \underbrace{2^n}_{\mathclap{\substack{\text{Cantidad m\'ax}\\ \text{de iteraciones}}}})
            \rightarrow \mathcal O(n \cdot \underbrace{n^4}_{\mathclap{\substack{\text{Cantidad m\'axima}\\
            \text{de iteraciones en}\\ \text{un grafo planar}}}}\footnote{%
                La maxima cantidad de iteraciones bajo la hip\'otesis de que
                se recibi\'o un grafo planar ser\'an todas las posibles combinaciones de 1, 2, 3 y 4 nodos (ya que
                no habr\'a cliques de tama\~no mayor a 4). Por lo tanto, podemos expresar esta cantidad como
                $\displaystyle\sum_{k=1}^4 \dbinom{n}{k} = \dfrac{1}{24}n(n^3-2n^2+11n+14)$%
            }) = \mathcal O(n^5)
        \end{equation*}
\end{description}

\subsubsection{Variante Golosa}
\par A punto de experimentar, tambi\'en ser\'ia interesante saber si la eficiencia
    de nuestro algoritmo es mejorada mediante una cota inferior para el $\delta(CMF)$.

\par Inicialmente, nuestro algoritmo calcula esta cota seg\'un se detalla en la
    Secci\'on~\ref{backtracking:poda:cota_front_min} (\nameref{backtracking:poda:cota_front_min}).

\par Adelant\'andonos un poco al desarrollo de este trabajo, decidimos utilizar como
    variante de esta cota el resultado devuelto por la heur\'istica golosa\footnote{%
    Secci\'on~\ref{golosa}, \emph{\nameref{golosa}}.}. Estos resultados se ver\'an
    reflejados a continuaci\'on.
