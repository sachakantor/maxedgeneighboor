\subsubsection{G\'enesis del algoritmo propuesto}
\par A la hora de pensar un algoritmo que pudiese resolver el problema presentado
    por el enunciado de este TP, comenzamos a investigar por trabajos previos que
    resolvieran problem\'aticas similares.

\par La resoluci\'on propuesta se bas\'o en un algoritmo desarrollado para encontrar
    todas las cliques maximales de un grafo no dirigido\footnote{%
    Bron C. y Kerbosch J. Finding All Cliques of an Undirected Graph. \emph{Journal
    of the ACM}, vol:16 575-579, 1973.}. El mismo planteaba un algoritmo basado en
    3 conjuntos (nodos candidatos, procesados y clique en construcci\'on) con los
    cuales iba generando las cliques de un grafo hasta no poder agregarle m\'as nodos
    sin que este dejase de ser clique, con lo cual se llegaba a una clique maximal.

\par Este algorimo claramente no resuelve el problema de la \emph{CMF}, pero esta
    estrechamente ligado al mismo, ya que la \emph{CMF} es una clique dentro del
    del grafo, y si tenemos un algoritmo que forma todas las cliques (formar todas
    las cliques maximales implica formar todas las cliques) podr\'iamos, en una
    primera versi\'on un tanto rudimentaria, calcular la frontera de cada una
    de ellas y luego quedarnos con la de mayor frontera.

\subsubsection{El algoritmo exacto}
\par Partiendo de la secci\'on anterior, dise\~namos nuestro algorimo, que lo que hace es
    ir construyendo cliques siempre y cuando determine que es posible, para la
    clique que se est\'a construyendo, superar la mejor frontera encontrada hasta
    el momento.

\par En esencia, nuestro algoritmo propuesto es un algoritmo de \emph{backtracking}
    como lo es el algoritmo de Bron y Kerbosch, pero adaptado nuestro problema
    particular, mediante el calculo de la frontera de las cliques que se van
    construyendo y guardando la clique de mayor frontera encontrada/constru\'ida.

\par Para la construcci\'on de las cliques, b\'asicamente la idea es comenzar por
    una clique (que inicialmente es un \'unico nodo/v\'ertice) y luego ver del conjunto
    de nodos restantes cuales nodos podr\'ian ser agregados a esta clique y mantener
    su propiedad de clique en el grafo. Luego, por cada uno de estos nodos \emph{%
    candidatos}, se agregar\'ian a la clique y se repetir\'ia el proceso hasta no tener
    m\'as nodos candidatos. De esta manera, se generar\'ian todas las cliques del grafo.

\par Ahora bien, hacer esto \'ultimo y calcular la frontera en cada paso puede ser
    muy costoso (m\'as all\'a de que estemos planteando un algoritmo de \emph{backtracking})
    e innecesario. Es por ello que se implementaron a su vez las siguientes podas
    que mejoran la eficiencia, y en algunos casos hasta la complejidad asint\'otica del
    algoritmo.

\subsubsection{Podas del algoritmo exacto}
\begin{description}
    \item[Tama\~no m\'aximo \emph{CMF}: ] Dado cualquier grafo $G$ conexo, proponemos
        que la \emph{CMF} tendr\'a a lo sumo $\left\lceil\frac{n}{2}\right\rceil$ nodos.
        Luego, en el algoritmo, podemos utilizar esto como l\'imite en las cliques que
        vamos construyendo. Si en el proceso de construcci\'on de la clique ya tenemos
        esta cantidad de v\'ertices, podemos considerar que no tiene m\'as sentido
        seguir agregando nodos a la clique, ya que seguro la \emph{CMF} no tendr\'a
        m\'as nodos que $\lceil\frac{n}{2}\rceil$ nodos.

        \par A continuaci\'on, y sin entrar en detalles formales, justificaremos
            esta proposici\'on.

        \par Sea una clique $K$ de m\'as de $\lceil\frac{n}{2}\rceil$ nodos en $G$%
            \footnote{$G$ es un grafo no trivial. El problema de \emph{CMF} no tiene
            mucho sentido ni conlleva dificultad alguna para grafos triviales.}

        \begin{enumerate}
            \item Supongamos primero que $n$ es par.
                \begin{enumerate}
                    \item Como $n$ es par, $K$ pasar\'ia a tener al menos $\frac{n}{2} + 1$
                        nodos. Luego, nos quedan $n - |K|$ nodos de $G$ que no son parte de $K$.
                        \label{tam_min_k_par}

                    \item Luego, sabemos que lo m\'aximo que un nodo $v \in K$  puede
                        sumarle a $\delta(K)$ es $n - |K|$ y lo m\'inimo es $1$\footnote{%
                        Si no sumase no tendr\'ia sentido tenerlo en $K$, ya que
                        se perder\'ian $|K|-1$ aristas para incluirlo en $K$ (ya que
                        debe ser clique) que podr\'ian ser parte de $\delta(K)$ si el
                        nodo no fuera parte de $K$.}.

                    \item Ahora, consideremos $K' = K-v$. $K'$ seguir\'a siendo clique
                        (pues $K$ era clique y se le quit\'o un nodo) y su frontera
                        ganar\'a $|K|-1$ aristas (las que "un\'ian" $v$ a $K$) y perder\'a
                        , a lo sumo, $n - |K|$. Es decir:
                        \par\bigskip
                        $\delta(K') \geq \delta(K) + (|K| - 1) - (n - |K|)$\footnote{%
                        Se puede observar que $|K|$ tiene siempre signo positivo en esta
                        inecuaci\'on.}
                        \bigskip

                    \item Ahora bien, sabemos por hip\'otesis que $|K| \geq \frac{n}{2} + 1$
                        (\ref{tam_min_k_par}), entonces:
                        \bigskip\par$
                            \delta(K') \geq \delta(K) + (\frac{n}{2}+1 - 1) - (n - (\frac{n}{2}+1))\\
                            \delta(K') \geq \delta(K) + \frac{n}{2} - n + \frac{n}{2} + 1\\
                            \delta(K') \geq \delta(K) + 1$
                \end{enumerate}\bigskip

            \item Supongamos ahora que $n$ es impar.
                \begin{enumerate}
                    \item Como $n$ es impar, $K$ pasar\'ia a tener al menos $\frac{n-1}{2} + 2$
                        nodos. Luego, nos quedan $n - |K|$ nodos de $G$ que no son parte de $K$.
                        \label{tam_min_k_impar}

                    \item Luego, sabemos que lo m\'aximo que un nodo $v \in K$  puede
                        sumarle a $\delta(K)$ es $n - |K|$ y lo m\'inimo es $1$.

                    \item Consideremos $K' = K-v$. $K'$ seguir\'a siendo clique
                        (pues $K$ era clique y se le quit\'o un nodo) y su frontera
                        ganar\'a $|K|-1$ aristas (las que "un\'ian" $v$ a $K$) y perder\'a
                        , a lo sumo, $n - |K|$. Es decir:
                        \par\bigskip
                        $\delta(K') \geq \delta(K) + (|K| - 1) - (n - |K|)$
                        \bigskip

                    \item Ahora bien, sabemos por hip\'otesis que $|K| \geq \frac{n-1}{2} + 2$
                        (\ref{tam_min_k_impar}), entonces:
                        \bigskip\par$
                            \delta(K') \geq \delta(K) + (\frac{n-1}{2}+2 - 1) - (n - (\frac{n-1}{2}+2))\\
                            \delta(K') \geq \delta(K) + \frac{n-1}{2}+1 - n + \frac{n-1}{2} + 2\\
                            \delta(K') \geq \delta(K) + 2$
                \end{enumerate}\bigskip

            \item Entonces, demostramos que efectivamente, tener una clique de m\'as de
                $\lceil\frac{n}{2}\rceil$ nodos no ser\'a una \emph{CMF}, ya que quit\'andole
                un nodo podemos incrementar su frontera.

        \end{enumerate}

    \item[Grafos completos: ] De los datos de entrada de un grafo sabemos ya $n$
        y $m$, por lo cual determinar si el grafo dado es un grafo completo tiene
        una complejidad $\mathcal O(1)$ (verificar si $m = \frac{n \cdot (n-1)}{2}$).

        \par Entonces, a modo de mejorar la eficiencia (pero no la complejidad temporal
            asint\'otica), podemos verificar esto al comienzo del algoritmo, y en caso
            de ser la entrada un grafo completo, podemos asegurar que la \emph{CMF} tendr\'a
            $\lfloor\sfrac{n}{2}\rfloor$ cualesquiera nodos, y por lo tanto una frontera de
            $\lfloor\sfrac{n}{2}\rfloor \cdot \lceil\sfrac{n}{2}\rceil$.

        \par Para justificar esto \'ultimo, afirmamos que la \emph{CMF} de cualquier grafo
            completo tendr\'a al menos $\lfloor\sfrac{n}{2}\rfloor$ nodos\footnote{La
            demostraci\'on de esta afirmaci\'on es similar (sino casi id\'entica)
            a la ya expuesta en la poda anterior, con la salvedad que se puede asegurar
            exactamente cuanto suma cada nodo a la frontera de $K$ y que la cota que se
            desea ver es la cota inferior y no la superior. Por lo tanto, se omite esta
            demostraci\'on por ser trivial.}. Asumiendo esto como
            correcto, y sumado a la propiedad ya demostrada en la poda anterior, nos quedar\'ia
            lo siguiente:

        \bigskip
        \par $\lfloor\sfrac{n}{2}\rfloor \leq |CMF| \leq \lceil\sfrac{n}{2}\rceil$

        \bigskip
        \par Por lo cual se deduce que $|CMF| = \lfloor\sfrac{n}{2}\rfloor$ o 
        $|CMF| = \lceil\sfrac{n}{2}\rceil$, pues la $|CMF| \in \mathbb{N}$ 

        \bigskip
        \par Luego, como la $|CMF|$ en un grafo completo es el tama\~no de la clique CMF
            por la cantidad de nodos que no est\'an en la clique, nos queda que, para
            cualquiera de los dos posibles tama\~nos de la \emph{CMF}:

        \par $|CMF| = \lfloor\sfrac{n}{2}\rfloor \cdot \lceil\sfrac{n}{2}\rceil$.

        \par Por lo cual, tomar $\lfloor\sfrac{n}{2}\rfloor$ cualesquiera nodos de $V(G)$
            es lo mismo que tomar $\lceil\sfrac{n}{2}\rceil$, ya que su frontera ser\'a
            la misma.

    \item[M\'axima frontera de la rama: ] Al comenzar a construir las cliques de $G$,
        lo que se hace es considerar el conjunto de nodos \emph{candidatos} que podr\'ian
        ser agregados preservando la caracter\'istica de "clique" (subgrafo completo
        inducido). Por lo tanto, teniendo ya una clique $K_k$ y un conjunto de nodos
        $candidatos$, se puede calcular una cota para la m\'axima frontera que
        se puede constru\'ir con $K_k$ y el conjunto $candidatos$ en un tiempo
        lineal respecto de $|candidatos|$.

        \par La idea es suponer que todos los nodos podr\'ian ser agregados a $K_k$
            al mismo tiempo manteniendo la propiedad de "clique" de $K$ (esto podr\'ia
            no ser posible, ya que al agregar un nodo a cualquier clique, sus nodos
            $candidatos$ podr\'ian verse reducidos ya que podr\'ian no ser adyacentes
            al nuevo nodo). Entonces, podr\'iamos comenzar a calcular cuanto sumar\'ia
            cada nodo $candidato$ a la clique si fuera, hipot\'eticamente, a\~nadido.

        \par De todos los nodos $candidatos$, el nodo de mayor grado ser\'a el que m\'as
            incremente a la frontera (suena razonable, y puede ser demostrado a partir
            de las \emph{Notas Preliminares} de este mismo documento). Consideremos
            cual ser\'ia la frontera si dicho nodo fuese agregado a $K_k$ (f\'acilmente
            calculable siguiendo las \emph{Notas Preliminares}).

        \par Supusimos que la lista de $candidatos$ no variar\'ia (salvo el nodo que
            pas\'o a ser parte de $K$). Entonces ahora tenemos una clique $K_{k+1}$
            (su tama\~no se increment\'o en uno). Repetimos este proceso de considerar
            agregar candidatos (el de mayor grado) a $K$ hasta que la frontera no se
            incremente m\'as. Esto ocurrir\'a seguro ya que la cantidad de nodos de $G$
            es finita, y por lo tanto tambien lo es $candidatos$ (puede que esto ocurra
            antes de quedarnos sin candidatos, ya que el tama\~no de $K$ se va incrementando
            en $1$ en cada "consideraci\'on", y como el grado del nodo candidato que se
            considera decrece, puede ocurrir que en algun momento el nodo candidato de
            mayor grado no tenga suficientes aristas para unirse a $K$ e incrementar
            la frontera).

        \par Al no incrementarse m\'as la frontera, habremos llegado a una cota superior
            para la frontera que se podr\'ia llegar a formar con $K_k$ y sus $candidatos$.
            Si esta no supera la frontera m\'axima encontrada hasta el momento (recordemos
            que el algoritmo de \emph{backtracking} va construyendo cliques y recordando
            cual es aquella de mayor frontera que pudo formar hasta el momento), podemos
            comenzar el \emph{backtracking}, ya que seguro no podremos mejorar la mejor
            soluci\'on encontrada hasta el momento con este $K_k$ y sus $candidatos$
            asociados.

    \item[Frontera m\'inima: ] Al comenzar a construir las cliques de $G$, la idea pasa
        por encontrar aquella clique que tenga la m\'axima frontera. La idea de esta
        poda es tratar de mejorar la frontera parcial inicial con la que comienza el
        algoritmo, para de esta manera poder cortar ramas del \emph{backtracking} mediante
        la poda anterior.

        \par Para poder hacer esto nos basamos en el \emph{N\'umero de Tur\'an}\footnote{%
            Bondy, J.A. y Murty U.S.R. Tur\'an's Theorem. \emph{Graph Theory With Applications}.
            7(3):109-111, 1976} \footnote{%
            Aigner, M. y Ziegler G. Tur\'an's Graph Theorem. \emph{Proofs from THE BOOK}.
            36:235-239, 1998}, que nos asegura la siguiente implicaci\'on:

        \par Sea $G$ un grafo libre de cliques $K_{r+1}$ $\implies$
            $m \leq \dfrac{r-1}{r}\cdot\dfrac{n^2}{2}$
        \bigskip

        \par Por el contra-rec\'iproco, podemos asegurar lo siguiente:
        \par Si $m > \dfrac{r-1}{r}\cdot\dfrac{n^2}{2}$ $\implies$
            $G$ tiene al menos una clique $K_{r+1}$ 
        \bigskip

        \par Trabajando un poco sobre esta implicaci\'on, llegamos a:
        \par Si $r < \dfrac{n^2}{-2m + n^2}$ $\implies$
            $G$ tiene al menos una clique $K_{r+1}$
        \bigskip

        \par Entonces, mediante una cantidad acotada de cuentas matem\'atica ($\mathcal O(1)$)
            podemos encontrar el m\'aximo $r$ para el cual esta \'ultima implicaci\'on nos
            asegura que existe un $K_{r+1}$, y con esto tenemos una cota inferior del tama\~no
            de una clique que seguro existe en $G$.

            \par Luego, utilizando el mismo concepto de la poda \emph{Grafos Completos}, podemos
            asegurar que en $G$ habr\'a, como m\'inimo, una clique de frontera
            $\lfloor\frac{r+1}{2}\rfloor \cdot \lceil\frac{r+1}{2}\rceil$, que en grafos con una
            relacion $\frac{m}{n}$ alta, es una cota un tanto mejor que empezar con 0.

    \item[Candidatos que sumen frontera: ] En cada paso nuestro algorimo tiene
        una clique $K$ y su $\delta(K)$. A su vez, calcula su conjunto de $candidatos$
        asociados, es decir, el conjunto de nodos de $G$ que no forman parte de $K$ y
        que podr\'ian ser a\~nadidos a este manteniendo su caracter\'istica de clique%
        \footnote{Y que no fueron ya a\~nadidos a $K$ en otra permutaci\'on del backtracking
        (recordemos que estamos hablando de conjuntos y estos no tienen orden asociado, por
        lo cual si se consider\'o ya la clique \{1;2;3\}, y ahora se est\'a trabajando con
        la clique \{2;3\}, no tiene sentido considerar al $1$ como candidato).}.

        \par Ahora bien, ¿tiene sentido considerar como nodos $candidatos$ aquellos que,
            a pesar de poder ser a\~nadidos a $K$, no incrementar\'ian su frontera? Claramente
            no. Esto se deduce de que ningun nodo que se agregue a una clique nos incrementar\'a
            la lista de $candidatos$, a lo sumo la mantendr\'a igual (es decir, al menos la reducir\'a
            en 1 nodo: el mismo). Por lo tanto, si agregasemos dicho nodo (llamemoslo $v$), y
            luego otro ($w$) que mantenga la propiedad de clique de $K+v+w$,
            quitando $v$ (que no incrementaba la frontera), habremos ganado respecto de $\delta(K)$
            al menos 1 arista para la frontera: aquella que ten\'ia en un extremo a $w$ y en
            el otro a $v$.

        \par Por lo tanto, s\'olo tiene sentido considerar como nodos candidatos aquellos que
            de ser agregados mantienen la caracter\'istica de clique y que a la vez incrementan
            la frontera.

        \par Matem\'aticamente, concluyendo a partir de las \emph{Notas Preliminares}, consideramos
            aquellos candidatos cuyo grado supere estrictamente el doble de la cantidad
            de nodos de la clique parcial\footnote{Sea $K$ una clique, al agregar un nodo $v$, se pierden
            $|K|$ aristas de la frontera, y se ganan $d(v) - |K|$}. Es decir:

        \bigskip
        \par Sea $v \in V(G)$ y $K$ la clique parcial que se est\'a construyendo,
        \par $v$ ser\'a candidato $\iff$ $K+v$ es clique y si $d(v) > 2|K|$.

        \bigskip
        \par De esta manera, quitamos de la lista de candidatos nodos que no nos
            llevar\'an seguro a resultados que mejoren la frontera parcial, reduciendo
            as\'i ramas del \emph{backtracking}

\end{description}

\subsubsection{Pseudoc\'odigo descriptivo}
\par Se presenta a continuaci\'on un pseudoc\'odigo que describe los pasos que sigue
    nuestro algoritmo propuesto. El mismo no contempla estructuras de datos sobre
    los cuales se implement\'o (cosa que se realiza en el c\'alculo de complejidad),
    simplemente tiene como objetivo describir el algoritmo.

\begin{pseudocodigo}[Algoritmo Exacto para \emph{CMF} - Descriptivo]
    \Require\Statex
        \begin{itemize}
            \item Un grafo $G$ de $n$ v\'ertices\footnote{Asumimos, sin p\'erdida 
                de generalidad, que los v\'ertices est\'an numerados de 1 a $n$.}
                y $m$ aristas.

            \item Una funci\'on $candidatos(K)$, que dado un conjunto de v\'ertices
                $K$, devuelve una secuencia de v\'ertices de $G$ ordenada por grado
                de mayor a menor que son adjacentes a todos los elementos de $K$ y
                cuyo grado es mayor $2|K|$.

            \item Una funci\'on $\delta(K)$, que dada una \emph{clique} $K$\footnote{Una
                \emph{clique} no es otra cosa que un conjunto de v\'ertices, con la
                caracter\'istica de inducir un grafo completo en $G$.} en $G$, calcula
                el cardinal de su frontera\footnote{Como ya se explic\'o en las notas preliminares,
                sea $K$ una \emph{clique} de $G$, entonces: $\delta(K) = - |K|(|K|-1) +
                \displaystyle\sum_{v \in K} d(v)$}.

            \item Una funci\'on $\delta_{cota}(K)$, que dada una \emph{clique} $K$
                en $G$, calcula la cota de su frontera seg\'un el m\'etodo
                ya explicado anteriormente.

        \end{itemize}
    \Statex
    \Ensure Una \emph{clique} $K$ de $G$ con una frontera $\delta(K)$ de m\'axima
        cardinalidad.

    \Statex

    \If{$m = \dfrac{n(n-1)}{2}$} \Comment{Si $G$ es un $K_n$}
        \State $K \gets \left\{1;\dots;\left\lfloor\dfrac{n}{2}\right\rfloor\right\}$

    \Else
        \State $\delta_{max} \gets \left\lfloor\dfrac{r+1}{2}\right\rfloor\cdot
            \left\lceil\dfrac{r+1}{2}\right\rceil$ para $r$ tal que:
            $\begin{pmatrix}
                \text{\Huge$\bigwedge$} &
                    \begin{matrix}
                        r \in [1..n)\\[0.5cm]
                        r < \dfrac{n^2}{n^2 - 2m}\\[0.5cm]
                        (\forall r' \in [1..n))$ $r' < \dfrac{n^2}{n^2 - 2m} \implies r'\leq r
                    \end{matrix}
            \end{pmatrix}$\footnote{Esta cota inferior para la cardinalidad de la
                frontera maximal en $G$ se deriv\'o a partir del \emph{Teorema de Tur\'an},
                como ya se explic\'o anteriormente.}

        \State $K \gets \emptyset$, $K' \gets \emptyset$, $YaProcesados(K')$\footnote{%
            $YaProcesados(K')$ representar\'a a los v\'ertices de $candidatos(K')$ que
            ya fueron tenidos en cuenta para la \emph{clique} $K'$.}$ \gets \emptyset$

        \While{$candidatos(K')$\footnote{Si $K'$ es vac\'io, $candidatos(K')$ son todos
            los nodos de $G$.}$\setminus YaProcesados(K')$\footnote{Peque\~no abuso
            de notaci\'on, ya que $candidatos(K')$ es una secuencia y $YaProcesados(K')$
            es un conjunto. Se asume que es la secuencia sin los elementos del conjunto.}$
            \neq \emptyset$ $\lor$ $K' \neq \emptyset$}

            \If{$candidatos(K') \setminus YaProcesados(K') \neq \emptyset$ $\land$ $\delta_{max} < \delta_{cota}(K')$%
                \footnote{Si $K'$ es vac\'io, $\delta_{cota}(K')$ es $m$.}}

                \State $v \gets cabeza(candidatos(K') \setminus YaProcesados(K'))$
                \State $K' \gets K' \cup \{v'\}$
                \If{$candidatos(K') = \emptyset$ $\land$ $\delta_{max} < \delta(K')$}
                    %\Comment{Llegu\'e al final de la rama y obtuve una mejor solucion}
                    \State $\delta_{max} \gets \delta(K')$
                    \State $K \gets K'$
                \EndIf

            \ElsIf{$K' \neq \emptyset$} \Comment{\emph{Backtracking}}
                \State $v \gets ultimo(K')$\footnote{Donde $ultimo(K')$
                    es el \'ulitmo v\'ertice agregado a $K'$.}
                \State $K' \gets K' \setminus \{v'\}$
                \State $YaProcesados(K') \gets YaProcesados(K') \cup \{v\}$

            \Else
                \State $YaProcesados(K') \gets YaProcesados(K') \cup candidatos(K')$
            \EndIf
        \EndWhile
    \EndIf

    \State \Return{$K$}
\end{pseudocodigo}
