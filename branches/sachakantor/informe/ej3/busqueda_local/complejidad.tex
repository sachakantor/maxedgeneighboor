\subsubsection{Estructuras de Datos}

\subsubsection{Pseudoc\'odigo de complejidad}
\par Se presenta a continuaci\'on un pseudoc\'odigo m\'as espec\'ifico de la implementaci\'on
    de este algoritmo provista junto con este trabajo. El mismo tiene en cuenta
    las estructuras de datos explicadas en el punto anterior.

\par Luego del pseudoc\'odigo se justifican detalladamente las complejidades
    expuestas a continuaci\'on que no sean evidentes\footnote{Consideramos
    como "complejidades evidentes" las asignaciones de variables, operaciones
    m\'atematicas simples, asignaciones/inicializaci\'on de posiciones de
    un vector/\emph{deque} o cualquier contenedor de acceso aleatorio/arbitrario}.

\bigskip

\begin{pseudocodigo}[Heur\'istica de B\'usqueda Local para \emph{CMF} - Complejidad]
    \Require Un grafo $G$ con $n$ v\'ertices numerados de $1$ a $n$ y $m$ aristas. El mismo
        cuenta con las siguientes estructuras de datos que lo modelan:
        \begin{itemize}
            \item Vectores de adyacencia: Dado un vertice $v$, $vecinos(v)$ nos da todos los
                nodos adyacentes a $v$ en $G$.

            \item Matriz de adyacencia: Dados los v\'ertices $v$ y $w$, $adyacentes(v,w)$ y
                $adyacentes(w,v)$ nos devuelven $true$ si y s\'olo si $v$ es adyacente
                a $w$ en $G$.

            \item Vector de nodos de $G$.
        \end{itemize}
    \Ensure\Statex
        \begin{itemize}
            \item Un vector $K$ correspondiente a la \emph{clique} de m\'axima frontera
                encontrada por la heur\'istica.

            \item El cardinal de $\delta(K)$, siendo $K$ la \emph{clique} del item anterior.
        \end{itemize}
    \Statex

    \State $K \gets \emptyset$ \Compl{Brown}{}{$n$}{}
    \If{$m = \frac{n(n-1)}{2}$} \Compl{Blue}{}{$1$}{}
        \State $K \gets \left\{1;\dots;\left\lfloor\sfrac{n}{2}\right\rfloor\right\}$ \Compl{Blue}{}{$n$}{}
        \State $\delta_{max} \gets \left\lfloor\sfrac{n}{2}\right\rfloor\cdot
            \left\lceil\sfrac{n}{2}\right\rceil$ \Compl{Blue}{}{$1$}{}
        \Statex

    \Else
    \EndIf \Compl{Brown}{Costo \emph{si}: }{}{}

    \State \Return{$\delta(K)$, $K$} \Compl{Brown}{}{$1$\label{bl:return}}{}
    \Statex
    \Statex \Compl{Brown}{Costo Total de la Heur\'istica: }{}{}
\end{pseudocodigo}

\bigskip

\par
