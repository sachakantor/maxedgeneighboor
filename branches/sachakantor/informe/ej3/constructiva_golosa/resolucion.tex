\subsubsection{G\'enesis de la heur\'istica propuesta}
\par La idea central de la heur\'istica que se describir\'a a continuaci\'on
    surgi\'on a partir del estudio/definici\'on del problema presentado
    y de un an\'alisis r\'apido de la estructura de un gran subconjunto
    de la familia de grafos conexos.

\par B\'asicamente, como lo que se trata de encontrar es una clique
    en un grafo conexo tal que se maximice la cantidad de aristas
    incidentes en la misma (que no sean parte de la clique, obviamente),
    existe una tenedencia a que en la soluci\'on de las distintas
    instancias del problema incluya a alg\'un(os) nodos de grado
    alto (o al menos, por encima del grado promedio del grafo).

\par Esto es razonable, aunque es claro que es una tendencia y no
    una norma general que se cumple siempre (de hecho, en la secci\'on
    de experimentaci\'on que corresponde a esta heur\'istica se exponen
    familias de grafos para las cuales justamente no se cumple). Asi pues,
    esto da motivo a pensar que a partir de los nodos de mayor grado
    de un grafo existe una buena probabilidad de encontrar la clique
    de frontera m\'axima, o alguna que tenga una frontera cercana
    a la m\'axima.

\subsubsection{La heur\'istica golosa constructiva}\label{goloso:explicacion}
\par El algoritmo que se propone a continuaci\'on es bastante simple.
    Por lo ya explicado, consideramos que los nodos de mayor/alto
    grado de un grafo seguramente esten inclu\'idos en una clique
    de frontera m\'axima o cercana a la m\'axima.

\par Tomamos entonces el nodo de mayor grado (ya que debemos elegir uno)
    del grafo de entrada. Este pasar\'ia a ser nuestra clique inicial.

\par Luego, consideramos como nodos candidatos a a\~nadirse a esta
    clique a aquellos que sean adyacentes a todos los nodos de la
    clique parcial (en el primer caso se\'an los nodos vecinos del
    nodo de mayor grado) y que cumplan con la condici\'on de incrementar
    la frontera en caso de ser inclu\'idos en la clique (este es
    exactamente el mismo principo explicado en la Secci\'on
    \ref{backtracking:poda:candidatos_que_sumen},
    \emph{\nameref{backtracking:poda:candidatos_que_sumen}}).

\par Luego, agregamos al nodo candidato (esta es la caracter\'istica \textbf{constructiva}
    de la heur\'istica) que m\'as incremente la frontera de la clique (la
    caracter\'istica \textbf{golosa}).

\par Al haber incrementado el tama\~no de la clique, la lista de
    nodos candidatos se habr\'a modificado, ya que ahora hay un nuevo
    nodo (y sus adyacencias) que deber\'an ser consideradas. En particular,
    podemos asegurar que la cantidad de candidatos se reducir\'a (ya que,
    al menos, hay un nodo candidato que pasa a ser parte de la clique,
    dejando este de ser un nodo candidato).

\par Por \'ultimo, repetimos este proceso hasta que no haya m\'as
    nodos candidatos para agregar a la clique.

\par Vale mencionar, que la cota/poda \emph{\nameref{backtracking:poda:grafo_completo}}
    explicada en la Secci\'on~\ref{backtracking:poda:grafo_completo} puede
    ser utilizada en esta heur\'istica tambi\'en para dar la soluci\'on
    exacta de manera m\'as eficiente (ya que es trivial notar que la heur\'istica
    descripta nos dar\'a la respuesta exacta para la familia de grafos
    grafos completos).

\bigskip

\subsubsection{Pseudoc\'odigo descriptivo}
\par Se presenta a continuaci\'on un pseudoc\'odigo que describe los pasos que sigue
    nuestra heur\'istica propuesto. El mismo no contempla estructuras de datos sobre
    los cuales se implement\'o (cosa que se realiza en el c\'alculo de complejidad),
    simplemente tiene como objetivo describir la heur\'istica.

\begin{pseudocodigo}[Heur\'istica Golosa Constructiva para \emph{CMF} - Descriptivo]
    \Require\Statex
        \begin{itemize}
            \item Un grafo $G$ de $n$ v\'ertices y $m$ aristas.

            \item Una funci\'on $candidatos(K)$, que dado un conjunto de v\'ertices
                $K$, devuelve el conjunto de v\'ertices de $G$ que son adjacentes
                a todos los elementos de $K$, cuyo grado es mayor $2|K|$.
        \end{itemize}

    \Statex

    \Ensure Una \emph{clique} $K$ de $G$ con una frontera $\delta(K)$ que se
        espera que sea de cardinalidad cercana o igual a la de $\delta(K_{max})$,
        siendo $K_{max}$ la \emph{clique} de m\'axima frontera de $G$.

    \Statex

    \If{$G$ es un grafo completo $K_n$}
        \State $K \gets \lfloor\sfrac{n}{2}\rfloor$ nodos distintos de $V(G)$ cualesquiera.
        \Statex

    \Else
        \State Sea $v$ el v\'ertice de mayor grado en $G$.
        \State $K \gets \{v\}$
        \While{$candidatos(K) \neq \emptyset$}
            \State Sea $v'$ el v\'ertice de mayor grado en $candidatos(K)$.
            \State $K \gets K \cup \{v'\}$
        \EndWhile
    \EndIf

    \State \Return{$K$}
\end{pseudocodigo}
