\label{notas_preliminares}
\phantomsection
\addcontentsline{toc}{subsection}{Sobre la nomenclatura}
\subsection*{Sobre la nomenclatura}
\par En este trabajo se utilizan algunos acr\'onimos los cuales es inc\'omodo
    aclarar en cada caso. Por lo tanto presentamos aqu\'i una lista de ellos
    de manera que queden ya explicados para la extensi\'on de todo este
    documento.

\begin{description}
    \item[G: ] Nos referimos a un grafo no dirigido y sin pesos en los ejes
        cualquiera.

    \item[n: ] Se refiere a la cantidad de nodos de un grafo $G$.

    \item[m: ] Se refiere a la cantidad de aristas de un grafo $G$.

    \item[V(G): ] Es el conjunto de nodos de un grafo $G$.

    \item[E(G): ] Es el conjunto de ejes de un grafo $G$.

    \item[CMF: ] Se refiere a una clique de frontera maximal en un grafo $G$.

    \item[$\delta$(K): ] Es la frontera de una clique $K$ en un grafo $G$.

    \item[$|$G'$|$: ] Es la cantidad de nodos de un subgrafo inducido $G'$ en un grafo $G$.

    \item[K$_k$: ] Es un grafo completo $K$ de $k$ v\'ertices.

    \item[d(v): ] Sea $v \in V(G)$, $d(v)$ es el grado de este en $G$.

    \item[K+v: ] Sea $v \in V(G)$, y $K \subset V(G)$ tal que induce un grafo
        $K$\footnote{Peque\~no abuso de notaci\'on, nos referimos a $K$ como el
        (sub)conjunto de nodos as\'i como al subgrafo que induce en $G$.}, decimos
        entonces que $K+v$ es agregar al nodo $v$ a $K$.
\end{description}

\phantomsection
\addcontentsline{toc}{subsection}{Sobre las referencias de este documento}
\subsection*{Sobre las referencias de este documento}
\par A lo largo de este informe existen referencias tanto a \emph{papers}
    encontrados durante la elaboraci\'on de las resoluciones como as\'i
    referencias a cap\'itulos espec\'ificos de libros que son parte de la
    bibliograf\'ia de la materia.

\par Los mismos han sido puestos a disposici\'on del corrector en un repositorio
    de archivos \emph{online}, cuya direcci\'on \emph{web} est\'a inclu\'ida
    en los archivos provistos con este documento.

\phantomsection
\addcontentsline{toc}{subsection}{Sobre los grafos no conexos}
\subsection*{Sobre los grafos no conexos\label{notas:conexos}}
\par Durante todo este trabajo se asumen que todos los grafos de entrada son
    grafos conexos. Esto lo hacemos ya que utilizando alg\'un algoritmo que recorra
    grafos (como por ejemplo \emph{DFS} o \emph{BFS}) se podr\'ian identificar
    las componentes conexas y luego resolver el problema para cada componente
    por separado, y al terminar quedarnos con la mejor soluci\'on, siendo esta
    la \emph{CMF} de todo el grafo.

\par En t\'erminos de complejidad, todos los algoritmos/heur\'isticas aqu\'i
    presentados tienen una complejidad mayor a $\mathcal O(n+m)$, que es
    la complejidad de los algoritmos que nos permiten identificar las componentes
    conexas de un grafo cualquiera. Por lo tanto, asint\'oticamente hablando,
    la complejidad de nuestros algoritmos no se ve afectada.

\par Por falta de tiempo, no se implement\'o esta rutina que identifica las componentes
    conexas, pero considerando que en la complejidad asint\'otica temporal no se
    ve influ\'ida, se decidi\'o hacer esta aclaraci\'on y asumir para todos los algoritmos que
    los grafos de entrada son conexos.

\phantomsection
\addcontentsline{toc}{subsection}{Sobre el c\'alculo de la frontera de una clique}
\subsection*{Sobre el c\'alculo de la frontera de una clique\label{notas:calc_front}}
\par Sea $K$ una clique de un grafo $G$, entonces su frontera depender\'a de los
    grados de los nodos que la componen. Sabemos que $K$ por ser clique, debe inducir
    un subgrafo completo en $G$, por lo cual cada nodo suma a la frontera
    todas sus aristas incidentes salvo aquellas que se usan para formar la
    clique, es decir que un nodo $v \in K$ le suma a $\delta(K)$ una cantidad
    de aristas expresable como $d(v)-(|K|-1)$.

\bigskip
\par Entonces, $\delta(K)$, es expresable como $\displaystyle\sum_{v \in K}d(v) - (|K|-1)$.
\bigskip

\par Como la cantidad de nodos en $K$ es justamente $|K|$, la expresi\'on
    que nos queda es:

\bigskip
\par $\delta(K) = -|K|(|K|-1)\displaystyle\sum_{v \in K}d(v)$
\bigskip

\par M\'as a\'un, de aqu\'i podemos tambi\'en calcular cuantas aristas
    incrementar\'ia un nodo al ser a\~nadido a una clique $|K|$. Supongamos
    que tenemos un nodo $v$ que puede ser agregado a $K$ tales que $K+v$
    es clique. Sea $\delta(K)$ la frontera de $K$. Entonces, como ya explicamos,
    $v$ le agrega $d(v)-(|K+v|-1)$ nodos a esta nueva clique $K+v$. Pero $|K+v|$
    no es otra cosa que $|K|+1$, quedando entonces que $v$ le a\~nade
    $d(v)-|K|$ aristas.

\par Pero al agreg\'arsele un nodo a $K$, esta pierde $|K|$ aristas de su frontera,
    que son las utilizadas para poder incluir a $v$ en $K+v$ y no perder la
    condici\'on de clique. Entonces, podemos expresar la frontera de $\delta(K+v)$
    en funci\'on de la frontera de $K$, las aristas que se pierden y ganan por agregar
    a $v$:

\bigskip

\par $\delta(K+v) = \delta(K) + \underbrace{d(v)-|K|}_{\text{Lo que se gana}}
    - \underbrace{|K|}_{Lo que se pierde} = \delta(K) + d(v) - 2|K|$

\phantomsection
\addcontentsline{toc}{subsection}{Sobre la complejidad de instanciar un grafo en la implementaci\'on}
\subsection*{Sobre la complejidad de instanciar un grafo en la implementaci\'on}\label{grafo:estructuras}
\par Entre los archivos fuente provistos junto con este trabajo, se opt\'o por
    crear una clase \emph{Graph} de \emph{C++} que encapsulase todos los algoritmos
    requeridos por este trabajo.

\par Ahora bien, esta clase contiene ciertas estructuras internas de datos
    para modelar los grafos de entrada. Las mismas en cuesti\'on son listas/\emph{deques}
    de adyacencias y una matriz de adyacencia. Si bien ambas estructuras nos
    brindan la misma informaci\'on (es decir, son redundantes), a la hora
    de implementar nuestros algoritmos, tener ambas estructuras nos brindan
    mayor comodidad seg\'un la operaci\'on que estemos realizando. No es lo
    mismo verificar si dos nodos son adyacentes en una matriz de adyacencia que
    recorrer la lista de adyacencia de uno de los nodos (la complejidad en
    el primer caso es $\mathcal O(1)$ y en el segundo $\mathcal O(n)$).

\par Sin entrar mucho en detalles, se puede prevenir al lector que la \emph{%
    instanciaci\'on} de dichas estructuras tendr\'a un costo de $\mathcal O%
    (n^2)$ (instanciar una matriz de $n^2$ posiciones, sumado a $n$ listas/\emph{deques} de
    hasta $n$ elementos) el cual no se considera en el c\'alculo de la complejidad
    asint\'otica temporal de los algoritmos dise\~nados durante el curso de este
    documento ya que todos estos superan dicha complejidad.

\par A su vez, en la estructura interna de esta clase \emph{Graph}, se tiene
    un vector\footnote{\url{http://www.cplusplus.com/reference/vector/vector/}}
    que contiene punteros a nodos (donde \emph{nodo} es un \emph{struct} definido
    por nosotros) que nos da acceso en tiempo constante a la lista de adyacencia
    de dicho nodo, a su "id" y a su grado en $G$.

\par En algunas de las variantes de los algoritmos presentados aqu\'i, se decide
    a su vez ordenar las listas de adyacencia con un cierto criterio (explicado
    en cada caso en particular), pasando la complejidad de la instanciaci\'on
    de esta clase de $\mathcal O(n^2)$ a $\mathcal O(n^2 \cdot log(n))$, y
    en algunos casos esta complejidad si ha de tenerse en cuenta para los algorimos
    propuestos. Estos casos est\'an aclarados apropiadamente en caso de ser
    necesario.

\par Por \'ultimo, entre las estructuras tambi\'en se guarda la cantidad
    de nodos y de ejes en dos \emph{unsigned int} para poder acceder a dicha
    informaci\'on en tiempo constante.

\phantomsection
\addcontentsline{toc}{subsection}{Sobre la compilaci\'on de los archivos fuente}
\subsection*{Sobre la compilaci\'on de los archivos fuente}
\par En los archivos provistos junto con este trabajo se pueden distinguir dos
    directorios donde hay archivos fuentes compilables: \emph{src} y
    \emph{experimentaci\'on}. En ambos hay ya un archivo \emph{Makefile} con el
    cual se pueden compilar todos los binarios desarrollados y utilizados en
    el marco de este trabajo.

\par Cabe destacar, que en la implementaci\'on provista, se permite compilar
    dichos binarios con un dise\~no \emph{multi-thread}\footnote{\url{%
    http://www.openmp.org}}, donde cada \emph{hilo} de ejecuci\'on tomar\'a
    una instancia distinta del problema provista por el \emph{standard input}
    y la resolver\'a con el algoritmo correspondiente en paralelo. Esto nos
    permite ejecutar muchas instancias en simult\'aneo, pero sin afectar
    el tiempo medido (ya que cada \emph{thread} es resuelta en un procesador
    diferente). En caso de querer compilar de esta manera los binarios, se
    debe ejecutar el siguiente comando desde el directorio \emph{src} o
    \emph{experimentaci\'on}:

\bigskip
\par make EXTRACXXFLAGS=-fopenmp
\bigskip

\par Para poder compilar el binario de esta manera, se requiere tener la 4.4
    o superior del compilador \emph{GNU Compiler Collection}\footnote{\url{%
    http://gcc.gnu.org/}}.

\bigskip
\par A su vez, dada la cantidad de variantes por heur\'istica que se
    proveen, existen reglas para compilar los binarios relacionados
    \'unicamente con una heur\'istica particular. Estas reglas son:

\begin{itemize}
    \item \textbf{backtracking: } Compila el algoritmo exacto.
    \item \textbf{golosa: } Compila las heur\'isticas golosas.
    \item \textbf{busqueda: } Compila las heur\'isticas de b\'usqueda local.
    \item \textbf{tabu: } Compila las heur\'isticas de b\'usqueda tab\'u\footnote{%
        Un total de 72 binarios para esta opci\'on. Compilar con paciencia}.
\end{itemize}.

\par Por ejemplo, si s\'olo se quisiera compilar el algoritmo exacto
    y la heur\'istica constructiva-golosa, habr\'ia que ejecutar:

\bigskip
\par make backtracking golosa
\bigskip

\bigskip
\par O, si se quiere habilitar la funcionalidad \emph{multi-thread}:
\bigskip
\par make EXTRACXXFLAGS=-fopenmp backtracking golosa

\phantomsection
\addcontentsline{toc}{subsection}{Sobre los generadores de instancias aleatorias}
\subsection*{Sobre los generadores de instancias aleatorias}
\par Una vez que se implementaron los algoritmos que resuelven los problemas
    presentados por el enunciado de este trabajo, se llega a la etapa de
    an\'alisis/experimentaci\'on donde se quieren verificar los resultados
    te\'oricos obtenidos con informaci\'on emp\'irica. Para ello, en cada caso/%
    algoritmo, se hizo un an\'alisis sobre alg\'un subconjunto/familia de grafos
    de inter\'es (explicado en cada caso particular durante este informe). Luego,
    se implement\'o un generador de instancias aleatorias para estas familias
    (el cual es compilado seg\'un como se indica en el punto anterior) el cual,
    mediante una interfaz de men\'u por l\'inea de comando, nos permite seleccionar
    la familia, el tama\~no de las instancias a generar, cuantas instancias por
    tama\~no generar y el archivo donde se desea guardar toda esta informaci\'on%
    \footnote{Puede ocurrir que para ciertas familias se requiera alg\'un otro dato,
    el cual ser\'a demandado al usuario por la pantalla.}.

\par El concepto de aleatoriedad en estos generadores ocurre a la hora de elegir
    los extremos de las aristas (siempre preservando la estructura de la familia
    elegida). Para poder hacer esta selecci\'on aleatoria, se utilizaron funciones
    del a \emph{STL} de \emph{C} portadas a \emph{C++}\footnote{%
    \url{http://www.cplusplus.com/reference/cstdlib/rand/}} y algunas de la
    \emph{STL} nativa de \emph{C++}\footnote{%
    \url{http://www.cplusplus.com/reference/algorithm/random_shuffle/}}.

\par A su vez, hay que mencionar que uno de los objetivos principales de este
    trabajo es poder comparar distintas formas de resolver el problema
    presentado. Entonces, es razonable pretender usar siempre el mismo conjunto
    de instancias de la familia que se desea comparar en los algoritmos. Por
    ello, si bien usamos instancias generadas aleatoriamente, decidimos guardar
    estas instancias en un archivo que luego puede ser utilizado m\'ultiples
    veces o en distintos algoritmos. Como estos archivos contienen miles de
    instancias que van incrementando su tama\~no, estos archivos adquierieron
    un gran tama\~no, por lo cual se decidi\'o dejarlos a disposici\'on del
    corrector tal como se hizo con con las referencias utilizadas en este
    documento: mediante un repositorio \emph{online} de archivos. La direcci\'on
    \emph{web} del mismo puede ser encontrada dentro del directorio \emph{%
    experimentaci\'on} de los archivos provistos junto con este trabajo.

\par Por cada familia fueron creadas 10 instancias aleatorias distintas para
    cara tama\~no, para as\'i poder tener una muestra un poco m\'as amplia 
    de los distintos tipos de grafos del mismo tama\~no dentro de una misma familia
    \footnote{Consideramos como medida de tama\~no de un grafo como su cantidad
    de nodos/v\'ertices.}.

\par Hay una familia de grafos particular que merece una menci\'on aparte dada
    la manera en la que implementamos su generador. Esta es la familia de los
    grafos \emph{planares}\footnote{Claude Berge. Planar Graphs. \emph{The Theory
    of Graphs}, 21:207-2013, 1966.}. La generaci\'on de esta familia conlleva
    una mayor dificultad, m\'as a\'un, es un problema en si mismo que ha sido
    y es estudiado a\'un\footnote{\'Eric Fusy. Uniform Random Sampling Of Planar
    Graphs in Linear Time.}. Por este motivo, se decidi\'o utilizar una
    biblioteca de funciones de \emph{C++} que ya nos provee dicho generador
    de grafos aleatorios\footnote{\url{http://www.ogdf.net/}}. Sus archivos
    fuentes son provistos junto con este trabajo como as\'i tambi\'en su
    compilaci\'on fue configurada en los \emph{Makefile}'s ya mencionados en
    esta secci\'on.\footnote{La versi\'on de \emph{python} 2.7 es requerida para
    que esto funcione sin ninguna modificaci\'on.}.

\phantomsection
\addcontentsline{toc}{subsection}{Sobre la experimentaci\'on}
\subsection*{Sobre la experimentaci\'on}
\par Llegados a esta instancia del ejercicio, lo que nos importa es poder
    comprobar emp\'iricamente las conclusiones te\'oricas, particularmente
    las relacionadas con la complejidad temporal y "cuan bueno" son los
    algoritmos/heur\'isticas.

\par Entonces nos encontramos en la situaci\'on de tratar de "medir"
    la complejidad del algoritmo propuesto e implementado. Solo que aqu\'i ya no podremos
    comprobar que el resultado devuelto por el algoritmo es correcto por dos
    motivos en especial: llegaremos a un punto donde probaremos con instancias
    muy grandes como para saber a ciencia cierta cual debe ser el resultado y
    pues decidimos darle un grado importante (sino principal) de aletoriedad
    a la selecci\'on/generaci\'on de estas intancias.

\par Para medir la complejidad se decidi\'o usar el \emph{clock} del sistema,
    el mismo es un n\'umero global relacionado al hardware/sistema operativo
    que se incrementa con cada instrucci\'on ejecutada por el ordenador. La idea
    es tomar este n\'umero \footnote{%
    \url{http://www.cplusplus.com/reference/chrono/high_resolution_clock/now/}}
    justo antes e inmediatamente despu\'es de que corra nuestro algoritmo
    (dejando de lado -siguiendo las directivas del enunciado de este TP-
    todas aquellas funcionalidades encargadas de cosas ajenas, como
    por ejemplo el \emph{parseo}\footnote{%
    \url{http://es.wikipedia.org/wiki/Analizador_sint\%C3\%A1ctico}} de los datos
    de entrada/salida). Luego, alcanza con calcular la diferencia entre estos dos
    valores para saber cu\'anto tiempo insumi\'o el programa para arribar a una soluci\'on.

\par El problema de esta metodolog\'ia para medir la complejidad es que no
    toma en cuenta que junto con nuestro algoritmo se encuentra ejecut\'ando
    muchas otras aplicaciones (\emph{Winning Eleven, Mortal Kombat, etc}), las
    cuales "comparten" el tiempo de CPU. Es decir, el sistema operativo podr\'ia
    (m\'as a\'un, lo hace) desalojar nuestra tarea del procesador y pasar a
    ejecutar otras cosas. Luego ser\'a nuevamente el turno de nuestra
    aplicaci\'on, la que podr\'a terminar. Pero mientras todo esto ocurr\'ia,
    el contador global de "clock" del sistema sigui\'o increment\'andose. Es
    decir, no llegamos a una medida exacta de lo que tard\'o nuestro programa.

\par Por estos motivos decidimos, para cada instancia del problema, ejecutarla
    m\'as de una vez (5 veces, espec\'ificamente) y luego quedarnos con aquella
    ejecuci\'on que requiri\'o menos tiempo en terminar. As\'i
    conseguimos eliminar \emph{outliers} y poder decir que tenemos una medida
    lo suficientemente precisa.

\phantomsection
\addcontentsline{toc}{subsection}{Sobre el entorno de experimentaci\'on}
\subsection*{Sobre el entorno de experimentaci\'on}
\par A la hora de ejecutar los algoritmos con las \emph{suites} de testeo
    generadas aleatoriamente (y en orden de informar al lector en caso
    de querer reproducir los resultados obtenidos), se utiliz\'o la versi\'on
    de los binarios compilada con la opci\'on de \emph{multi-thread} sobre
    un \emph{HP ProLiant DL580 G5}\footnote{\url{%
    http://h20566.www2.hp.com/portal/site/hpsc/template.PAGE/action.process/public%
    /psi/learnUseDisplay/?sp4ts.oid=3454575&javax.portlet.action=true&spf_p.tpst=%
    psiContentDisplay&javax.portlet.begCacheTok=com.vignette.cachetoken&spf_p.prp_%
    psiContentDisplay=wsrp-interactionState\%3DdocId\%253Demr_na-c01154573\%257CdocLocale%
    \%253Den_US&javax.portlet.endCacheTok=com.vignette.cachetoken}} con 32GB
    de memoria RAM y 4 procesadores \emph{Intel Xeon X7460 (family 6, model 29)}
    de 6 n\'ucleos de 2.66GHZ.

\par Dicho \emph{hardware} contaba al momento de la experimentaci\'on con un 
    sistema operativo \emph{CentOS 6.4} x86\_64\footnote{\url{http://www.centos.org/}}.
