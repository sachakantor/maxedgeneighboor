\begin{pseudocodigo}[Algoritmo Exacto para \emph{CMF} - Descriptivo]
    \Require\Statex
        \begin{itemize}
            \item Un grafo $G$ de $n$ v\'ertices\footnote{Asumimos, sin p\'erdida 
                de generalidad, que los v\'ertices est\'an numerados de 1 a $n$.}
                y $m$ aristas.

            \item Una funci\'on $candidatos(K)$, que dado un conjunto de v\'ertices
                $K$, devuelve el conjunto de v\'ertices de $G$ que son adjacentes
                a todos los elementos de $K$, cuyo grado es mayor $2|K|$ y tales
                que su n\'umero de v\'ertice sea mayor al mayor de todos los elementos
                de $K$.

            \item Una funci\'on $\delta(K)$, que dada una \emph{clique} $K$\footnote{Una
                \emph{clique} no es otra cosa que un conjunto de v\'ertices, con la
                caracter\'istica de inducir un grafo completo en $G$.} en $G$, calcula
                el cardinal de su frontera\footnote{Como ya se explic\'o, sea $K$ una
                \emph{clique} de $G$, entonces: $\delta(K) = - |K|(|K|-1) +
                \displaystyle\sum_{v \in K} d(v)$}.

        \end{itemize}
    \Statex
    \Ensure Una \emph{clique} $K$ de $G$ con una frontera $\delta(K)$ de m\'axima
        cardinalidad.

    \Statex

    \If{$m = \dfrac{n(n-1)}{2}$} \Comment{Si $G$ es un $K_n$}
        \State $K \gets \left\{1;\dots;\left\lfloor\dfrac{n}{2}\right\rfloor\right\}$

    \Else
        \State $\delta_{max} \gets \left\lfloor\dfrac{r+1}{2}\right\rfloor\cdot
            \left\lceil\dfrac{r+1}{2}\right\rceil$ para $r$ tal que:
            $\begin{pmatrix}
                \text{\Huge$\bigwedge$} &
                    \begin{matrix}
                        r \in [1..n)\\[0.5cm]
                        r < \dfrac{n^2}{n^2 - 2m}\\[0.5cm]
                        (\forall r' \in [1..n))$ $r' < \dfrac{n^2}{n^2 - 2m} \implies r'\leq r
                    \end{matrix}
            \end{pmatrix}$\footnote{Esta cota inferior para la cardinalidad de la
                frontera maximal en $G$ se deriv\'o a partir del \emph{Teorema de Tur\'an},
                como ya se explic\'o anteriormente.}

        \State $K \gets \emptyset$, $K' \gets \emptyset$
        \For{$v$}{$1$}{$n$}
            \State $K' \gets \{v\}$
            \While{$K' \neq \emptyset$}\label{existe_ultimo}
                \NoEndIf{$\delta_{max} < \delta(K')$}
                    \State $K \gets K'$
                \If{$candidatos(K') \setminus YaProcesados(K') \neq \emptyset$}
                    \State Sea $v' \in candidatos(K') \setminus YaProcesados(K')$
                    \State $K' \gets K' \cup \{v'\}$
                    \State $YaProcesados(K')$\footnote{$YaProcesados(K')$ representar\'a a
                        los v\'ertices de $candidatos(K')$ que ya fueron tenidos en cuenta
                        para la \emph{clique} $K'$.}$ \gets \emptyset$

                \Else \Comment{Ac\'a est\'a el \emph{Backtracking}}
                    \State $v' \gets ultimo(K')$\footnote{Donde $ultimo(K')$
                        es el \'ulitmo v\'ertice agregado a $K'$, y sabemos que no se
                        indefine pues la guarda del ciclo \emph{mientras} de la
                        l\'inea~\ref{existe_ultimo} nos lo asegura.}

                    \State $K' \gets K' \setminus \{v'\}$
                    \State $YaProcesados(K') \gets YaProcesados(K') \cup \{v'\}$

                \EndIf
            \EndWhile
        \EndFor
    \EndIf

    \State \Return{$K$}
\end{pseudocodigo}
